
\section*{Midterm 1 Review}

\subsection*{Transistors}

\begin{center} 
\begin{tabular}[t]{|c|c|p{200px}|}
\hline
Type & Circuit Element & Behavior \\ \hline
PMOS & \begin{circuitikz}[american] 
\draw (0, 0) node[pmos] (nmos) {};
\draw (nmos.G) node[left]{$G$};
\draw (nmos.S) node[left]{$S$};
\draw (nmos.D) node[left]{$D$};
\end{circuitikz} & Closed switch: gate voltage is at least $|V_{tp}|$ below source voltage.

Open switch: otherwise. \\ \hline

NMOS & \begin{circuitikz}[american] 
\draw (0, 0) node[nmos] (nmos) {};
\draw (nmos.G) node[left]{$G$};
\draw (nmos.S) node[left]{$S$};
\draw (nmos.D) node[left]{$D$};
\end{circuitikz} & 

Closed switch: gate voltage is at least $|V_{tn}|$ above source voltage.

Open switch: otherwise. \\ \hline
\end{tabular} \end{center}

\begin{center} \begin{tabular}{|c|c|c|c|}
\hline
Type & (Voltage-Controlled) Switch & Resistor-Switch & Resistor-Capacitor-Switch \\ \hline
PMOS & \begin{circuitikz}[scale=0.9]
			\draw (0, -2) node[label=left:$D$] {}
			to[switch, l_= $V_{GS} \leq -|V_{tp}|$, *-] (0,2)
			to[short, -*] ++(0, 0.2)
            node[label=right:$S$] {};;
			\draw (0, 2) -- (-1, 2)
			to[short, -*] ++(0, -2) node[label=left:$G$] {};;
			\draw (0, -1.75) to [short, -*, l=$V_{out}$] ++(1,0);;
		\end{circuitikz} & 
        \begin{circuitikz}[scale=0.9]
			\draw (0, -2) node[label=left:$D$] {}
			to[switch,l_= $V_{GS} \leq -|V_{tp}|$, *-] (0,0)
			to[R, l_=$R_{on}$, i<=$I_D$] ++(0, 2)
			to[short, -*] ++(0, 0.2) node[label=right:$S$] {};;
			\draw (0, 2) -- (-1, 2)
			to[short, -*] ++(0, -2) node[label=left:$G$] {};;
			\draw (0, -1.75) to [short, -*, l=$V_{out}$] ++(1,0);;
		\end{circuitikz} &
        \begin{circuitikz}[scale=0.9]
			\draw (0, -2) node[label=left:$D$] {}
			to[switch,l_= $V_{GS} \leq -|V_{tp}|$, *-] (0,0)
			to[R, l_=$R_{on}$, i<=$I_D$] ++(0, 2)
			to[short, -*] ++(0, 0.2)
            node[label=right:$S$] {};;
			\draw (0, 2) -- (-1, 2)
			to[C, -*, l_=$C_{GS}$] ++(0, -2) node[label=left:$G$] {};;
			\draw (0, -1.75) to [short, -*, l=$V_{out}$] ++(1,0);;
		\end{circuitikz} \\ \hline

NMOS &  \begin{circuitikz}[scale=0.9]
            \draw (0,-2)
            to[switch,l_= $V_{GS} \geq V_{tn}$]
            (0,2) to[short, -*] ++(0,0)
            node[label=left:$D$] {};;
            \draw (0,-2) -- (-1,-2)
            to[short, -*] ++(0,2) node[label=left:$G$] {};;
            \draw (0,1.75) to [short,-*,l=$V_{out}$] ++ (1,0);;
            \draw (0, -2) to [short, -*] ++(0, -0.2) node[label=right:$S$] {};;
        \end{circuitikz} &
        \begin{circuitikz}[scale=0.9]
            \draw (0,-2)
            to[switch,l_= $V_{GS} \geq V_{tn}$]
            (0,0) to[R,-*,l=$R_{on, N}$,i<=$I_D$] ++(0,2)
            node[label=left:$D$] {};;
            \draw (0,-2) -- (-1,-2)
            to[short, -*] ++(0,2) node[label=left:$G$] {};;
            \draw (0,1.75) to [short,-*,l=$V_{out}$] ++ (1,0);;
            \draw (0, -2) to [short, -*] ++(0, -0.2) node[label=right:$S$] {};;
        \end{circuitikz} & 
        \begin{circuitikz}[scale=0.9]
            \draw (0,-2)
            to[switch,l_= $V_{GS} \geq V_{tn}$]
            (0,0) to[R,-*,l=$R_{on, N}$,i<=$I_D$] ++(0,2)
            node[label=left:$D$] {};;
            \draw (0,-2) -- (-1,-2)
            to[C,-*,l=$C_{GS}$] ++(0,2) node[label=left:$G$] {};;
            \draw (0,1.75) to [short,-*,l=$V_{out}$] ++ (1,0);;
            \draw (0, -2) to [short, -*] ++(0, -0.2) node[label=right:$S$] {};;
        \end{circuitikz} \\ \hline
\end{tabular} \end{center}

% NMOS, PMOS: knowing what happens with different gate and source potentials for both
% Switch, Resistor Switch, and RC model
% CMOS inverter: replace with RC model, get diff eqs, solve
% Time-domain circuits
% Homogenous ODE, non-homogenous ODE (change of variables), ODE with non-constant input
% RC, LR circuits: get diff eq using KCL and KVL, then solve + apply initial conditions
% Change of Basis and Diagonalization
% Multivariate ODE: get matrix-vector equation, diagonalize, solve in the eigenbasis, change back to standard basis
% Second-order ODE (using x and x' as state variables) and LRC
% Frequency Domain
% Phasors (e^{j \phi}, intuition: cos(\omega t + \phi) is the real part of e^{j(\omega t + \phi)}, get rid of the \omega t part because the circuit cannot introduce new frequencies)
% Impedance of common circuit elements(R, L, C)
% Filters (high-pass, low-pass, band-pass) and transfer functions, reading but not creating bode plots
% Resonance: transfer function for LRC circuit
