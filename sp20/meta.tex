\documentclass[11pt]{article}
\usepackage{../ee16}
\usepackage{../markup}
\usepackage{changepage}
\usepackage{tikz}
\usetikzlibrary{calc}
\usepackage{color,hyperref,listings,enumitem}
\usepackage{algorithm}
\usepackage{algpseudocode}
\usepackage{tkz-euclide}
\usepackage{physics}
\usepackage{multicol}
\usepackage{pgfplots}
\usepackage{adjustbox}
\usepackage{cancel}
\usepackage{commath}
\usepackage{mathdots}
\usepackage[american,siunitx]{circuitikz}
\sisetup{per-mode=fraction}
\sisetup{quotient-mode=fraction}
\DeclareSIUnit \decade {dec}
\lstset{basicstyle=\ttfamily}
\newcommand{\fillin}[1]{\underline{\hskip #1}}
\newcommand{\doublehrule}{\hrule \vskip 0.02in \hrule}
\newcommand*\circled[1]{\tikz[baseline=(char.base)]{
  \node[shape=circle,draw,inner sep=2pt] (char) {#1};}}

\definecolor{blueish}{rgb}{0.7,0.1,.7}
\newcommand{\sol}[1]{{\color{blueish} \textbf{Solution: } #1}} % solutions in pink
\newcommand{\ws}[1]{} % worksheet exclusive material
\newcommand{\meta}[1]{{\vspace{2pt} \color{blue} \textbf{Meta: } #1}} % for meta in blue

% eg, ie, etc, wrt, viz newcommands
\newcommand{\eg}{{\it e.g.\@}} % \@ for normal spacing after. use: \eg\/ or \eg,
\newcommand{\ie}{{\it i.e.\@}} 
\newcommand{\wrt}{{\it wrt}}
\newcommand{\viz}{{\it viz.\@}}
\newcommand{\vs}{{\it vs.\@}}
\newcommand{\aka}{{\it aka}}
\newcommand{\apriori}{{\it a priori}}
\newcommand{\etc}{\textit{etc.\@}}    
\newcommand{\bank}{../../questionBank}                                            
% etc defined so that there is only a single period after it if it is followed 
% by a period
% see: http://en.wikibooks.org/wiki/TeX/def                                      
% see: http://www.tug.org/pipermail/texhax/2004-July/002449.html                 
% see: http://en.wikibooks.org/wiki/TeX/if                                       
%\ifx \etc \undefined                                                             
%  %\def\myarg{#1}                                                                
%  %\def\myfullstop{.}                                                            
%  %\def\etc{#1}{\textit{etc} arg is ``\myarg'' \ifx\myarg\myfullstop\then.\else.#1\fi\/}
%  \def\etc#1{\textit{etc}\if#1..\else.#1\fi}                                     
%\fi                                                                              
%                        

% this should be moved to a new newcommands file
\newcommand{\ignore}[1]{}

% label and ref newcommands
% now mostly deprecated (but redefined and kept for backward compatibility) - move to \cref from the cleveref package
%\crefname{algocf}{alg.}{algs.}
%\Crefname{algocf}{Algorithm}{Algorithms}
%\crefname{equation}{}{}
%\Crefname{equation}{}{}

% this really should be in a newcommands-annotation.tex
\newcommand{\redHL}[1]{{\color{red}\bf#1}}

%DONE: xyzref{} should be able to handle multiple arguments so that you can
%reference multiple figures/equations at the same time -AG
%Now fixed with some package/command CREF.
\newcommand{\corrlabel}[1]{\label{corr:#1}}
\newcommand{\corrref}[1]{Corollary \ref{corr:#1}}
%\newcommand{\assref}[1]{Assumption \ref{ass:#1}}
\newcommand{\asslabel}[1]{\label{ass:#1}}
%\newcommand{\eqnref}[1]{(\ref{eq:#1})}
%\newcommand{\eqnlabel}[1]{\label{eq:#1}}
\newcommand{\eqnlabel}[1]{\label{#1}}
\newcommand{\eqnref}[1]{\cref{#1}}
%\newcommand{\figlabel}[1]{\label{fig:#1}}
%\newcommand{\figref}[1]{Fig.~\ref{fig:#1}}
\newcommand{\lstlabel}[1]{\label{lst:#1}}
\newcommand{\lstref}[1]{Listing~\ref{lst:#1}}
\newcommand{\figlabel}[1]{\label{#1}}
%\newcommand{\figref}[1]{\cref{#1}}
\newcommand{\chaplabel}[1]{\label{chap:#1}}
\newcommand{\seclabel}[1]{\label{sec:#1}}
%\newcommand{\secref}[1]{Sec.~\ref{sec:#1}}
%\newcommand{\chapref}[1]{Chapter~\ref{chap:#1}}
\newcommand{\exampleref}[1]{Example~\ref{ex:#1}}
\newcommand{\applabel}[1]{\label{app:#1}}
%\newcommand{\appref}[1]{Appendix~\ref{app:#1}}
\newcommand{\tablabel}[1]{\label{#1}}
%\newcommand{\tabref}[1]{\cref{#1}}
\newcommand{\thlabel}[1]{\label{th:#1}}
\newcommand{\thref}[1]{Theorem~\ref{th:#1}}
\newcommand{\thmlabel}[1]{\label{thm:#1}}
%\newcommand{\thmref}[1]{Theorem \ref{thm:#1}}
\newcommand{\deflabel}[1]{\label{def:#1}}
\newcommand{\defref}[1]{Definition~\ref{def:#1}}
\newcommand{\lemlabel}[1]{\label{lem:#1}}
%\newcommand{\lemref}[1]{Lemma \ref{lem:#1}}
\newcommand{\examplelabel}[1]{\label{ex:#1}}
\newcommand{\labexample}[1]{\label{example:#1}}
\newcommand{\refexample}[1]{Example{ }\ref{example:#1}}
\newcommand{\httpref}[1]{}
\let\el=\eqnlabel
\let\er=\eqnref

%%%%%% cross references (moved here from ee16.sty)
% \newcommand{\chapref}[1]{Chapter~\ref{#1}}
% \newcommand{\secref}[1]{Section~\ref{#1}}
% \newcommand{\figref}[1]{Figure~\ref{#1}}
% \newcommand{\tabref}[1]{Table~\ref{#1}}
% \newcommand{\exref}[1]{Exercise~\ref{#1}}
% %\newcommand{\eqref}[1]{Equation~(\ref{#1})}
% \newcommand{\partref}[1]{Part~\ref{#1}}
% \newcommand{\appref}[1]{Appendix~\ref{#1}}
% \newcommand{\sideref}[1]{the sidebar titled #1}
% \newcommand{\pgref}[1]{page~\pageref{#1}}


%\newcommand{\ddt}[2]{{\frac{d}{d #2}\left. #1 \right.}}
\newcommand{\ddt}[2]{{\frac{d #1}{d #2}}}
\newcommand{\dddt}[2]{{\frac{d^2 #1}{d #2 ^2}}}
\newcommand{\ddx}[2]{\frac{\partial{#1}}{\partial{#2}}}
\newcommand{\Ddx}[2]{{\frac{D}{d{#2}}\left[ #1 \right]}}
\newcommand{\qhat}{\hat{q}}
\newcommand{\Qhat}{\hat{Q}}
\newcommand{\fhat}{\hat{f}}
\newcommand{\Fhat}{\hat{F}}
\newcommand{\deq}{\triangleq}
%\newcommand{\I}{I\left[\right]}
\newcommand{\ELL}[1]{L\left[#1\right]}
\newcommand{\w}{\omega}
\newcommand{\calD}{{\cal D}}
\renewcommand{\jmath}{j}
\newcommand{\Pinvt}{{P(t)}^{-1}}
\newcommand{\halmos}{\hskip\textwidth minus\textwidth \rule{6pt}{6pt}}
\newcommand{\calA}{{\cal A}}
\newcommand{\calAT}{{\cal A}^T}
\newcommand{\calL}{{\cal L}}
\newcommand{\calLadj}{\calL^\dagger}
\newcommand{\calR}{{\cal R}}
\newcommand{\Reals}{\Bbb{R}}
\newcommand{\Real}{\Bbb{R}}
\newcommand{\Safe}{\ensuremath{{\bf S}}}
\newcommand{\Pade}{{Pad\'e}}
\newcommand{\Complexes}{\Bbb{C}}
\newcommand{\HvecqLanc}{\Hvec^{(L)}_q}
\newcommand{\HvecqArn}{\Hvec^{(A)}_q}
\newcommand{\Fvec}{\vec{F}}
\newcommand{\calDT}{{\cal D}^T}
\newcommand{\calCtil}{\tilde{\cal C}}
\newcommand{\calJtil}{\tilde{\cal J}}
\newcommand{\calDtil}{\tilde{\cal D}}
\newcommand{\JstarDphi}{J_{\Delta \phi}^*}
\newcommand{\Jstarphi}{J_{\phi}^*}
\newcommand{\vecbphi}{\vec b_{\phi}}
\newcommand{\vecbext}{\vec b_{\text{ext}}}
\newcommand{\phistarIdeal}{\phi_{\text{ideal}}^*}
\newcommand{\apdx}[1]{\appref{#1}} 
\newcommand{\delbydel}[2]{\ensuremath{\frac{\partial {#1}}{\partial {#2}}}}
\newcommand{\delbydeln}[3]{\ensuremath{\frac{\partial^{{#1}}
{#2}}{\partial{#3}^{{#1}}}}}
\newcommand{\delbydelat}[3]{\ensuremath{\left. \frac{\partial
{#1}}{\partial {#2}} \right|_{{#3}}}}
\newcommand{\dee}{\ensuremath{\mathrm{d}}}
\newcommand{\dbyd}[2]{\ensuremath{\frac{\mathrm{d} {#1}}{\mathrm{d}
{#2}}}}
\newcommand{\dbydn}[3]{\ensuremath{\frac{\mathrm{d}^{{#1}}{#2}}
{\mathrm{d}{#3}^{{#1}}}}}
\newcommand {\dist}[1] {{{\rm dist}\left({#1}\right)}}
\def\Matrix#1{\begin{pmatrix} #1 \end{pmatrix}}
\newcommand{\textdiag}{{\text{diag}}}
\newcommand{\ihat}{\hat{\imath}}
\newcommand{\jhat}{\hat{\jmath}}
\newcommand{\khat}{\hat{k}}
%\newcommand{\del}{\nabla}
\newcommand{\Rxx}{R_{\vec x \vec x}}
\newcommand{\Sxx}{S_{\vec x \vec x}}
\newcommand{\Snn}{S_{\vec n \vec n}}
\newcommand{\vecphi}{\vec \phi}
\newcommand{\Dphi}{\Delta \phi}
\newcommand{\vecDphi}{\vv{\Delta \phi}}
\newcommand{\dphi}{\delta \phi}
\newcommand{\vecdphi}{\vv{\delta \phi}}
\newcommand{\ah}{\alpha_g} % g for group
%\newcommand{\tDagger}{{\tilde t}}
\newcommand{\tDagger}{{t^{\!\dagger}}}
%\newcommand{\norm}[1]{\left\lVert#1\right\rVert}
%\newcommand{\abs}[1]{\left\lvert#1\right\rvert}
\newcommand{\texthyphen}{\text{--}}
\newcommand{\vect}[2]{ \vec{#1} \left[ #2 \right] }
\newcommand{\vecht}[2]{ \vect{\hat{#1}}{#2} }
\newcommand{\pp}[1]{\left( #1 \right)}

%The substack command can be used to produce a multi-line subscripts or superscripts:
%\sum_{\substack{0\le i\le m\\ 0<j<n}} P(i,j) produces a two-line subscript underneath the sum.
%A slightly more general form is the subarray environment which allows you to specify that each line should be left-aligned instead of centered, as here:
% \sum_{\begin{subarray}{l} i\in\Lambda\\ 0<j<n \end{subarray}} P(i,j) 

%\newcommand{\be}[1]{\begin{equation}\eqnlabel{#1}} % now moved to cleveref
\newcommand{\be}[1]{\begin{equation}\label{#1}}
\def\ee{\end{equation}}

\def\endproofmark{$\Box$}
\newenvironment{proof}{\par{\bf Proof}:}{\endproofmark}
\newenvironment{proofof}{\par{\bf Proof }}{\endproofmark}


\newcounter{axiomnumber}
\setcounter{axiomnumber}{0}
\renewcommand\theaxiomnumber{\the\lecturenumber.\arabic{axiomnumber}}
\newenvironment{axiom}[1]{\par\refstepcounter{axiomnumber}
{\bf Axiom \theaxiomnumber\ (#1)}:
\begingroup}%
{\endgroup}

\newcounter{defnnumber}
\setcounter{defnnumber}{0}
\renewcommand\thedefnnumber{\the\lecturenumber.\arabic{defnnumber}}
\newenvironment{defn}[1]{\par\refstepcounter{defnnumber}
{\bf Definition \thedefnnumber\ (#1)}:
\begingroup}%
{\endgroup}

\newcounter{theoremnumber}
\setcounter{theoremnumber}{0}
\renewcommand\thetheoremnumber{\the\lecturenumber.\arabic{theoremnumber}}
\newenvironment{theorem}{\par\refstepcounter{theoremnumber}
{\bf Theorem \thetheoremnumber}:
\begingroup}%
{\endgroup}

\newcounter{lemmanumber}
\setcounter{lemmanumber}{0}
\renewcommand\thelemmanumber{\the\lecturenumber.\arabic{lemmanumber}}
\newenvironment{lemma}{\par\refstepcounter{lemmanumber}
{\bf Lemma \thelemmanumber}:
\begingroup}%
{\endgroup}

\newenvironment{corollary}{\par\refstepcounter{theoremnumber}
{\bf Corollary \thetheoremnumber}:
\begingroup}%
{\endgroup}

%%%%%% additional symbols or names for them

%\newcommand{\implies}{\:\;{\Rightarrow}\:\;}
\newcommand{\impliessymbol}{\Rightarrow}
\newcommand{\entails}{\models}
\newcommand{\lequiv}{\;\;{\Leftrightarrow}\;\;}
\newcommand{\lequivsymbol}{\Leftrightarrow}
\newcommand{\xor}{\not\lequiv}
\newcommand{\All}[1]{\forall\,#1\;\;}
\newcommand{\Exi}[1]{\exists\,#1\;\;}
\newcommand{\Exii}[1]{\exists!\,#1\;\;}% -pnorvig
\newcommand{\Iot}[2]{\iota\,#1\,#2}
\newcommand{\Lam}[2]{\lambda #1\;#2}
\newcommand{\Qua}[3]{[#1\,#2\;#3]}

\def\<{\langle}
\def\>{\rangle}

\newcommand{\union}{{\,{\cup}\,}}
\newcommand{\elt}{{\,{\in}\,}}  %%%cuts down on spacing
\newcommand{\eq}{{\,{=}\,}}     %%%cuts down on spacing
\def\stimes{{\,\times\,}}       %%%cuts down on spacing

%\def\ceil#1{\lceil #1 \rceil}
%\def\floor#1{\lfloor #1 \rfloor}
\DeclarePairedDelimiter\ceil{\lceil}{\rceil}
\DeclarePairedDelimiter\floor{\lfloor}{\rfloor}
\DeclarePairedDelimiterX{\innp}[2]{\langle}{\rangle}{#1, #2}
%\DeclareMathOperator*{\argmin}{argmin}
\newcommand*{\argmin}{\operatornamewithlimits{argmin}\limits} % moved to ee16.sty
\newcommand*{\argmax}{\operatornamewithlimits{argmax}\limits} % moved to ee16.sty

%\def\cents{\mbox{c}}
%\AtBeginDocument{\def\cents{\hbox{\rm\rlap{\hspace{.07ex}$/$}c}}}
\newcommand{\cents}{\text{\textcent{}}}
\def\cons{{}\bullet{}}

%%%%%% bold font in math mode; this sucks but is simplest for now
\newcommand{\mbf}[1]{\mbox{{\bfseries #1}}}
\newcommand{\smbf}[1]{\mbox{{\scriptsize\bfseries #1}}}

\def\N{\mathbb{N}}
\def\R{\mathbb{R}}
\def\C{\mathbb{C}}
\def\X{\mbf{X}}
\def\x{\mbf{x}}
\def\sx{\smbf{x}}
\def\Y{\mbf{Y}}
\def\y{\mbf{y}}
\def\sy{\smbf{y}}
\def\E{\mbf{E}}
\def\e{\mbf{e}}
\def\T{\mbf{T}}
\def\O{\textrm{O}}  % repeated below because it gets redefined by some package?
\def\Q{\mathbb{Q}}
\def\se{\smbf{e}}
\def\Z{\mathbb{Z}}
\def\z{\mbf{z}}
\def\sz{\smbf{z}}
\def\F{\mathbb{F}}
\def\f{\mbf{f}}
\def\A{\mbf{A}}
\def\P{\mbf{P}}
\def\B{\mbf{B}}
\def\b{\mbf{b}}
\def\m{\mbf{m}}
\def\I{\mbf{I}}
\def\ones{\mbf{1}}
\def\ev{\mbf{ev}}
\def\fv{\mbf{ev}}
\def\sv{\mbf{sv}}
\def\e{\mathop{\mathrm{e}}\nolimits}  % for e = 2.718...

\def\third{{\textstyle{1\over 3}}}
\def\half{{\textstyle{1\over 2}}}
\def\quarter{{\textstyle{1\over 4}}}
\def\VarOmega{\mathchar"10A }
\def\Ex#1{{\mathbb E}(#1)}
\def\Var#1{{\rm Var}(#1)}
\def\Aset{{\cal A}}
\def\Bset{{\cal B}}

\def\Bin{\text{Bin}}
\def\Geom{\text{Geom}}
\def\Poiss{\text{Poiss}}
\def\Exp{\text{Exp}}
\def\Norm{N}

\newcommand{\mymod}{\textup{mod}}

\def\mdw@dots#1{\ensuremath{\mathpalette\mdw@dots@i{#1}}}
\def\mdw@dots@i#1#2{%
  \setbox\z@\hbox{$#1\mskip1.8mu$}%
  \dimen@\wd\z@%
  \setbox\z@\hbox{$#1.$}%
  #2%
}
\def\Ddots{%
  \mdw@dots{\mathinner{%
    \mkern1mu%
    \raise\dimen@\vbox{\kern7\dimen@\copy\z@}%
    \mkern2mu%
    \raise4\dimen@\copy\z@%
    \mkern2mu%
    \raise7\dimen@\box\z@%
    \mkern1mu%
  }}%
}

\newcommand{\circulantmatrix}[1]{
\begin{bmatrix}
    #1[0] & #1[N - 1] & #1[N - 2] & \dots & #1[1] \\
    #1[1] & #1[0] & #1[N - 1] & \dots & #1[2] \\
    #1[2] & #1[1] & #1[0] & \dots & #1[3] \\
    \vdots & \vdots & \vdots & \ddots & \vdots \\
    #1[N-1] & #1[N - 2] & #1[N - 3] & \dots & #1[0] \\
\end{bmatrix}
}






\begin{document}

\def\title{Worksheet 2}

\newcommand{\qitem}{\qpart\item}

\renewcommand{\labelenumi}{(\alph{enumi})} % change default enum format to (a)
\renewcommand{\theenumi}{(\alph{enumi})} % fix reference format accordingly.
\renewcommand{\labelenumii}{\roman{enumii}.} % second level labels.
\renewcommand{\theenumii}{\roman{enumii}.}

\maketitle

\vspace{0.5em}

\begin{qunlist}

% Authors: Justin Yu
% Email: justinvyu@berkeley.edu

\qns{Introduction to Inductors}

\meta {
    This problem is meant to be more practice for solving differential equations with constant inputs. Make sure students are comfortable with this and give them time to work it out on their own.
}

Now that we are comfortable solving for the transient behavior of charging and discharging capacitors, we can move to analyzing a new circuit element: the inductor. An inductor has the physical property of \textit{inductance}, represented by a constant $L$. Inductors are characterized by the following I-V relationship:

\vspace{-15px}

\begin{align}
V_L(t) = L \frac{d}{dt} i_L(t)
\end{align}

Let's analyze the following LR circuit, with the initial condition $i_L(0) = 0$ with the switch open before $t = 0$:

\begin{center}
    \begin{circuitikz}[scale=0.8]
        \draw (-1,4)
        to [V = $V_{in}$] (-1,0)
        (-1,4) to [switch, l^=\mbox{$t = 0$} ] (1,4)
        (1,4) to [L = $L$,i=$i_L(t)$, v = $V_L(t)$] (4,4)
        (4,4) to [short] (6,4)
        to [R = $R$, v = $V_{out}$] (6,0)
        to [short] (-1,0);
    \end{circuitikz}
\end{center}

\begin{enumerate}

\qitem \textbf{Write out the differential equation for the current $i_L(t)$ of the inductor starting at $t = 0$ when the switch is closed.}

\ws{
\vspace{3em}
}

\sol{
    KCL and Ohm's law gives us:
    \begin{align*}
        V_{in} - V_{out} &= V_L(t) = L \frac{d}{dt} i_L(t) \\
        V_{out} &= R i_L(t) \\
        \implies L \frac{d}{dt} i_L(t) &= V_{in} - R i_L(t) \\
        \implies \frac{d}{dt} i_L(t) &= -\frac{R}{L}i_L(t) + \frac{V_{in}}{L}
    \end{align*}
}

\qitem \textbf{Solve the differential equation for $i_L(t)$.}

\ws{
\vspace{50px}
}

\meta {
Emphasize that this is another nonhomogenous differential equation similar to the discharging capacitor.
It may help to review the general form of a nonhomogenous differential equation.
}

\sol {
    \begin{align*}
        \frac{d}{dt} i_L(t) &= -\frac{R}{L}i_L(t) + \frac{V_{in}}{L} \\
        \implies i_L(t) &= \frac{V_{in}}{R} (1 - e^{-\frac{R}{L}t})
    \end{align*}
}

\qitem \textbf{What is the steady-state current through the inductor as $t \rightarrow \infty$?} Sketch a plot of the current through the inductor over time, labeling the asymptote after reaching the steady-state.
This should provide you with some intuition as to the physical behavior of an inductor once an inductor is at steady state.

\ws{
\vspace{3em}
}

\sol {
    \begin{align*}
        \lim_{t \rightarrow \infty} i_L(t) &= \lim_{t \rightarrow \infty} \frac{V_{in}}{R} (1 - e^{-\frac{R}{L}t}) \\
        &= \frac{V_{in}}{R} (1 - 0) \\
        &= \frac{V_{in}}{R}
    \end{align*}

    The physics of the inductor opposes current flow initially, but it approaches the steady-state current as determined by the rest of the circuit (the resistor and voltage source).
}

\qitem \textbf{What is the steady-state voltage drop across the inductor as $t \rightarrow \infty$?}

\ws{
\vspace{3em}
}

\sol {
    \begin{align*}
        \lim_{t \rightarrow \infty} V_L(t) &= \lim_{t \rightarrow \infty} L \frac{d}{dt} i_L(t)  \\
        &= \lim_{t \rightarrow \infty} L (\frac{d}{dt} \frac{V_{in}}{R} (1 - e^{-\frac{R}{L}t})) \\
        &= \lim_{t \rightarrow \infty} L \frac{V_{in}}{R} \frac{R}{L} e^{-\frac{R}{L}t} \\
        &= \lim_{t \rightarrow \infty} V_{in} e^{-\frac{R}{L}t} \\
        &= 0
    \end{align*}
}

\qitem \textbf{What circuit element does the inductor act like at steady-state?}

\sol{It acts like a wire element, since the voltage drop across the inductor goes to 0.}

\end{enumerate}

\newpage
\qns{Multivariate ODE with Coordinate Changes}

\meta{
The general procedure for solving this type of problem is: First, convert to the eigenbasis. Solve the problem there. Then convert back to the problem basis to find the final answer.
}

\begin{enumerate}

\qitem Consider a system of differential equations (valid for $t\geq 0$)
\begin{equation}
\frac{d}{dt}x_1(t) = 5 x_1(t) + 2 x_2(t)
\end{equation}
\begin{equation}
\frac{d}{dt}x_2(t) = -8 x_1(t) -5 x_2(t)
\end{equation}

with initial conditions $x_1(0) = 3$ and $x_2(0) = 3$.

\textbf{Write out the differential equations and initial conditions in matrix/vector form.}

\ws{
	\vspace{120px}
}

\meta{
	When we say differential matrix, we're referring to a matrix that performs the act of differentation on the state variable $\vec{x}.$
}

\sol{
	We define our state variable $\vec{x}(t) = \begin{bmatrix}x_1(t) \\ x_2(t)\end{bmatrix}.$
	$$\ddt{}{t} \vec{x}(t) = \begin{bmatrix}\frac{d}{dt}x_1(t) \\ \frac{d}{dt}x_2(t)\end{bmatrix} = \begin{bmatrix}5 & 2 \\ -8 & -5\end{bmatrix}\begin{bmatrix}x_1(t) \\ x_2(t)\end{bmatrix} = \begin{bmatrix}5 & 2 \\ -8 & -5\end{bmatrix} \vec{x}(t)$$

	The initial condition is:
    $$\vec{x}(0) = \begin{bmatrix}x_1(0) \\ x_2(0)\end{bmatrix} =\begin{bmatrix}3 \\ 3\end{bmatrix} $$

    We will define the differential matrix as $A$, where

    $$A = \begin{bmatrix}5 & 2 \\ -8 & -5\end{bmatrix} $$
}

% \bigskip

% \begin{adjustwidth}{-20pt}{0pt}
% 	We already know how to solve the system of differential equations if $\frac{d}{dt}y_1(t)$ only depends on $y_1(t)$ and $\frac{d}{dt}y_2(t)$ only depends on $y_2(t)$.
% 	However, we can't directly solve a system of ODEs where $\frac{d}{dt}y_1(t)$ and $\frac{d}{dt}y_2(t)$ each depend on both $y_1(t)$ and $y_2(t)$. \\
% 	The solution? Change coordinates to the eigenbasis to diagonalize our transformation matrix.\\
% 	Then, we will have $\frac{d}{dt}z_{\lambda_1}(t) = \lambda_1 z_{\lambda_1}(t)$ and $\frac{d}{dt}z_{\lambda_2}(t) = \lambda_2 z_{\lambda_2}(t)$, which we know how to solve.

% \end{adjustwidth}


\qitem \textbf{Find the eigenvalues $\lambda_1, ~\lambda_2$ and eigenspaces for the differential equation matrix above.}

\ws{
	\vspace{150px}
}

\sol{
	In order to find the eigenvalues of $A,$ we look at the determinant of $A - \lambda I.$
	 $$\text{det}\left( \begin{bmatrix}5-\lambda & 2 \\ -8 & -5-\lambda\end{bmatrix} \right) = (-5-\lambda)(5-\lambda) + 16 = \lambda^2 - 9 = 0$$
	Therefore, we see that
	 $$ \lambda_1 = -3, \lambda_2 = 3$$

	We can find the eigenspaces by looking at the null-spaces of $A - \lambda I.$

	For $\lambda_1 = -3,$
	$$ (A + 3I) \vec v_{1}= \begin{bmatrix} 5 - (-3) & 2 \\ -8 & -5 - (-3) \end{bmatrix} \vec v_{1} = \begin{bmatrix}0 \\ 0\end{bmatrix}$$
    $$ \begin{bmatrix} 8 & 2 \\ -8 & -2 \end{bmatrix} \vec v_{1} = \begin{bmatrix}0 \\ 0\end{bmatrix}$$
    $$\vec v_{1} = \begin{bmatrix} -1 \\ 4\end{bmatrix} $$

	For $\lambda_1 = 3,$
	$$ (A - 3I) \vec v_{2} = \begin{bmatrix} 5 - 3 & 2 \\ -8 & -5 - 3 \end{bmatrix} \vec v_{\lambda_2} = \begin{bmatrix}0 \\ 0\end{bmatrix}$$
	$$ \begin{bmatrix} 2 & 2 \\ -8 & -8 \end{bmatrix} \vec v_{2} = \begin{bmatrix} 0 \\ 0\end{bmatrix}$$
	$$\vec v_{2} = \begin{bmatrix} -1 \\ 1\end{bmatrix} $$
}


\qitem
\textbf{Change coordinates into the eigenbasis to re-express the differential equations in terms of new variables $z_{1}(t), ~
z_{2}(t)$.}

Let $\vec{z}(t) = \begin{bmatrix} z_1(t) \\ z_2(t) \end{bmatrix}$ represent the vector $\vec{x}$ using the eigenbasis for its coordinate representation.
\textit{Find a matrix $\widetilde{A}$ such that $\frac{d}{dt} \vec{z}(t) = \widetilde{A} \vec{z}(t)$. Don't forget about the initial conditions.}

\ws{
	\vspace{200px}
}

\meta {
	$x_i$ coordinates are standard coordinates, while the $z_i$ coordinates are using the eigenbasis for representation.
}

\sol{
	\begin{centering}
	\begin{tikzpicture}

	\draw (-1,0) node[anchor = east] {$z_{i}$ coordinates};
	\draw (0,0) circle (0.5cm);
	\draw (-1,2) node[anchor = east] {$x_{i}$ coordinates};
	\draw[->] (0,0.5) -- (0,1.5) node[anchor=north east] {$V$};
	\draw (0,2) circle (0.5cm);
	\draw[->] (0.5,2) -- (4.5,2) node[anchor=south east] {$ A = \begin{bmatrix}5 & 2 \\ -8 & -5\end{bmatrix}$} ;
	\draw (3,2) node[anchor=north] {differentiation};
	\draw (5,2) circle (0.5cm);
	\draw[->] (5,1.5) -- (5,0.5) node[anchor=south west] {$V^{-1}$};
	\draw (5,0) circle (0.5cm);
	\draw[dashed,->] (0.5,0) -- (4.5,0) node[anchor=south east] {$ \widetilde{A} =V^{-1} A V$} ;

	\end{tikzpicture}
\end{centering}

$$\vec x = z_1 \vec v_{1} + z_2 \vec v_{2}$$
$$\vec x =\begin{bmatrix} -1 & -1 \\ 4 & 1\end{bmatrix}\begin{bmatrix}z_{1} \\ z_{2}\end{bmatrix}$$

We can define the change-of-coordinates matrix from the eigenbasis to our original basis as

$$V=\begin{bmatrix} -1 & -1 \\ 4 & 1\end{bmatrix}$$

Changing coordinates to the eigenbasis:

$$\begin{bmatrix}z_{1} \\ z_{2}\end{bmatrix} = V^{-1}\begin{bmatrix} x_1 \\ x_2\end{bmatrix}   $$

$$V^{-1}=\begin{bmatrix}\frac{1}{3} & \frac{1}{3} \\ -\frac{4}{3} & -\frac{1}{3}\end{bmatrix}$$

$$\widetilde{A} = V^{-1} A V = \begin{bmatrix}\frac{1}{3} & \frac{1}{3} \\ -\frac{4}{3} & -\frac{1}{3}\end{bmatrix}\begin{bmatrix}5 & 2 \\ -8 & -5\end{bmatrix}\begin{bmatrix} -1 & -1 \\ 4 & 1\end{bmatrix}$$
% $$A_{z_\lambda} = \begin{bmatrix}\frac{2}{3} & -\frac{1}{3} \\ \frac{1}{3} & \frac{1}{3}\end{bmatrix}\begin{bmatrix}-5 & -2 \\ 5 & -4\end{bmatrix}$$
$$\widetilde{A} = \begin{bmatrix}-3 & 0 \\ 0 & 3\end{bmatrix}$$

That is:

	$$\ddt{}{t} \vec{z}(t) = \begin{bmatrix}\frac{d}{dt} z_{1}(t) \\ \frac{d}{dt}z_{2}(t)\end{bmatrix} = \begin{bmatrix}-3 & 0 \\ 0 & 3\end{bmatrix}\begin{bmatrix}z_{1}(t) \\ z_{2}(t)\end{bmatrix}$$

And our initial condition is:

$$\vec z_{\lambda}(0) = \begin{bmatrix}\frac{1}{3} & \frac{1}{3} \\ -\frac{4}{3} & -\frac{1}{3}\end{bmatrix}\begin{bmatrix} 3 \\ 3 \end{bmatrix} = \begin{bmatrix} 2 \\ -5 \end{bmatrix} $$
}

\qitem \textbf{Solve the differential equation for $z_{1}(t)$, $z_{2}(t)$ in the eigenbasis.}

\ws{
	\vspace{150px}
}

\sol{
Our initial condition is $$\vec{z}(0) = \begin{bmatrix} 2 \\ -5 \end{bmatrix}$$

We can unroll our system of equations to get:

$$\ddt{}{t} z_{1}(t) = -3 z_{1}(t)$$
$$\ddt{}{t} z_{2}(t) = 3 z_{2}(t)$$

From previous differential equation experience, we see that the solution to $z_{i}(t)$ is:

$$z_{1}(t) = 2e^{-3t}$$
$$z_{2}(t) = -5e^{3t}$$
}

\qitem \textbf{Convert your solution back into the original coordinates to find $x_1(t)$, $x_2(t)$.}

\sol{
$$ \vec x(t) = V \vec z(t) =  \begin{bmatrix} -1 & -1 \\ 4 & 1\end{bmatrix}\begin{bmatrix} 2 e^{-3t} \\ -5 e^{3t}  \end{bmatrix} = \begin{bmatrix} -2e^{-3t} + 5 e^{3t}\\ 8e^{-3t}-5e^{3t} \end{bmatrix}$$
}


% \qitem We can solve this equation using a slightly shorter approach by observing that the solutions for $y_i(t)$ will all be of the form
% $$y_i(t) = \sum_k K_{i,k} e^{\lambda_k t}$$

% where $\lambda_k$ is an eigenvalue of our differential equation
% relation matrix and the $K_{i,k}$ are constants derived from our
% initial conditions and the coordinate changes involved.

% Since we have observed that the solutions will include
% $e^{\lambda_i t}$ terms, once we have found the eigenvalues for our
% differential equation matrix, we can guess the forms of the $y_i(t)$ as

%  	$$\begin{bmatrix}y_1(t) \\ y_2(t)\end{bmatrix}
%         = \begin{bmatrix}\alpha e^{\lambda_1t} + \beta e^{\lambda_2t}
%           \\ \gamma e^{\lambda_1 t}  + \kappa e^{\lambda_2 t} \end{bmatrix}$$
% where $\alpha, ~\beta, ~\gamma, ~\kappa$ are all constants.

% \begin{enumerate}
% 	\qitem Take the derivative to write out
% 	$$\begin{bmatrix}\frac{d}{dt}y_1(t) \\ \frac{d}{dt}y_2(t)\end{bmatrix}.$$ in matrix-vector form.

% 	\sol{
% 	 	$$\begin{bmatrix}y_1(t) \\ y_2(t)\end{bmatrix} = \begin{bmatrix}\alpha e^{-3t} + \beta e^{3t}  \\ \gamma e^{-3t}  + \kappa e^{3t} \end{bmatrix}$$
% 		$$\frac{d}{dt} \vec y(t) = \begin{bmatrix}-3\alpha e^{-3t} +3 \beta e^{3t}  \\ -3\gamma e^{-3t} + 3 \kappa e^{3t} \end{bmatrix}$$
% 		If we notice that the right-hand side can be written as linear combinations of $e^{-3t}$ and $e^{3t}$, we can write the previous equation as:
% 		$$ \frac{d}{dt} \vec y(t) =
% 		\begin{bmatrix}
% 			-3 \alpha & 3 \beta  \\ -3 \gamma & 3 \kappa
% 		\end{bmatrix}
% 		\begin{bmatrix}
% 			e^{-3t} \\ e^{3t}
% 		\end{bmatrix}
% 		$$
% 	}

% 	\qitem Connect this differential equation to the matrix-vector equation you found in part (a).\\
% 	\sol{
% 		$$\frac{d}{dt} \vec y(t) =
% 		\begin{bmatrix}
% 			5 & 2 \\
% 			-8 & -5
% 		\end{bmatrix}
% 		\vec{y}(t) =
% 		\begin{bmatrix}
% 			-3 \alpha & 3 \beta  \\ -3 \gamma & 3 \kappa
% 		\end{bmatrix}
% 		\begin{bmatrix}
% 			e^{-3t} \\ e^{3t}
% 		\end{bmatrix}
% 		$$
% 		Substituting
% 		$$ \begin{bmatrix}
% 			\alpha e^{-3t} + \beta e^{3t}  \\ \gamma e^{-3t}  + \kappa e^{3t}
% 		\end{bmatrix} =
% 		\begin{bmatrix}
% 			\alpha & \beta  \\ \gamma & \kappa
% 		\end{bmatrix}
% 		\begin{bmatrix}
% 			e^{-3t} \\ e^{3t}
% 		\end{bmatrix}
% 		$$
% 		for $\vec{y}(t)$, we get:
% 		$$
% 		\begin{bmatrix}
% 			5 & 2 \\
% 			-8 & -5
% 		\end{bmatrix}
% 		\begin{bmatrix}
% 			\alpha & \beta  \\ \gamma & \kappa
% 		\end{bmatrix}
% 		\begin{bmatrix}
% 			e^{-3t} \\ e^{3t}
% 		\end{bmatrix} =
% 		\begin{bmatrix}
% 			-3 \alpha & 3 \beta  \\ -3 \gamma & 3 \kappa
% 		\end{bmatrix}
% 		\begin{bmatrix}
% 			e^{-3t} \\ e^{3t}
% 		\end{bmatrix}
% 		$$
% 		Doing the matrix multiplication on the left-hand side of the equation, we get:
% 		$$
% 		\begin{bmatrix}
% 			5 \alpha + 2 \gamma & 5 \beta + 2 \kappa \\
% 			-8 \alpha - 5 \gamma & -8 \beta - 5 \kappa
% 		\end{bmatrix}
% 		\begin{bmatrix}
% 			e^{-3t} \\ e^{3t}
% 		\end{bmatrix} =
% 		\begin{bmatrix}
% 			-3 \alpha & 3 \beta  \\ -3 \gamma & 3 \kappa
% 		\end{bmatrix}
% 		\begin{bmatrix}
% 			e^{-3t} \\ e^{3t}
% 		\end{bmatrix}
% 		$$
% 	}

% 	\qitem Use what you found in the previous step to solve for $\gamma$ and $\kappa$ in terms of $\alpha$ and $\beta$, respectively. \\
% 	\sol {
% 		Equating terms in the matrices on the left- and right-hand sides of the equation, we get:
% 		$5 \alpha + 2 \gamma = -3 \alpha$,
% 		$5 \beta + 2 \kappa = 3 \beta$,
% 		$-8 \alpha - 5 \gamma = -3 \gamma$, and
% 		$-8 \beta - 5 \kappa = 3 \kappa$.

% 		From the first equation, we get $\gamma = -4 \alpha$. From the second, we get $\kappa = - \beta$. \\
% 		We could get the same result from using the third and fourth equations.
% 	}

% 	\qitem Use initial conditions to finish solving for $\vec{y}(t)$. \\
% 	\sol {
% 		From the initial condition, we have:
% 		$$\vec{y}(0) =
% 		\begin{bmatrix} 3 \\ 3 \end{bmatrix} =
% 		\begin{bmatrix}
% 			\alpha & \beta  \\ \gamma & \kappa
% 		\end{bmatrix}
% 		\begin{bmatrix}
% 			e^{-3 \cdot 0} \\ e^{3 \cdot 0}
% 		\end{bmatrix} =
% 		\begin{bmatrix}
% 			\alpha & \beta  \\ \-4 \alpha & - \beta
% 		\end{bmatrix}
% 		\begin{bmatrix}
% 			1 \\ 1
% 		\end{bmatrix}$$
% 		From this, we get the equations $\alpha + \beta = 3$ and $-4 \alpha - \beta = 3$.
% 		$$\alpha=-2$$
% 	 	$$\beta=5$$

% 		We can now plug in our 4 constants to find $\vec{y}(t)$:
% 		$$\begin{bmatrix}y_1(t) \\ y_2(t)\end{bmatrix} = \begin{bmatrix} -2e^{-3t} + 5e^{3t}  \\ 8e^{-3t}  -5  e^{3t} \end{bmatrix}$$
% 		Note that this is the same as the answer from part (e)
% 	}
% \end{enumerate}

\end{enumerate}

\newpage
% {\Large \textbf{Mechanical:}}
\qns{Complex Numbers}

A complex number, $z$, is composed of a real part and imaginary part.
If $z = a + bj$, then $re(z) = a$ (the real portion equals a), and $im(z) = b$ (the imaginary portion equals b).
Complex numbers can be expressed in two ways:

\begin{center}
Rectangular Form: $z = a + bj$ \hspace{1em} Polar Form: $z = re^{j\theta}$
\end{center}

In polar form, $r$ represents the magnitude and $\theta$ represents the angle of the complex number with respect to the origin of the complex plane.
Rectangular form makes adding and subtracting complex numbers easier; whereas, polar form makes multiplying and dividing numbers easier.
Some handy equations to switch between forms include:

\begin{center}
\begin{tabular}{ c c c }
 $tan(\theta) = \frac{b}{a}$ & $r = |z| = \sqrt{a^2 + b^2}$ \\ \\
 $sin(\theta) = \frac{b}{|z|}$ & $cos(\theta) = \frac{a}{|z|}$ \\  \\
\end{tabular}
\end{center}

\begin{enumerate}

\qitem Prove algebraically that $\frac{1}{j} = -j$.

\sol{
The key is to multiply the left-hand side of the equation by $\frac{j}{j}$: \\
$$\frac{1}{j} = \frac{1 * j}{j * j} = \frac{j}{j^2}$$
$$= \frac{j}{-1} = -j$$
}

\end{enumerate}

A complex number, $z = a + bj$ has a complex conjugate, $\overline{z} = a - bj$.
Note that the sum of a complex number and its conjugate is always real, but the difference between a complex number and its conjugate is always imaginary.

\begin{enumerate}[resume]

\qitem Use a polar graph to show that the sum of any complex number and its conjugate is always real.

\sol{

}

\qitem Recall that Euler's Formula states that $e^{j\theta} = cos(\theta) + jsin(\theta)$.
Using Euler's identity, show that $cos(\theta) = \frac{1}{2}(e^{j\theta} + e^{-j\theta})$.

\sol{

  $$e^{j\theta} = cos(\theta) + jsin(\theta)$$

  Note that $e^{j\theta}$ has the complex conjugate $e^{-j\theta}$, which means:

  $$e^{-j\theta} = cos(\theta) - jsin(\theta)$$
  $$e^{j\theta} +  e^{-j\theta} = cos(\theta) + jsin(\theta) + cos(\theta) - jsin(theta)$$
  $$e^{j\theta} +  e^{-j\theta} = 2cos(\theta)$$
  $$cos(\theta) = \frac{1}{2}(e^{j\theta} +  e^{-j\theta})$$

}

\end{enumerate}

\newpage
\qns{Phasors Through Impedance}

\meta{Prereqs: Complex Numbers, Complex Exponentials, Euler's Formula}

Analyzing circuits with only resistors is easy due to Ohm's Law: $v(t)=i(t) \cdot R$.
When adding capacitors or inductors however, the voltage/current relationship becomes complicated due to derivatives resulting in differential equations.
In fact, for each capacitor/inductor added to a circuit, a higher order differential equation arises, which becomes difficult to solve.

However, let's look at the $i-v$ relationships across a capacitor when $v(t) = e^{st}$ for some scalar $s.$
\begin{equation}
i(t) = C\ddt{}{t} v(t) = sC e^{st}
\end{equation}
Notice that this is still the same exponential function multiplied by a scaling factor of $sC.$ \vskip 0.5pt
Now if we look at a sinusoidal voltage $v(t) = V_{0} \cos(\omega{} t+\phi{}),$ we can write this voltage as a sum of complex exponentials using Euler's formula.
\begin{equation}
v(t) = V_{0} \cos(\omega{} t+\phi{}) = \frac{V_{0}}{2} e^{j(\omega{} t + \phi{})} + \frac{V_{0}}{2} e^{-j(\omega{} t + \phi{})}
\end{equation}
We can similarly look at the current across a capacitor, use trigonometric identities and see that:
\begin{equation} 
i(t) = \omega C \cdot V_{0} \cos(\omega{} t + \phi{} + \frac{\pi}{2})
\end{equation}
Which again is a cosine wave with the same frequency $\omega.$ \vskip 0pt
Therefore, to relate the current and voltages, we will define a quantity called a phasor:
\begin{equation}
\widetilde{V} = \frac{1}{2} V_{0} e^{j \phi{}} \ \  \text{and} \ \ \widetilde{I} = \frac{1}{2} I_{0} e^{j \phi{}}
\end{equation}
We will use these phasors to create an extension of Ohm's Law for capacitors and inductors through the relationship
\begin{equation}
\widetilde{V} = \widetilde{I} \cdot Z
\end{equation}
The quantity $Z$ is defined as the \textbf{impedance} of a circuit component.

% \begin{enumerate}

% \qitem Let $v(t) = V_{0} \cos(\omega{}t + \phi{})$.
% Using Euler's Formula, write $v(t)$ as a sum of two complex exponentials.

% \sol{
% Euler's formula says that $e^{j \theta} = \cos(\theta) + j \sin(\theta).$ \vskip 1pt
% Since we want to write $\cos(\theta)$ as a sum of exponentials, we want to cancel the $j \sin(\theta)$ term. \vskip 1pt
% This can be done by using $e^{-j \theta} = \cos(-\theta) + j \sin(-\theta) = \cos(\theta) - j \sin(\theta).$ \vskip 1pt
% Adding the two equations, we see that $\cos(\theta) = \frac{1}{2} e^{j \theta} + \frac{1}{2} e^{-j \theta}.$ \vskip 1pt
% Using this information we see that, 
% $$v(t) = \frac{V_{0}}{2} e^{j(\omega{} t + \phi{})} + \frac{V_{0}}{2} e^{-j(\omega{} t + \phi{})}.$$
% }

% \end{enumerate}

% Using these complex exponentials, we define a phasor: $\frac{\widetilde{V}}{2} = V_{0}e^{j\phi{}}$ and its complex conjugate $\overline{\widetilde{V}} = \frac{V_{0}}{2}e^{-j\phi{}}$ and it follows that $v(t) = \widetilde{V}e^{j\omega{}t} + \overline{\widetilde{V}}e^{-j\omega{}t}$. \vskip 1pt
% In the following parts, we define relationships between $\widetilde{V}$ and $\widetilde{I}$ where $i(t) = \widetilde{I}e^{j\omega{}t} + \overline{\widetilde{I}}e^{-j\omega{}t}$.

\begin{enumerate}
\qitem Let $v(t) = V_{0} \cos(\omega{}t + \phi{})$ be the voltage across a capacitor. Derive an expression for $i(t)$ in the form $i(t) = A\widetilde{V}e^{j\omega{}t} + B\overline{\widetilde{V}}e^{-j\omega{}t}$ for $A, B \in{} \mathbb{C}$.\\ 
\textit{Hint: Remember that $e^{j\frac{\pi{}}{2}} = j$.}

\sol{
The capacitor voltage-current relation is: $i(t) = C\ddt{v(t)}{t} = \omega{}C \cdot V_0 \cos(\omega{}t + \phi + \frac{\pi{}}{2}).$ 
\begin{align*} \text{Then, } \, i(t) &= \omega{}C\big(\frac{V_{0}}{2} e^{j\phi}e^{j\omega{}t}e^{j\frac{\pi}{2}} + \frac{V_{0}}{2} e^{-j\phi}e^{-j\omega{}t}e^{-j\frac{\pi}{2}}\big)
= \omega{}C\big(\widetilde{V}e^{j\omega{}t}e^{j\frac{\pi}{2}} + \overline{\widetilde{V}}e^{-j\omega{}t}e^{-j\frac{\pi}{2}}\big) \\
&= \omega{}C\big(\widetilde{V}e^{j\omega{}t} \cdot{} j + \overline{\widetilde{V}}e^{-j\omega{}t} \cdot{} (-j) \big) = j\omega{}C \widetilde{V}e^{j\omega{}t} - j \omega{} C 
\overline{\widetilde{V}}e^{-j\omega{}t} \end{align*}
}
\qitem From the previous part, try writing the expression $\widetilde{V} = \widetilde{I} \cdot Z_{c}$. 
We refer to the quantity $Z_{c}$ as the impedance of a capacitor.

\sol{
Recalling that $i(t) = \widetilde{I}e^{j\omega{}t} + \overline{\widetilde{I}}e^{-j \omega{} t},$ and equating with the previous part, we see that 
$\widetilde{I} = j \omega{} C \cdot \widetilde{V}$ and $\overline{\widetilde{I}} = -j \omega{} C \cdot \overline{\widetilde{V}}.$ 
The reason we can equate the $e^{j \omega{} t}$ and $e^{-j \omega{} t}$ terms is because they are linearly independent functions.
It follows that $Z_c = \frac{\widetilde{V}}{\widetilde{I}} = \frac{1}{j \omega{} C}.$

}

\qitem Now let's look at an inductor with current $i(t) = I_{0} \cos(\omega{}t + \phi{})$.
Write an expression for $v(t)$, similar to the previous part but in the form $v(t) = A\widetilde{I}e^{j\omega{}t} + B\overline{\widetilde{I}}e^{-j\omega{}t},$ for $A, B \in \mathbb{C}.$ \\
\textit{Hint: The identity $cos(x + \frac{\pi}{2}) = -\sin(x)$ may be helpful.}

\sol{
The inductor voltage-current relation is: $v(t) = L\ddt{i(t)}{t} = - \omega{} L \cdot I_{0} \sin (\omega{} t + \phi{}).$ 
Applying the hint, we see that $v(t) = \omega{} L \cdot I_{0} \cos(\omega{} t + \phi{} + \frac{\pi}{2}),$ and reapplying the same steps from part (b), we see that
$v(t) = j \omega{} L \widetilde{I}e^{j\omega{}t} - j \omega{} L \overline{\widetilde{V}}e^{-j\omega{}t}$
}

\qitem What is the impedance of an inductor?

\sol{
Equating $v(t)$ with its phasor form, we see that $\widetilde{V} = \widetilde{I} \cdot j \omega{} L.$ Therefore, $Z_L = j \omega L.$
}

\qitem \textbf{(Optional):} Show that the impedance of a resistor $Z_{R} = R$.

\sol{
We can start with a sinusoid $v(t) = V_{0} \cos(\omega{}t + \phi),$ it follows from Ohm's Law that $i(t) = \frac{V_{0}}{R} \cos(\omega{}t + \phi{}) = \frac{1}{R} (\frac{V_{0}}{2} e^{j\phi}e^{j\omega{}t} + \frac{V_{0}}{2} e^{-j\phi}e^{-j\omega{}t}\big) = \frac{1}{R} \big(\widetilde{V}e^{j\omega{}t} + \overline{\widetilde{V}}e^{-j\omega{}t}\big) = 
\widetilde{I}e^{j\omega{}t} - \overline{\widetilde{I}}e^{-j\omega{}t}.$ Equating both sides, we see that $\widetilde{V} = \widetilde{I} \cdot R$ so $Z_{R} = R.$
}

\end{enumerate}

\newpage
% Author: Taejin Hwang
% Email: taejin@berkeley.edu

\qns{Phasor analysis}

\meta{Prereqs: An understanding of what a phasor is, and what impedances are in a circuit.}

Any sinusoidal time-varying function $v(t)$, representing a voltage or a current, can be expressed in the form
\begin{align}
v(t) = \widetilde{V}e^{j\omega t} + \overline{\widetilde{V}}e^{-j\omega t},
\end{align} 
For a sinusoidal cosine wave $v(t) = V_{0} \cos(\omega t + \phi),$ a phasor is defined as:
$$\widetilde{V} = \frac{V_{0}}{2}e^{j\phi{}} \ \  \text{and its complex conjugate}  \ \ \overline{\widetilde{V}} = \frac{V_{0}}{2}e^{-j\phi{}}$$

$\widetilde{V}$ is a time-independent function, and is referred to as the phasor representation of $v(t).$

The phasor analysis method consists of five steps.
Consider the RC circuit below.

	\begin{center}
		\begin{circuitikz}
			\draw (0,3)
			to[vsourcesin =$v_s$] (0,0)
			(0,3) -- (2,3)
			to[R = $R$] (3,3)
			to[short,i>= \mbox{$i(t)$}] (6,3)
			to[C = $C$, v=$v_\text{c}(t)$] (6,0)
			to[short] (0,0);
			
		\end{circuitikz}
	\end{center}

The voltage source is given by
\begin{align}
v_s(t) = 8 \cos(\omega t - \frac{\pi}{4}),
\end{align}
with $\omega = 10^3$ rad/s, $R = 1$ $\text{k}\Omega$, and $C = 1$ $\mu\text{F}$.

Our goal is to obtain a solution for $v_{\text{c}}(t)$ and $i(t)$ with the sinusoidal voltage source $v_s(t)$.

\begin{enumerate}

\qitem \textbf{Step 1: Convert $v(t)$ as a sum of complex exponentials}

All voltages and currents with known sinusoidal functions should be expressed in the standard exponential format.
Convert $v_s(t)$ into an exponential and write down its phasor representation $\widetilde{V}_s$.

Note: To make things clear, the standard form of phasors in 16B will include the $\frac{1}{2}$ that comes from the cosine.

\sol{
\begin{align}
v_s(t) = 8 \cos(\omega t - \frac{\pi}{4})
\end{align}

The two formulas below will help us convert a sinusoid into a sum of exponentials:
\begin{gather*}
\cos(\theta) = \frac{1}{2} e^{j \theta} + \frac{1}{2} e^{-j \theta} \ \ \text{and} \ \ 
\sin(\theta) = \frac{1}{2j} e^{j \theta} - \frac{1}{2j} e^{-j \theta}
\end{gather*}


Using the fact that $\cos(\theta) = \frac{1}{2} e^{j \theta} + \frac{1}{2} e^{-j\theta},$
\begin{align*}
v_s(t) &= 8 \big( \frac{1}{2} e^{j(\omega t - \frac{\pi}{4})} + \frac{1}{2} e^{-j(\omega t - \frac{\pi}{4})} \big) = 4e^{j(\omega t - \frac{\pi}{4})} + 4e^{-j(\omega t - \frac{\pi}{4})} \\
&= 4e^{j \omega t} \cdot e^{-j \frac{\pi}{4}} + 4e^{-j \omega t} e^{j \frac{\pi}{4}}
\end{align*}

The phasor is given by
\begin{align}
\widetilde{V}_s = 4 e^{-j\frac{\pi}{4}}
\end{align}
}

% \ans{
% \begin{align}
% v_s(t) = 12 \cos (\omega t -\frac{\pi}{4} -\frac{\pi}{2}) = 12 \cos(\omega t - \frac{3\pi}{4})
% \end{align}

% The phasor is given by
% \begin{align}
% V_s = 12 e^{-j\frac{3\pi}{4}}
% \end{align}
% }

\qitem \textbf{Step 2: Transform circuits to phasor domain}

The voltage source is represented by its phasor $\widetilde{V}_s$.
The current $i(t)$ is related to its phasor counterpart $\widetilde{I}$. 
% by
% \begin{align}
% i(t) = \mathfrak{Re}[I e^{j\omega t}].
% \end{align}
What are the phasor domain representations of $R$ and $C$?

\sol{
The phasor domain representations of $R$ and $C$ will be the complex impedances:
\begin{align}
Z_R &= R\\
Z_C &= \frac{1}{j\omega C}
\end{align}
}

% \ans{
% \begin{align}
% Z_R &= R\\
% Z_C &= \frac{1}{j\omega C}
% \end{align}
% }

\qitem \textbf{Step 3: Cast KCL and/or KVL equations in phasor domain}

Use Kirchhoff's laws to write down an equation that relates all phasors in Step 2. 

\sol{
By KCL, the current through the resistor is equal to the current through the capacitor.
$$i(t) = i_{r}(t) = i_{\text{c}}(t) \rightarrow \widetilde{I} = \widetilde{I}_R = \widetilde{I}_C $$
By KVL, the sum of the voltages across the resistor and capacitor is equal to that of the source.
$$v_{s}(t) = v_{\text{c}}(t) + v_{r}(t) \rightarrow \widetilde{V}_s = \widetilde{V}_{\text{c}} + \widetilde{V}_{R}$$
Substituting the impedances in, we get:
$$\widetilde{V}_s = \widetilde{I} \frac{1}{j \omega C} + \widetilde{I} R = \widetilde{I} (R + \frac{1}{j\omega C})$$
}

\qitem \textbf{Step 4: Solve for unknown variables}

Solve the equation you derived in Step 3 for $\widetilde{I}$ and $\widetilde{V_{\text{c}}}$.
What is the polar form of $\widetilde{I}$ ($Ae^{i\theta}$, where $A$ is a positive real number)? 

\sol{
We first solve for the current $\widetilde{I}:$
$$\widetilde{I} = \frac{\widetilde{V}_s}{R + \frac{1}{j \omega C}}$$
We plug in for $\omega = 10^3, R = 10^3 \Omega, C = 10^{-6} \mu$F
$$\widetilde{I} = \frac{4 e^{-j \frac{\pi}{4}}}{10^3 + \frac{1}{j 10^{3} 10^{-6}}} = \frac{4 e^{-j \frac{\pi}{4}}}{10^3 - 10^3 j} = \frac{4 e^{-j \frac{\pi}{4}}}{10^{3} \sqrt{2} e^{-j \frac{\pi}{4}.}} = \frac{2 \sqrt{2}}{10^{3}} \mbox{ A} = 2 \sqrt{2} \mbox{ mA}$$

To solve for $\widetilde{V}_{\text{c}}$ we use the voltage-current relationship $\widetilde{V} = \widetilde{I} \frac{1}{j \omega C}$
$$\widetilde{V}_{\text{c}} = \frac{1}{10^{-3} j} \frac{2 \sqrt{2}}{10^{3}} = 2 \sqrt{2} e^{-j \frac{\pi}{2}} \mbox{ V}$$
}

% \ans{
% \begin{align}
% I = \frac{12 e^{-j\frac{3\pi}{4}}}{R + \frac{1}{j\omega C}} = \frac{j12\omega C e^{{-j\frac{3\pi}{4}}}}{1+ j\omega RC}
% \end{align}
% \begin{align*}
% V_C=IZ_C=\frac{j12\omega C e^{{-j\frac{3\pi}{4}}}}{1+ j\omega RC}*\frac{1}{j\omega C}= \frac{12e^{-j\frac{3\pi}{4}}}{1+j\omega RC}
% \end{align*}

% To derive the polar form,
% \begin{align}
% I = \frac{j12e^{-j\frac{3\pi}{4}}*10^{-3}}{1+j\sqrt{3}} = \frac{12e^{-j\frac{3\pi}{4}}e^{j\frac{\pi}{2}}*10^{-3}}{2e^{j\frac{\pi}{3}}}
% = 6e^{-j\frac{7\pi}{12}} \mbox{ mA}.
% \end{align}
% \begin{align}
% V=\frac{12e^{-j\frac{3\pi}{4}}}{1+j\omega RC} = \frac{12e^{-j\frac{3\pi}{4}}}{1+j\sqrt{3}}=\frac{12e^{-j\frac{3\pi}{4}}}{2e^{j\frac{\pi}{3}}}= 6e^{-j\frac{13\pi}{12}}  \mbox{ V}
% \end{align}
% }

\qitem \textbf{Step 5: Transform solutions back to time domain}

To return to time domain, we apply the fundamental relation between a sinusoidal function and its phasor counterpart.
What is $i(t)$ and $v_{\text{c}}(t)$? What is the phase difference between $i(t)$ and $v_{\text{c}}(t)$? 

\sol{
\begin{align}
i(t) = Ie^{j\omega t} + \widetilde{I}e^{-j\omega t} = 4 \sqrt{2} \cos (\omega t)  \mbox{ mA}
\end{align}
\begin{align}
v_C(t)= Ve^{j\omega t} + \widetilde{V}e^{-j\omega t}= 4 \sqrt{2} \cos(\omega t - \frac{\pi}{2}) \mbox{ V}
\end{align}
The phase difference between the two, with respect to $i(t)$ is $-\frac{\pi}{2}$
}

% \ans{
% \begin{align}
% i(t) = \mathfrak{Re}[Ie^{j\omega t}] = \mathfrak{Re} [6e^{-j\frac{7\pi}{12}} e^{j\omega t}] = 6 \cos (\omega t - \frac{7\pi}{12})  \mbox{ mA}
% \end{align}
% \begin{align}
% v_C(t)=\mathfrak{Re}[V_Ce^{j\omega t}]=\mathfrak{Re}[6e^{-j\frac{13\pi}{12}}e^{j\omega t}]=6 \cos(\omega t -\frac{13\pi}{12}) \mbox{ V}
% \end{align}
% The phase difference between the two, with respect to $i(t)$ is $-\frac{\pi}{2}$
% }

\qitem It's important to keep in mind that the phasor analysis above only applies to sinusoidal inputs. 
We make the observation that DC inputs however, are not sinusoidal. 
That being said, what would happen if we tried doing phasor analysis on DC inputs?

\sol {
	We can actually see DC inputs as a special case of sinusoidal inputs where $\omega = 0.$
	The capacitors will have impedance $Z_{c} = \frac{1}{0} \to \infty$ meaning capacitors will turn into open circuits.
	The inductors will have impedance $Z_{L} = 0$ meaning inductors will turn into short circuits.
	The results of the phasor analysis will be the steady state behavior of the circuit.
}

\end{enumerate}

\end{qunlist}

\end{document}


