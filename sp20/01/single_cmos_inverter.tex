% Authors: Naomi Sagan, Justin Yu

\qns{RC Transistor Model}

Consider a single CMOS inverter with a voltage input, except there is some resistance between the voltage source and the gates of the transistors.

\vspace{-1em}

\begin{center}
    \begin{figure}[H]
	\begin{centering}
		\begin{circuitikz}[scale=0.8]
				\draw (-1,4) 
				to [V=$V_{in}$](-1, 2) to (-1,2) node[ground]{}; 
				\draw (4,5) node[pmos, emptycircle](pm){} ;
				\draw (4,3) node[nmos, ](nm){} ;
				\draw (-1,4) to [R = $R$] (2,4) to [short] (2,5)
				to (pm.gate);
				\draw (pm.source) to [short, -*] (4,6) node[above]{$V_{DD}$};;
				\draw (pm.drain) to (nm.drain);
				\draw (2,4) to [short] (2, 3) to (nm.gate);
				\draw (nm.source) to (4,2) node[ground]{};
				\draw (4,4) [short] to [short, -*] (5,4) node[right]{$V_{out}$};
		\end{circuitikz}
		\caption{\label{fig:circuit}CMOS Inverter with Resistance}
	\end{centering}
\end{figure}
\end{center}

\vspace{-4em}

\begin{enumerate}

\qitem First, let's introduce a way to model CMOS transistors.
A transistor can be modeled with a capacitor between the gate $G$ and source $S$, an internal resistance, and a voltage-controlled switch between drain $D$ and $S$.

\textbf{Using the NMOS and PMOS transistor models below, redraw the inverter as a simple RC circuit.}

\centering
\begin{figure}[H]
    \centering
    \begin{subfigure}{0.45\linewidth}
        \centering
        \begin{circuitikz}
            \draw (0,-2) node[ground,label=right:$S$] {}
            to[switch,l_= $V_{GS} \geq V_{tn}$]
            (0,0) to[R,-*,l=$R_{on}$,i<=$I_D$] ++(0,2)
            node[label=left:$D$] {};;
            \draw (0,-2) -- (-1,-2)
            to[C,-*,l=$C_G$] ++(0,2) node[label=left:$G$] {};;
            \draw (0,1.75) to [short,-*,l=$V_{out}$] ++ (1,0);;
        \end{circuitikz}
        \caption*{\label{fig:nmoscap} \small{NMOS Resistor-Switch-Capacitor Model}}
    \end{subfigure}
	\begin{subfigure}{0.45\linewidth}
        \centering
		\begin{circuitikz}
			\draw (0, -2) node[label=left:$D$] {}
			to[switch,l_= $V_{GS} \leq -|V_{tp}|$, *-] (0,0)
			to[R, -*,l_=$R_{on}$, i<=$I_D$] ++(0, 2)
			node[vcc,label=right:$S$] {};;
			\draw (0, 2) -- (-1, 2)
			to[C, -*, l_=$C_G$] ++(0, -2) node[label=left:$G$] {};;
			\draw (0, -1.75) to [short, -*, l=$V_{out}$] ++(1,0);;
		\end{circuitikz}
		\caption*{\label{fig:pmoscap} \small{PMOS Resistor-Switch-Capacitor Model}}
	\end{subfigure}
\end{figure}

% \noindent
% \begin{figure}[ht]
% \centering
% \begin{subfigure}{.48\textwidth}
% \flushleft
% \begin{subfigure}{.3\textwidth}
% \centering
% \begin{circuitikz}[american] 
% \draw (0, 0) node[pmos] (nmos) {};
% \draw (nmos.G) node[left]{$G$};
% \draw (nmos.S) node[left]{$S$};
% \draw (nmos.D) node[left]{$D$};
% \end{circuitikz}
% \end{subfigure}
% \centering
% \begin{subfigure}{.2\textwidth}
% \flushright\begin{circuitikz}[american] 
% \draw (0, 1) node[ocirc]{} node[above]{$G$}
%       to[short] (0, -1)
%       to[C = $C_{GS}$, v=$V_{GS}$] (2, -1);
% \draw (2, -1) node[ocirc]{} node[below]{$S$}
%       to[switch, l_=$V_{SG} \geq |V_{tp}|$] (2, 1)
%       to[R=$R_{on_, P}$] (2, 2.2)
%       to[short] (2, 3.4)
%       node[ocirc]{} node[above]{$D$} (2, 4);
% \end{circuitikz}
% \end{subfigure}
% \caption{PMOS Transistor Model}
% \end{subfigure}
% \begin{subfigure}{0.48\textwidth}
% \flushleft
% \begin{subfigure}{.3\textwidth}
% \centering
% \begin{circuitikz}[american] 
% \draw (0, 0) node[nmos] (nmos) {};
% \draw (nmos.G) node[left]{$G$};
% \draw (nmos.S) node[left]{$S$};
% \draw (nmos.D) node[left]{$D$};
% \end{circuitikz}
% \end{subfigure}
% \centering
% \begin{subfigure}{.2\textwidth}
% \flushright
% \begin{circuitikz}[american] 
% \draw (0, 1) node[ocirc]{} node[above]{$G$}
%       to[short] (0, -1)
%       to[C = $C_{GS}$, v=$V_{GS}$] (2, -1);
% \draw (2, -1) node[ocirc]{} node[below]{$S$}
%       to[switch, l_=$V_{GS} \geq V_{tn}$] (2, 1)
%       to[R=$R_{on_, N}$] (2, 2.2)
%       to[short] (2, 3.4)
%       node[ocirc]{} node[above]{$D$};
% % \draw (0, 0) to[open, v=$V_{GS}$] (1, -1);
% \end{circuitikz}
% \end{subfigure}
% \caption{NMOS Transistor Model}
% \end{subfigure}
% \end{figure}

% \begin{figure}
% \centering
% \begin{subfigure}{.2\textwidth}
% \centering
% \begin{circuitikz}[american] 
% \draw (0, 0) node[nmos] (nmos) {};
% \draw (nmos.G) node[left]{$G$};
% \draw (nmos.S) node[left]{$S$};
% \draw (nmos.D) node[left]{$D$};
% \end{circuitikz}
% \end{subfigure}
% \begin{subfigure}{.2\textwidth}
% \centering
% \begin{circuitikz}[american] 
% \draw (0, 1) node[ocirc]{} node[above]{$G$}
%       to[short] (0, -1)
%       to[C = $C_{GS}$, v=$V_{GS}$] (2, -1);
% \draw (2, -1) node[ocirc]{} node[below]{$S$}
%       to[switch, l_=$V_{GS} \geq V_{tn}$] (2, 1)
%       to[R=$R_{on_, N}$] (2, 2.2)
%       to[short] (2, 3.4)
%       node[ocirc]{} node[above]{$D$};
% % \draw (0, 0) to[open, v=$V_{GS}$] (1, -1);
% \end{circuitikz}
% \end{subfigure}
% \caption{NMOS Transistor Model}
% \end{figure}

% \centering
% \begin{circuitikz}[american] 
% \draw (0, 0) node[ocirc]{} node[above]{$G$};
% \draw (1, -1) node[ocirc]{} node[below]{$S$} to[switch] (1, 1) node[ocirc]{} node[above]{$D$};
% \draw (0, 0) to[open, v=$V_{GS}$] (1, -1);
% \end{circuitikz}
% % \caption{$V_{GS} > V_{th, N}$}

\newpage

\sol{}

\qitem
We want to analyze the behavior of the inverter by solving for the gate voltage $V_{GS}$ \textit{of the bottom NMOS transistor} as a function of time. \textbf{Using KCL or KVL, write out a differential equation for $V_{GS}$.}

\ws{\vspace{200px}}

\sol{}

\qitem 
Assume that $V_{in}$ has been low ($V_{in} = 0$) for a long time and switches to high ($V_{in} = V_{DD}$) at time $t = 0$. \textbf{Solve for $V_{GS}(t)$ using the differential equation from the last part and the above initial condition.}

\ws{\vspace{200px}}

\sol{}
\meta{}

\end{enumerate}