% Authors: Justin Yu
% Email: justinvyu@berkeley.edu

\qns{CR Circuit}

\meta {
}

We're already familiar with the charging RC circuit from the second problem of this worksheet, 
but what happens if we flip the orientation of the circuit components and switch to discharging?

% Assume that $V_{in}(t)=\SI{5}{\volt}$, $R=\SI{100}{\ohm}$, and $C=\SI{10}{\micro\farad}$.
Consider the CR circuit below:
\begin{center}
    \begin{circuitikz}[scale=0.8]
        \draw (-1,4) 
        to [V = $V_{in}(t)$] (-1,0)
        (-1, 4) to [short] (1, 4)
        % (-1,4) to [opening switch, l_=\mbox{$t = 0$} ] (1,4)
        (1,4) to [C = $C$,i=$i_C(t)$, v = $V_C(t)$] (4,4)
        (4,4) to [short] (6,4)
        to [R = $R$, v = $V_{out}$] (6,0)
        to [short] (-1,0);
    \end{circuitikz}
\end{center}

\begin{enumerate}

\qitem \textbf{Write out the differential equation for the voltage $V_{C}(t)$ across the capacitor in terms of constants and $V_{in}(t)$.}

\ws{
\vspace{50px}
}

\qitem 
Assume that when $t\leq 0$, the capacitor has been fully charged with an input voltage $V_{DD}$, with the initial condition $V_C(t=0) = V_{DD}$.
At $t=0$, the input voltage switches from high to low, so that $V_{in}(t) = 0$ for $t \geq 0$.

\textbf{Plug in these conditions to the differential equation from the previous part and solve for $V_{C}(t)$ for $t \geq 0$.}

\ws{
\vspace{50px}
}

\sol{
    KCL and Ohm's law at the $V_{out}$ node gives us:
    \begin{align*}
        C \frac{d V_C(t)}{dt} &= \frac{V_{in} - V_C(t)}{R} \\
        \implies \frac{d}{dt} V_C(t) &= - \frac{1}{RC} V_C(t) + \frac{1}{RC} V_{in}
    \end{align*}
}

\qitem \textbf{What is $V_{out}(t)$?} Sketch a plot of the voltage across the resistor over time, labeling the asymptote it reaches at steady-state.

\ws{
\vspace{30px}
}

\sol {
    We begin with a change of variables:
    \begin{align*}
        V_C(t) &= \widetilde{V_C}(t) - \frac{\frac{1}{RC}}{-\frac{1}{RC}} V_{in} \\
               &= \widetilde{V_C}(t) + V_{in}
    \end{align*}

    We plug this new definition of $V_C(t)$ into the original differential equation to remove the constant term.
    \begin{align*}
        \frac{d}{dt} (\widetilde{V_C}(t) + V_{in}) &= - \frac{1}{RC} (\widetilde{V_C}(t) + V_{in}) + \frac{1}{RC} V_{in} \\
        \frac{d}{dt} \widetilde{V_C}(t) &= - \frac{1}{RC} \widetilde{V_C}(t)
    \end{align*}

    Now, we can solve for the new variable $\widetilde{V_C}(t)$:
    \begin{align*}
        \widetilde{V_C}(t) &= Ae^{-\frac{1}{RC} t}
    \end{align*}

    Plug this general solution back into the change of variables definition, then solve using the initial solution: $V_C(0) = 0$.
    \begin{align*}
        V_C(t) &= Ae^{-\frac{1}{RC} t} + V_{in} \\
        \implies V_C(0) = 0 &= Ae^0 + V_{in} \\
        \implies A &= -V_{in}
    \end{align*}
    
    We're now finished, with the final solution:
    \begin{align*}
        V_C(t) &= -V_{in}e^{-\frac{1}{RC} t} + V_{in} \\
               &= V_{in} (1 - e^{-\frac{1}{RC} t})
    \end{align*}
}

\qitem \textbf{What is the steady-state voltage across the capacitor as $t \rightarrow \infty$?}

\ws{
\vspace{40px}
}

\sol {
    \begin{align*}
        \lim_{t \rightarrow \infty} i_L(t) &= \lim_{t \rightarrow \infty} \frac{V_{in}}{R} (1 - e^{-\frac{R}{L}t}) \\
        &= \frac{V_{in}}{R} (1 - 0) \\
        &= \frac{V_{in}}{R}
    \end{align*}

    The physics of the inductor opposes current flow initially, but it approaches the steady-state current as determined by the rest of the circuit (the resistor and voltage source).
}

\qitem \textbf{What is the steady-state current across the capacitor as $t \rightarrow \infty$?}

\ws{
\vspace{40px}
}

\sol {
    \begin{align*}
        \lim_{t \rightarrow \infty} V_L(t) &= \lim_{t \rightarrow \infty} L \frac{d}{dt} i_L(t)  \\
        &= \lim_{t \rightarrow \infty} L (\frac{d}{dt} \frac{V_{in}}{R} (1 - e^{-\frac{R}{L}t})) \\
        &= \lim_{t \rightarrow \infty} L \frac{V_{in}}{R} \frac{R}{L} e^{-\frac{R}{L}t} \\
        &= \lim_{t \rightarrow \infty} V_{in} e^{-\frac{R}{L}t} \\
        &= 0
    \end{align*}
}

\qitem \textbf{What circuit element does the capacitor act like at steady-state ($t \rightarrow \infty$)?}

\sol{
    It acts like a short circuit element.
}

\end{enumerate}
