% Authors: Justin Yu
% Email: justinvyu@berkeley.edu

\qns{CR Circuit}

\meta {
}

We're already familiar with the RC circuit from the second problem of this worksheet, but what happens if we flip the orientation of the circuit components?

Consider the circuit below, assume that when $t\leq 0$, the capacitor has no charge stored $(V_{\text{C}}(t=0) = 0)$.
At $t=0$, the switch closes. Assume that $V_{in}=\SI{5}{\volt}$, $R=\SI{100}{\ohm}$, and $C=\SI{10}{\micro\farad}$.

\begin{center}
    \begin{circuitikz}[scale=0.8]
        \draw (-1,4) 
        to [V = $V_{in}$] (-1,0)
        (-1,4) to [switch, l^=\mbox{$t = 0$} ] (1,4)
        (1,4) to [C = $C$,i=$i_C(t)$, v = $V_C(t)$] (4,4)
        (4,4) to [short] (6,4)
        to [R = $R$, v = $V_{out}$] (6,0)
        to [short] (-1,0);
    \end{circuitikz}
\end{center}

\begin{enumerate}

\qitem \textbf{Write out the differential equation for the voltage $V_C(t)$ of the capacitor when the switch is closed.}

\ws{
\vspace{3em}
}

\meta{}

\sol{
    KCL and Ohm's law gives us:
    \begin{align*}
        V_{in} - V_{out} &= V_L(t) = L \frac{d}{dt} i_L(t) \\
        V_{out} &= R i_L(t) \\
        \implies L \frac{d}{dt} i_L(t) &= V_{in} - R i_L(t) \\
        \implies \frac{d}{dt} i_L(t) &= -\frac{R}{L}i_L(t) + \frac{V_{in}}{L}
    \end{align*}
}

\qitem \textbf{Solve the differential equation for $V_C(t)$.}

\ws{
\vspace{3em}
}

\sol {
    \begin{align*}
        \frac{d}{dt} i_L(t) &= -\frac{R}{L}i_L(t) + \frac{V_{in}}{L} \\
        \implies i_L(t) &= \frac{V_{in}}{R} (1 - e^{-\frac{R}{L}t})
    \end{align*}
}

\qitem \textbf{What is the steady-state current through the inductor as $t \rightarrow \infty$?} Sketch a plot of the current through the inductor over time, labeling the asymptote after reaching the steady-state.

\ws{
\vspace{3em}
}

\sol {
    \begin{align*}
        \lim_{t \rightarrow \infty} i_L(t) &= \lim_{t \rightarrow \infty} \frac{V_{in}}{R} (1 - e^{-\frac{R}{L}t}) \\
        &= \frac{V_{in}}{R} (1 - 0) \\
        &= \frac{V_{in}}{R}
    \end{align*}

    The physics of the inductor opposes current flow initially, but it approaches the steady-state current as determined by the rest of the circuit (the resistor and voltage source).
}

\qitem \textbf{What is the steady-state voltage drop across the inductor as $t \rightarrow \infty$?}

\ws{
\vspace{3em}
}

\sol {
    \begin{align*}
        \lim_{t \rightarrow \infty} V_L(t) &= \lim_{t \rightarrow \infty} L \frac{d}{dt} i_L(t)  \\
        &= \lim_{t \rightarrow \infty} L (\frac{d}{dt} \frac{V_{in}}{R} (1 - e^{-\frac{R}{L}t})) \\
        &= \lim_{t \rightarrow \infty} L \frac{V_{in}}{R} \frac{R}{L} e^{-\frac{R}{L}t} \\
        &= \lim_{t \rightarrow \infty} V_{in} e^{-\frac{R}{L}t} \\
        &= 0
    \end{align*}
}

\qitem \textbf{What circuit element does the inductor act like at steady-state?}

\sol{It acts like a wire element, since the voltage drop across the inductor goes to 0.}

\end{enumerate}
