\qns{Building an Observer}

In this question we will look how to design an \textbf{observer} that estimates the state vector $\vec{x}.$ 
Recall that a system is observable if the observability matrix $\mathcal{O}$ has $n$ linearly independent rows.
\begin{equation}
 \mathcal{O} = \begin{bmatrix} C \\ CA \\ \vdots \\ CA^{n-1} \end{bmatrix}
\end{equation}
While we have a way to recover our initial state if the observability criterion is met, it currently requires us to wait $n$ time steps to collect all of the output measurements in order to solve for $\vect{x}{0}.$

Using an observer, we will estimate the state at each time step, $\vecht{x}{t}$ and use feedback to continually improve our estimates.
This is done by comparing the observed output $\vect{y}{t}$ with the estimated output $\vecht{y}{t} = C \vecht{x}{t}$ and a feedback matrix $L.$

The new updated State-space equations for the observer system will be:
\begin{align*}
  \vecht{x}{t+1} & = A \vecht{x}{t} + B \vect{u}{t} + \underbrace{ L \pp{ \vecht{y}{t} - \vect{y}{t}   }}_{\text{output feedback} } \\
  \vecht{y}{t} &= C \vecht{x}{t} \qquad \qquad \vect{y}{t} = C \vect{x}{t}
\end{align*}

\begin{enumerate}
  \qitem We'll start off by defining an error vector $\vect{e}{t} = \vecht{x}{t} - \vect{x}{t}.$ Use the State-space equations for $\vecht{x}{t}$ and $\vect{x}{t}$ to write $\vect{e}{t + 1}$ in terms of $\vect{e}{t}.$

  \sol {
    \begin{align*}
      \vect{e}{t + 1} &= \vecht{x}{t + 1} - \vect{x}{t + 1} \\
      &= A \vecht{x}{t} + B \vect{u}{t} + L \pp{ \vecht{y}{t} - \vect{y}{t} } - (A \vect{x}{t} + B \vect{u}{t}) \\
      &= A \pp{\vecht{x}{t} - \vect{x}{t}} + L \pp{ \vecht{y}{t} - \vect{y}{t} } \\
      &= A \pp{\vecht{x}{t} - \vect{x}{t}} + L \pp{ C \vecht{x}{t} - C \vect{x}{t} } \\
      &= (A + LC) \pp{\vecht{x}{t} - \vect{x}{t}} \\
      &= (A + LC) \vect{e}{t}
    \end{align*} 
  }

  \qitem Under what conditions will $\vect{e}{t} \to \vec{0}$ as $t \to \infty?$ 

  \sol {
    Since this is a discrete-time system, if all of the eigenvalues of the matrix $A + LC$ have magnitude less than 1, then the error will converge to $\vec{0}$ as $t \to \infty.$
  }
\end{enumerate}

Note that the structure of the matrix $A + LC$ looks very similar to that of $A + BK$ when doing feedback control.
Therefore we should start suspecting that if we pick specific values for $L,$ then we can control the eigenvalues of the observer error.
However, how can we be sure that we can \textbf{always} pick values of $L$ to pick all of our eigenvalues? 

We will show this by looking at the \textbf{dual system} with state variable $\vect{z}{t}$ and input $\vect{w}{t}.$
\begin{equation}
  \vect{z}{t + 1} = A^{T} \vect{z}{t} + C^{T} \vect{w}{t}
\end{equation}

\begin{enumerate}[resume]
  \qitem Under what conditions is the dual system controllable? Show that if the original system is observable, the dual system will always be controllable.

  \sol {
    Remember that a system is controllable if its controllabilty matrix $\mathcal{C}$ spans $\mathbb{R}^{n}.$ \\
    In this case, the controllability matrix is 
    \[ \begin{bmatrix} C^{T} & A^{T} C^{T} \dots \pp{A^{T}}^{n-1} C^{T} \end{bmatrix} \]
    If the original system is observable, then the observability matrix is of full rank, or rank $n.$
    \begin{equation}
     \mathcal{O} = \begin{bmatrix} C \\ CA \\ \vdots \\ CA^{n-1} \end{bmatrix}
    \end{equation}
    However, since the observabiltiy and controllability matrices are transposes of each other, they must have the same rank!
    Therefore, we conclude by saying that the dual system will be controllable when the original system is observable.
  }

  \qitem What are the implications of the dual system being controllable? Does this follow your intuition about observability?

  \sol {
    When the dual system is controllable, it means that we can pick inputs $\vect{w}{t}$ such that $\vect{z}{t}$ goes to any desired vector in $\mathbb{R}^{n}.$ This means we can pick values inputs $\vect{w}{t} = L^{T} \vect{z}{t}$ to make $\vect{z}{t}$ go to zero. Also, note that the dual system is the transpose of the error system $\vect{e}{t}.$ 

    Therefore, if the original system is observable, the dual is controllable, meaning we can pick values of $L$ to make the error go to zero. This should not be a surprise since we knew that if a system was observable, we could recover $\vect{x}{t}$ after a sufficient number of time steps, which is equivalent to the error going to zero. 
  }

\end{enumerate}
