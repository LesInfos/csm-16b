%author Alexander Feng

\qns{Important Properties of the DFT}

In this question, we will prove basic and important properties of the DFT.

Suppose that we have the DFT matrix for a signal sampled at $N$ points is defined as follows:

\begin{align*}
F_N =
  \begin{bmatrix}
  1 & 1 & 1 & \cdots  & 1 \\
  1 & \omega & \omega^2 & \cdots & \omega^{N-1} \\
  1 & \omega^2 & \omega^4  & \cdots &  \omega^{2(N-1)}\\
  \vdots & \vdots & \vdots & & \vdots \\
  1 & \omega^{N-1} & \omega^{2(N-1)}  & \cdots  & \omega^{(N-1)(N-1)}
  \end{bmatrix}
\end{align*}

{\em Note that the DFT we defined is commonly used as a polynomial basis.}


\begin{enumerate}

\qitem
Prove that the columns of the DFT matrix are orthogonal.
{\em
Notice that it's easy to make the entire matrix orthogonal.
}

\sol{
In order to prove that the DFT columns are orthogonal, it suffices to show that an arbitrary column dot producted with any other column produces 0.

\vspace{0.5em}

Suppose we took the $k^{th}$ column and the $l^{th}$ column where we're zero indexing.

\vspace{0.5em}

We have
$F_{N}[k] =
\begin{bmatrix}
1\\
\omega^{k}\\
\omega^{2k}\\
\vdots\\
\omega^{l(N-1)}\\
\end{bmatrix}
$
and
$F_{N}[l] =
\begin{bmatrix}
1\\
\omega^{l}\\
\omega^{2l}\\
\vdots\\
\omega^{l(N-1)}\\
\end{bmatrix}
$

\vspace{0.5em}

When we dot product the two columns together, we end up with a summation, which is just a geometric series. Defining $p = \omega^{k+l}$,

\begin{align*}
\sum^{N-1}_{b=0}(\omega^{bk})\omega^{bl} =  \sum^{N-1}_{b=0}\omega^{b(k+l)} =  \sum^{N-1}_{b=0}(\omega^{(k+l)})^{b} = \frac{1 - p^{N}}{1-p}
\end{align*}

Recall that $(\omega^{k+l})^{N}$ is just 1.
Thus, our summation is 0 and columns of the DFT are orthogonal.

}

\qitem What happens when you take a column of a DFT matrix and dot product it with itself?

\sol{
You can just apply the same idea as part (a), and you should arrive at the conclusion that a DFT column dotted with itself results in $N$, the size of the matrix.
}


\qitem
How do I make it so that the DFT matrix we defined becomes an orthonormal matrix?
{\em HINT: An orthonormal matrix is a matrix that when multiplied with itself gives us the identity}

\sol{
We just need to normalize all the columns so that they're unit vectors.
From part (b), that just means multiplying the matrix $F_{N}$ by the normalization factor $\frac{1}{\sqrt{N}}$.
The matrix $\frac{1}{\sqrt{N}}{F_{N}}$ is called the synthesis matrix or orthonormal DFT basis.
}

\qitem What's the inverse of $F_{N}$ (not the orthonormal version from part c)?

\meta { 
$F^{\star}_{N}$
Is the conjugate transpose of $F_N$.
The Lecture uses $\frac{1}{\sqrt{N}}$ for the DFT Matrix.
}

\sol{
$\frac{1}{N}{F^{\star}_{N}}$
%$\frac{1}{N}{F^{\dagger}_{N}}$
}

\qitem
What's the sum of the roots of unity?

\sol{
The roots of unity perfectly slice the unit circle in $n$ pieces if we're working with $n$ roots.
Thus, the sum of the roots is zero.
Alternatively, you can use a geometric summation to prove that the roots of unity sum to 0.
}

\qitem
Prove that multiplying all the roots of unity by a nonzero number results in the same sum (as if multiplying had no effect)

\sol{
Using a geometric summation from part (e), we can factor out the nonzero number.
The summation still adds up to 0.
Thus, multiplying the roots of unity by a nonzero number doesn't change the sum.
}




\end{enumerate}
