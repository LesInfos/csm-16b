%author Alexander Feng

\qns{Properties of the DFT}

In this question, we will prove basic and important properties of the DFT.

The DFT matrix for a signal sampled at $N$ points is defined as follows:

\begin{align*}
F_N =
\begin{bmatrix}
1 & 1 & 1 & \cdots & 1 \\
1 & \omega_N^{-1} & \omega_N^{-2} & \cdots & \omega_N^{-(N-1)} \\
1 & \omega_N^{-2(1)} & \omega_N^{-2(2)} & \cdots & \omega_N^{-2(N-1)} \\
1 & \omega_N^{-3(1)} & \omega_N^{-3(2)} & \cdots & \omega_N^{-3(N-1)} \\
\vdots & \vdots & \vdots  & \ddots & \vdots \\
1 & \omega_N^{-(N-1)(1)} & \omega_N^{-(N-1)(2)} & \cdots & \omega_N^{-(N-1)(N-1)} \\
\end{bmatrix}
\end{align*}

\begin{enumerate}

\qitem
Prove that the columns of the DFT matrix are orthogonal.
{\em
Notice that it's easy to make the entire matrix orthogonal.
}

\sol{
In order to prove that the DFT columns are orthogonal, it suffices to show that an arbitrary column dot producted with any other column produces 0.

Suppose we took the $i^{th}$ column and the $j^{th}$ column.
We have
$\vec{F_{N}[i] =
\begin{bmatrix}
\end{bmatrix}
}$
and
$\vec{F_{N}[j] = }$
}



\end{enumerate}
