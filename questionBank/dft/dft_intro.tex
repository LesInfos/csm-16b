\qns{Discrete Fourier Transform Introduction}

% Sketch (from content channel):
% Intro about sampling points from an AC signal and wanting to extract the original signal (frequencies, magnitudes, phase shifts).
% Define the DFT matrix and how it can be applied to signals (what you get out when you input a signal of a certain frequency).
% (a) Given N = 5, write out the kth DFT basis vector/ column of the DFT matrix.
% (b) Write out the DFT matrix for N = 5
% (c) Given a sinusoid (constant term + something with frequency (2 pi /5) * 2), write out your x vector
% (d) Apply the DFT to get your frequency domain (X) vector.
% (e) How do the elements of the X vector correspond to your original signal?
% (d) What is the highest frequency you can get out of the DFT for N = 5?

To motivate the DFT, imagine a scenario where you would like to analyze an important signal
(for example, a radio transmission from someone sending a message to you).
First, you periodically sample the incoming signal, collecting a total of $N$
points. Then, you put all of these samples into a vector $\vec{x}$. As we will show in this
problem, applying the DFT to this vector $\vec{x}$ will allow you to extract important information such as
the different frequencies that are present in the signal. In other words, applying the DFT to $N$ samples
of a time-domain signal gives us its frequency-domain representation, showing us how much of each frequency is in our signal.


The DFT matrix for a signal sampled at $N$ points is defined as follows:

\begin{align*}
F_N = 
\begin{bmatrix}
1 & 1 & 1 & \cdots & 1 \\
1 & \omega_N^{-1} & \omega_N^{-2} & \cdots & \omega_N^{-(N-1)} \\
1 & \omega_N^{-2(1)} & \omega_N^{-2(2)} & \cdots & \omega_N^{-2(N-1)} \\
1 & \omega_N^{-3(1)} & \omega_N^{-3(2)} & \cdots & \omega_N^{-3(N-1)} \\
\vdots & \vdots & \vdots  & \ddots & \vdots \\
1 & \omega_N^{-(N-1)(1)} & \omega_N^{-(N-1)(2)} & \cdots & \omega_N^{-(N-1)(N-1)} \\
\end{bmatrix} 
\end{align*}

Recall that $\omega_N$ is the basis $N^{th}$ root of unity, defined as $\omega_N = e^{j\frac{2\pi}{N}}$.

\begin{enumerate}

\qitem
Given $N = 5$, { \bf write out the $k^{th}$ DFT basis vector. }

{\em HINT:
The columns of the DFT matrix are its basis vectors.
}

\sol{
Following the hint, the $k^{th}$ DFT basis vector is the $k^{th}$ column of the DFT matrix. For $N$ = 5 and an arbitrary $k$,
\begin{align*}
\vec{v_k} =
\begin{bmatrix}
1 \\
e^{-kj\frac{2\pi}{5}} \\
e^{-2kj\frac{2\pi}{5}} \\
e^{-3kj\frac{2\pi}{5}} \\
e^{-4kj\frac{2\pi}{5}} \\
\end{bmatrix}
= 
\begin{bmatrix}
1 \\
e^{-\frac{2{\pi}kj}{5}} \\
e^{-\frac{4{\pi}kj}{5}} \\
e^{-\frac{6{\pi}kj}{5}} \\
e^{-\frac{8{\pi}kj}{5}} \\
\end{bmatrix} 
\end{align*}
}

\qitem 
{ \bf Write out the DFT matrix for $N = 5$. }

{\em HINT:
Use the general formula from the previous part to fill in the full matrix.
}

\sol{
Using the formula above, we get:
\begin{align*}
F_5 =
\begin{bmatrix}
1 & 1 & 1 & 1 & 1 \\
1 & e^{-j\frac{2\pi}{5}} & e^{-2j\frac{2\pi}{5}} & e^{-3j\frac{2\pi}{5}} & e^{-4j\frac{2\pi}{5}} \\
1 & e^{-2j\frac{2\pi}{5}} & e^{-4j\frac{2\pi}{5}} & e^{-6j\frac{2\pi}{5}} & e^{-8j\frac{2\pi}{5}} \\
1 & e^{-3j\frac{2\pi}{5}} & e^{-6j\frac{2\pi}{5}} & e^{-9j\frac{2\pi}{5}} & e^{-12j\frac{2\pi}{5}} \\
1 & e^{-4j\frac{2\pi}{5}} & e^{-8j\frac{2\pi}{5}} & e^{-12j\frac{2\pi}{5}} & e^{-16j\frac{2\pi}{5}} \\
\end{bmatrix} 
\end{align*}
}

% (c) Given a sinusoid (constant term + something with frequency (2 pi /5) * 2), write out your x vector

\qitem

Consider the following signal:

\begin{align*}
f(t) = cos(\frac{4\pi}{5}t + \frac{\pi}{2})
\end{align*}

Let our sampling frequency be 1 Hz. Starting at t = 0, {\bf sample $f$ at $N = 5$ points to construct $\vec{x}$. }

{\em HINT:
We can get samples of a time-domain signal by evaluating it at different points in time.
}

\sol{
}

\qitem
{\bf Apply the DFT to find $\vec{X}$}, your sampled signal in the frequency domain.

\sol{
}

\qitem
How do the elements of the $\vec{X}$ correspond to your original signal?

\sol{
}

\qitem
What is the highest frequency you can get out of the DFT for $N = 5$?

\sol{
}

\end{enumerate}


