\qns{Discrete Fourier Transform Introduction}

% Sketch (from content channel):
% Intro about sampling points from an AC signal and wanting to extract the original signal (frequencies, magnitudes, phase shifts).
% Define the DFT matrix and how it can be applied to signals (what you get out when you input a signal of a certain frequency).
% (a) Given N = 5, write out the kth DFT basis vector/ column of the DFT matrix.
% (b) Write out the DFT matrix for N = 5
% (c) Given a sinusoid (constant term + something with frequency (2 pi /5) * 2), write out your x vector
% (d) Apply the DFT to get your frequency domain (X) vector.
% (e) How do the elements of the X vector correspond to your original signal?
% (d) What is the highest frequency you can get out of the DFT for N = 5?

The DFT matrix for a signal sampled at $N$ points is defined as follows:

\begin{align*}
F_N = 
\begin{bmatrix}
1 & 1 & 1 & \cdots & 1 \\
1 & \omega_N^{-1} & \omega_N^{-2} & \cdots & \omega_N^{-(N-1)} \\
1 & \omega_N^{-2(1)} & \omega_N^{-2(2)} & \cdots & \omega_N^{-2(N-1)} \\
1 & \omega_N^{-3(1)} & \omega_N^{-3(2)} & \cdots & \omega_N^{-3(N-1)} \\
\vdots & \vdots & \vdots  & \ddots & \vdots \\
1 & \omega_N^{-(N-1)(1)} & \omega_N^{-(N-1)(2)} & \cdots & \omega_N^{-(N-1)(N-1)} \\
\end{bmatrix} 
\end{align*}

Recall that $\omega_N$ is the basis $N^{th}$ root of unity, defined as $\omega_N = e^{j\frac{2\pi}{N}}$.

\vspace{0.5em}
When we apply the DFT to a vector of sample points, the elements of the resulting $\vec{X}$ correspond to the frequencies present in the signal. \\
The first element (index 0) of $\vec{X}$ is $N$ times whatever constant/DC term is present in the signal. \\
For every wave of the form $A\cos(\frac{2\pi}{N}(kt) + \phi)$, the $k^{\text{th}}$ index (again, zero-indexed) of $\vec{X}$ is $\frac{AN}{2}e^{j\phi}$, 
and the $(N - k)^{\text{th}}$ index of $\vec{X}$ is the complex conjugate of the $k^{\text{th}}$ element, or $\frac{AN}{2}e^{-j\phi}$.

\begin{enumerate}

\qitem
Given $N = 5$, { \bf write out the $k^{th}$ DFT basis vector. }

{\em HINT:
The columns of the DFT matrix are its basis vectors.
}

\sol{
Following the hint, the $k^{th}$ DFT basis vector is the $k^{th}$ column of the DFT matrix. For $N$ = 5 and an arbitrary $k$,
\begin{align*}
\vec{v_k} =
\begin{bmatrix}
1 \\
e^{-kj\frac{2\pi}{5}} \\
e^{-2kj\frac{2\pi}{5}} \\
e^{-3kj\frac{2\pi}{5}} \\
e^{-4kj\frac{2\pi}{5}} \\
\end{bmatrix}
= 
\begin{bmatrix}
1 \\
e^{-\frac{2{\pi}kj}{5}} \\
e^{-\frac{4{\pi}kj}{5}} \\
e^{-\frac{6{\pi}kj}{5}} \\
e^{-\frac{8{\pi}kj}{5}} \\
\end{bmatrix} 
\end{align*}
}

\qitem 
{ \bf Write out the DFT matrix for $N = 5$. }

{\em HINT:
Use the general formula from the previous part to fill in the full matrix.
}

\sol{
Using the formula above, we get:
\begin{align*}
F_5 =
\begin{bmatrix}
1 & 1 & 1 & 1 & 1 \\
1 & e^{-j\frac{2\pi}{5}} & e^{-2j\frac{2\pi}{5}} & e^{-3j\frac{2\pi}{5}} & e^{-4j\frac{2\pi}{5}} \\
1 & e^{-2j\frac{2\pi}{5}} & e^{-4j\frac{2\pi}{5}} & e^{-6j\frac{2\pi}{5}} & e^{-8j\frac{2\pi}{5}} \\
1 & e^{-3j\frac{2\pi}{5}} & e^{-6j\frac{2\pi}{5}} & e^{-9j\frac{2\pi}{5}} & e^{-12j\frac{2\pi}{5}} \\
1 & e^{-4j\frac{2\pi}{5}} & e^{-8j\frac{2\pi}{5}} & e^{-12j\frac{2\pi}{5}} & e^{-16j\frac{2\pi}{5}} \\
\end{bmatrix} 
\end{align*}
}

% (c) Given a sinusoid (constant term + something with frequency (2 pi /5) * 2), write out your x vector

\qitem

Consider the following signal:

\begin{align*}
f(t) = \cos(\frac{4\pi}{5}t + \frac{\pi}{2})
\end{align*}

Let our sampling frequency be 1 Hz. Starting at t = 0, {\bf sample $f$ at $N = 5$ points to construct $\vec{x}$. } \\
Feel free to leave your answer in terms of the cosine function.

{\em HINT:
We can get samples of a time-domain signal by evaluating it at different points in time. \\
}

\sol{
    \begin{align*}
        \vec{x} = 
        \begin{bmatrix}
        f(0) \\ f(1) \\ f(2) \\ f(3) \\ f(4)
        \end{bmatrix} = 
        \begin{bmatrix}
        \cos(\frac{\pi}{2}) \\ 
        \cos(\frac{13\pi}{10}) \\
        \cos(\frac{21\pi}{10}) \\
        \cos(\frac{29\pi}{10}) \\
        \cos(\frac{37\pi}{10})
        \end{bmatrix}
    \end{align*}
}

\qitem
{\bf Apply the DFT to find $\vec{X}$}, your sampled signal in the frequency domain.

\sol{
    \begin{align*}
        \vec{X} =
        \begin{bmatrix}
        0 && 0 && \frac{5}{2}e^{\frac{\pi}{2}j} && \frac{5}{2}e^{-\frac{\pi}{2}j} && 0
        \end{bmatrix}^T
    \end{align*}
}

\qitem
How do the elements of the $\vec{X}$ correspond to your original signal?

\sol{
    The element at index $2$ is $N$ times the phasor representation of our signal with frequency $\frac{4\pi}{5}$, 
    and the element at index $5 - 2 = 3$ is $N$ times the complex conjugate of that phasor representation.
}

\qitem
Consider the following signals: 
\begin{align*}
f(t) = 3\cos(\frac{2\pi}{5}t)
\end{align*}
\begin{align*}
f(t) = 3\cos(\frac{8\pi}{5}t)
\end{align*}
What do you get for your sample vector, and what do youout of the DFT for each signal? 
Is it what you expect? \\
If it is not what you expect, think about why. What could you do to fix this?

\meta{
    The motivation behind this part is to get students started thinking about aliasing. 
    They should notice that they get the exact same $\vec{x}$ for the two signals, even though the signals have different frequencies.
    Try to guide them towards the realization that we need to sample faster to pick up higher freuencies with the DFT.
}

\sol{
    For both signals, 
    \begin{align*}
        \vec{x} = 
        \begin{bmatrix}
        \cos(\frac{\pi}{2}) \\ 
        \cos(\frac{9\pi}{10}) \\
        \cos(\frac{13\pi}{10}) \\
        \cos(\frac{17\pi}{10}) \\
        \cos(\frac{21\pi}{10})
        \end{bmatrix}
    \end{align*}
    \begin{align*}
        \vec{X} =
        \begin{bmatrix}
        0 && \frac{15}{2} && 0 && 0 && \frac{15}{2}
        \end{bmatrix}^T
    \end{align*}

    We get the same exact frequency domain representation for two different signals because we end up sampling the same points for the two signals. \\
    In order to fix this, we need to sample faster and sample more points.
}


\end{enumerate}


