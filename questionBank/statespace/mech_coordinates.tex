% {\Large \textbf{Mechanical:}}
\qns{Coordinate Change Practice}
\qcontributor{Yi Zhao}


\begin{enumerate}

\qitem Let $\vec{v} = [2,-1]^{T}$. What equation gives the coordinates of $\vec{v}$ in the below basis?

\begin{gather*}
\vec{x} = 
\begin{bmatrix}
4 \\
-2
\end{bmatrix},
\vec{y} = \begin{bmatrix}
-3 \\
-3
\end{bmatrix}
\end{gather*}

\sol{
    
    $$
        \vec{c} = 
        \begin{bmatrix}
        4 & -3 \\
        -2 & -3
        \end{bmatrix}^{-1}
        \begin{bmatrix}
        2 \\
        -1
        \end{bmatrix}
        $$
}


\qitem Let $\vec{v} = [3,3]^{T}$. We are told that $\vec{v}$ is represented in the basis:
\begin{gather*}
\vec{x} = 
\begin{bmatrix}
1 \\
1
\end{bmatrix},
\vec{y} = \begin{bmatrix}
1 \\
-1
\end{bmatrix}
\end{gather*}
What equation gives the coordinates of $\vec{v}$ in the standard basis?

\sol {
    $$
        \vec{c} = 
        \begin{bmatrix}
        1 & 1 \\
        1 & -1
        \end{bmatrix}
        \begin{bmatrix}
        3 \\
        3
        \end{bmatrix}
        $$
}







\qitem Let $\vec{x_2}$ be the result when a linear transformation, which we will refer to as $A$, is applied on $\vec{x_1}$.
$$\vec{x_2} = 
\begin{bmatrix}
3 & -1 \\
-2 & 4
\end{bmatrix}
\vec{x_1}
$$

where $x_1$ and $x_2$ are in the standard basis. \\
Now, let us change to the eigenbasis of $A$, represented by $\vec{v_1}$ and $\vec{v_2}$
$$
\vec{v_1} = 
\begin{bmatrix}
1 \\
-2
\end{bmatrix}, 
\vec{v_2} =
\begin{bmatrix}
-1 \\
1
\end{bmatrix}
$$

$$ V =
\begin{bmatrix}
\vec{v_1} & \vec{v_2}
\end{bmatrix}
$$

Suppose $\vec{z_1}$ and $\vec{z_2}$ are the representations of $\vec{x_1}$ and $\vec{x_2}$ in the eigenbasis. 
So, $\vec{z_1} = V^{-1}\vec{x_1}$ and $\vec{z_2} = V^{-1}x_2$. 
We see that $\vec{z_1}$ and $\vec{z_2}$ are related by $\vec{z_2} = A'\vec{z_1}$. 
What is the matrix $A'$? \\
\textit{
Write your answer in terms of matrix multiplications. 
Given $\lambda_1 = 5$ and $\lambda_2 = 2$, what are the elements of $A'$?
} \\
\sol{
    Start by writing $\vec{x_1}$ in terms of $\vec{z_1}$:
    $$ \vec{x_1} = V \vec{z_1} $$
    Then, apply the transformation $A$ to $\vec{x_1}$, substituting $V \vec{z_1}$ for $\vec{x_1}$:
    $$ \vec{x_2} = A V \vec{x_1} $$
    Finally, left-multiply both sides by $V^{-1}$ to change $\vec{x_2}$ to the eigenbasis:
    $$ \vec{z_2} = V^{-1} \vec{x_2} = V^{-1} A V \vec{z_1} $$
    
    $$A' = V^{-1} A V = 
    \begin{bmatrix}
    1 & -1 \\
    -2 & 1
    \end{bmatrix}^{-1}
    \begin{bmatrix}
    3 & -1 \\
    -2 & 4
    \end{bmatrix}
    \begin{bmatrix}
    1 & -1 \\
    -2 & 1
    \end{bmatrix}
    $$
    
    $A'$ represents the transformation $A$ in the eigenbasis of $A$, so we know that $A'$ is the diagonal matrix:
    $$ A' = 
    \begin{bmatrix}
    \lambda_1 & 0 \\
    0 & \lambda_2
    \end{bmatrix} =
    \begin{bmatrix}
    5 & 0 \\
    0 & 2
    \end{bmatrix} $$
}

\qitem \textbf{True or False}: If a matrix is invertible, it can be diagonalized. \\
\sol{ False, there are some matrices that are invertible but not diagonalizable.
    
}

\qitem \textbf{True or False} There is a unique eigenvalue decomposition for any matrix. \\
\sol{False, you can always find a different basis that diagonalizes the matrix.}






\end{enumerate}