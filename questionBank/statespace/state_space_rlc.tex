\qns{RLC circuit}
   
    Now consider the circuit shown below: 

    \begin{center}
		\begin{circuitikz}[scale=0.8]
			\draw (-1,4) 
			to [V = $V_s$,i=$i_s$] (-1,0)
			(-1,4) to [switch, l^=\mbox{$t = 0$} ] (1,4)
			to [opening switch,l^=\mbox{$t = 0$} ] (1,0) 
			(1,4) to [R = $R_s$] (4,4)
			to [C = $C$,i=$i_C$] (4,0)
			to [short] (-1,0)
			(4,4) to [short] (8,4)
			to [L = $L$,i=$i_L$, v = $V_{out}$] (8,0)
			to [short] (-1,0);
		\end{circuitikz}
	\end{center}
	
\begin{enumerate}
	\qitem Assuming the circuit reaches steady state for $t<0$, find the differential equation for $V_{out}$ for $t\geq0$
	
	\sol{
\[i_s=i_c+i_l\]
Using $i_c=C\frac{dV_{out}}{dt}$ and Ohm's law:
\[\frac{V_s-V_{out}}{R_s}=C\frac{dV_{out}}{dt}+i_l\]

Substituting $V_{out}=L\frac{di_l}{dt}$
\[\frac{V_s}{R_s}=\frac{L}{R_s}\frac{di}{dt}+LC\frac{d^2i}{dt^2}+i_l\]	
\[\frac{V_s}{R_sLC}=\frac{1}{R_sC}\frac{di}{dt}+\frac{d^2i}{dt^2}+\frac{i_l}{LC}\]
Since we have $V_{out}=L\frac{di_l}{dt}$, we take the derivative of both sides to get an equation in the form of $V_{out}$
\[0=\frac{1}{R_sC}\frac{d^2i}{dt^2}+\frac{d^3i}{dt^3}+\frac{1}{LC}\frac{di_l}{dt}\]	
\[0=\frac{d^2V_{out}}{dt^2}+\frac{1}{R_sC}\frac{dV_{out}}{dt}+\frac{1}{LC}V_{out}\]
	}
	\qitem What are the initial conditions at $t=0$ for this differential equation?
	
	\sol{
	
	We know that at $t=0$ $V_{out}=0$, $i_c=0$, and $i_l=0$. We also know that the voltage on a capacitor and the current through an inductor can't change instantaneously, so it follows that
\begin{align*}
V_{out}(0)&=0 \\
i_l(0)&=0
\end{align*}
Consequentially 
\[i_s(0)=\frac{V_s}{R_s}\]	
\[i_c(0)=\frac{V_s}{R_s}\]
from $i_c(0)=C\frac{dV_{out}(0)}{dt}$
\[\frac{dV_{out}(0)}{dt}=\frac{V_s}{R_sC}\]
	}
	\qitem Solve the differential equation. Consider all cases (underdamped, critically damped, overdamped)
	
	\sol{The general solution to the trsolient equation of this form is \[V_{out}(t)=K_1e^{\lambda_1 t}+K_2e^{\lambda_2 t}\]
	where 
	\[\lambda_1=-\alpha +\sqrt{\alpha^2-\omega_0^2}  \]
	\[\lambda_2=-\alpha -\sqrt{\alpha^2-\omega_0^2}\]
	In this case, 
	\[\alpha=\frac{1}{2R_sC}\]
	\[\omega_0=\frac{1}{\sqrt{LC}}\]
	Now to find the constants $K_1$ and $K_2$, we use the initial conditions for $v_{out}(0)$ and $\frac{dv_{out}(0)}{dt}$ and set the general solution to these values:
	\begin{align*}
		\frac{dv_{out}(0)}{dt}=\frac{V_s}{R_sC}&=K_1\lambda_1+K_2\lambda_2 \\
	v_{out}(0)=0&=K_1+K_2 \\
	\frac{V_s}{R_sC}&=-K_2\lambda_1+K_2\lambda_2 \\
	K_2&=\frac{V_s}{R_sC(\lambda_2-\lambda_1)} \\
	K_1&=-\frac{V_s}{R_sC(\lambda_2-\lambda_1)}
	\end{align*}
	Finally, plugging back into the general solution:
	\[V_{out}(t)=-\frac{V_s}{R_sC(\lambda_2-\lambda_1)}e^{\lambda_1 t}+\frac{V_s}{R_sC(\lambda_2-\lambda_1)}e^{\lambda_2 t}\]
	 
	For the $\lambda_1=\lambda_2$, which happens in a critically damped system, the general solution is:
\[V_c=K_1e^{\lambda t}+tK_2e^{\lambda t}\] 
and
\[\frac{dV_c}{dt}=K_1\lambda e^{\lambda t}+K_2e^{\lambda_t}+t K_2 \lambda e^{\lambda t}\]
so when plugging our initial conditions into these equations 
\begin{align*}
V_{out}(0)&=0=K_1 \\
\frac{dV_c (0)}{dt}&=\frac{V_s}{R_sC}=K_1\lambda+K_2 \\
K_2&= \frac{V_s}{R_sC} \\
K_1&=0 \\
K_2&=\frac{V_s}{R_sC}
\end{align*}
	 
	 
	}
	


\end{enumerate}