\qns{Multivariate ODE with Coordinate Changes}
\qcontributor{Anant Sahai}
\qcontributor{Regina Eckert}
\qcontributor{Elena Jia}

\begin{enumerate}

\qitem Consider a system of differential equations (valid for
$t\geq 0$)
\begin{equation}
\frac{d}{dt}y_1(t) = 5 y_1(t) + 2 y_2(t)
\end{equation}
\begin{equation}
\frac{d}{dt}y_2(t) = -8 y_1(t) -5 y_2(t)
\end{equation}

with initial conditions $y_1(0) = 3$ and $y_2(0) = 3$.

Write out the differential equations and initial conditions in matrix/vector form.

\meta{
TA Guidance: Explain that usually they will be getting problems in this form. 

The general procedure for solving this type of problem is: \\
First, convert to the eigenbasis. Solve the problem there. Then convert back to the problem basis to find the final answer.
}

\sol{
	$$\begin{bmatrix}\frac{d}{dt}y_1(t) \\ \frac{d}{dt}y_2(t)\end{bmatrix} = \begin{bmatrix}5 & 2 \\ -8 & -5\end{bmatrix}\begin{bmatrix}y_1(t) \\ y_2(t)\end{bmatrix}$$
	
$$\begin{bmatrix}y_1(0) \\ y_2(0)\end{bmatrix} =\begin{bmatrix}3 \\ 3\end{bmatrix} $$

We will define the differential matrix as $A_y$, where 

$$A_y = \begin{bmatrix}5 & 2 \\ -8 & -5\end{bmatrix} $$
} 

\bigskip

\begin{adjustwidth}{-20pt}{0pt}
	We already know how to solve the system of differential equations if $\frac{d}{dt}y_1(t)$ only depends on $y_1(t)$ and $\frac{d}{dt}y_2(t)$ only depends on $y_2(t)$.
	However, we can't directly solve a system of ODEs where $\frac{d}{dt}y_1(t)$ and $\frac{d}{dt}y_2(t)$ each depend on both $y_1(t)$ and $y_2(t)$. \\
	The solution? Change coordinates to the eigenbasis to diagonalize our transformation matrix.\\
	Then, we will have $\frac{d}{dt}z_{\lambda_1}(t) = \lambda_1 z_{\lambda_1}(t)$ and $\frac{d}{dt}z_{\lambda_2}(t) = \lambda_2 z_{\lambda_2}(t)$, which we know how to solve.
	
\end{adjustwidth}



\qitem Find the eigenvalues $\lambda_1, ~\lambda_2$ and eigenspaces for the differential equation matrix above.

\meta{
TA Guidance:  The determinant is something that finds the
oriented volume of the matrix acting on the standard unit cube
(represented by the identity matrix). This is what they have been
taught about determinants in 16A. As a result, if a matrix has a
nontrivial nullspace, there are directions that it maps to zero. 
\medskip
As a result, it ``squashes'' things that have volume into pancakes or
lines or dots that have no volume. Consequently, the determinant of
a matrix with a nontrivial nullspace must be zero. 
\medskip
Fortunately, the
determinant of a matrix can be easily computed via Gaussian
Elimination (since every step in Gaussian elimination has an
interpretation in terms of what it does to oriented volume --- a
swap of rows negates the sign of the oriented volume (since this is
like looking into a mirror.); scaling a row
scales the oriented volume by the same amount, and adding a multiple
of one row to another does nothing to oriented volumes (for the
same reason that pushing a deck of cards does nothing to its
volume).
\medskip 

They also know that this Gaussian elimination approach can be
done directly on simple symbolic 2x2 and 3x3 matrices to give the
simple formulas that we know. They barely know that the determinant is
the product of its eigenvalues, although that clearly makes sense from
this perspective. That's about all they know about determinants from
16A. 
\medskip
They do not know the cofactor formulas nor do they know the
signed permutation interpretation nor the alternating linear forms
interpretation. Those topics are viewed as voodoo at the maturity
level of 16 students (they would just memorize them without
understanding, so why teach something that they cannot truly
understand) and so are not to be invoked.

}

\sol{
	Eigenvalues $\lambda$ and eigenvectors $v$ of matrix $A$ are given by
	
	$$A_yv = \lambda v$$
	In order to find the eigenvalues, we take the determinant:
	
$$\text{det}\left( A-\lambda I \right) = 0$$

$$\text{det}\left( \begin{bmatrix}5-\lambda & 2 \\ -8 & -5-\lambda\end{bmatrix} \right) = 0$$

We can solve this using a $2 \times 2$ determinant form seen in 16A,

$$\text{det}\left( \begin{bmatrix}a & b \\ c & d\end{bmatrix} \right) = ad - bc,$$

or by Gaussian elimination. \\
This produces the characteristic polynomial of the matrix:

\begin{align*}
(-5-\lambda)(5-\lambda) + 16 &=0 \\
% 12 + 7\lambda + \lambda^2 -2 &=0 \\
% \lambda ^2 + 7\lambda + 10 & = 0 \\
(\lambda + 3)(\lambda -3) &= 0	
\end{align*}

Giving:

$$\lambda = 3, -3$$

The eigenspace associated with $\lambda_1 = -3$ is given by the span of the eigenvector $\vec v_{\lambda_1}$:

	 $$ \begin{bmatrix}-4 + 5 & 1 \\ 2 & -3 + 5\end{bmatrix} \vec v_{\lambda_1} = \begin{bmatrix}0 \\ 0\end{bmatrix}$$
	 	$$ \begin{bmatrix}1 & 1 \\ 2 & 2\end{bmatrix} \vec v_{\lambda_1} = \begin{bmatrix}0 \\ 0\end{bmatrix}$$
	 $$\vec v_{\lambda_1}=
	 \begin{bmatrix} -1 \\ 4\end{bmatrix} $$

The eigenspace associated with $\lambda_2 = 3$ is given by the span of the eigenvector $\vec v_{\lambda_2}$:
	
	 $$ \begin{bmatrix} 
	 -4 + 2 & 1 \\ 
	 2 & -3 + 2
	 \end{bmatrix} 
	 \vec v_{\lambda_2} = 
	 \begin{bmatrix}0 \\ 0\end{bmatrix}
	 $$
	 $$ \begin{bmatrix}
	 	-2 & 1 \\ 
		2 & -1
	\end{bmatrix} 
	\vec v_{\lambda_2} = \begin{bmatrix} 0 \\ 0\end{bmatrix}$$
	$$\vec v_{\lambda_2}=
	\begin{bmatrix} -1 \\ 1\end{bmatrix} $$

}


\qitem Change coordinates into the eigenbasis to re-express the
differential equations in terms of new variables $z_{\lambda_1}(t), ~
z_{\lambda_2}(t)$. (These variables represent eigenbasis-aligned coordinates.) \\
\textit{
 i.e. find a matrix A' such that $\frac{d}{dt} \vec{z_{\lambda}} = A' \vec{z_{\lambda}}$
} \\

\sol{
		\begin{centering}
	\begin{tikzpicture}
	
	\draw (-1,0) node[anchor = east] {$z_{\lambda_i}$ coordinates};
	\draw (0,0) circle (0.5cm);
	\draw (-1,2) node[anchor = east] {$y_i$ coordinates};
	\draw[->] (0,0.5) -- (0,1.5) node[anchor=north east] {$V$};
	\draw (0,2) circle (0.5cm);
	\draw[->] (0.5,2) -- (4.5,2) node[anchor=south east] {$ A_y = \begin{bmatrix}5 & 2 \\ -8 & -5\end{bmatrix}$} ;
	\draw (3,2) node[anchor=north] {differentiation};
	\draw (5,2) circle (0.5cm);
	\draw[->] (5,1.5) -- (5,0.5) node[anchor=south west] {$V^{-1}$};
	\draw (5,0) circle (0.5cm);
	\draw[dashed,->] (0.5,0) -- (4.5,0) node[anchor=south east] {$ A_{z_\lambda} =V^{-1} A V$} ;
	
	\end{tikzpicture}
\end{centering}

$$\vec y =\vec v_{\lambda_1} z_{\lambda_1} + \vec v_{\lambda_2} z_{\lambda_2}$$
$$\vec y =\begin{bmatrix} -1 & -1 \\ 4 & 1\end{bmatrix}\begin{bmatrix}z_{\lambda_1} \\ z_{\lambda_2}\end{bmatrix}$$

We can define the change-of-coordinates matrix from the eigenbasis to our original basis as 

$$V=\begin{bmatrix} -1 & -1 \\ 4 & 1\end{bmatrix}$$

Changing coordinates to the eigenbasis:

$$\begin{bmatrix}z_{\lambda_1} \\ z_{\lambda_2}\end{bmatrix} = V^{-1}\begin{bmatrix}y_1 \\ y_2\end{bmatrix}   $$

$$V^{-1}=\begin{bmatrix}\frac{1}{3} & \frac{1}{3} \\ -\frac{4}{3} & -\frac{1}{3}\end{bmatrix}$$

$$A_{z_\lambda} = V^{-1} A_y V = \begin{bmatrix}\frac{1}{3} & \frac{1}{3} \\ -\frac{4}{3} & -\frac{1}{3}\end{bmatrix}\begin{bmatrix}5 & 2 \\ -8 & -5\end{bmatrix}\begin{bmatrix} -1 & -1 \\ 4 & 1\end{bmatrix}$$
% $$A_{z_\lambda} = \begin{bmatrix}\frac{2}{3} & -\frac{1}{3} \\ \frac{1}{3} & \frac{1}{3}\end{bmatrix}\begin{bmatrix}-5 & -2 \\ 5 & -4\end{bmatrix}$$
$$A_{z_\lambda} = \begin{bmatrix}-3 & 0 \\ 0 & 3\end{bmatrix}$$

That is:

	$$\begin{bmatrix}\frac{d}{dt}z_{\lambda_1}(t) \\ \frac{d}{dt}z_{\lambda_2}(t)\end{bmatrix} = \begin{bmatrix}-3 & 0 \\ 0 & 3\end{bmatrix}\begin{bmatrix}z_{\lambda_1}(t) \\ z_{\lambda_2}(t)\end{bmatrix}$$


}

\qitem Solve the differential equation for $z_{\lambda_i}(t)$ in the eigenbasis.

\sol{
First we get the initial condition:

$$\vec z_{\lambda}(0) = \begin{bmatrix}\frac{1}{3} & \frac{1}{3} \\ -\frac{4}{3} & -\frac{1}{3}\end{bmatrix}\begin{bmatrix} 3 \\ 3 \end{bmatrix} = \begin{bmatrix} 2 \\ -5 \end{bmatrix} $$

Then we solve based on the form of the problem and our previous differential equation experience:

$$\vec z_{\lambda}(t) = \begin{bmatrix} K_1 e^{-3t} \\ K_2 e^{3t} \end{bmatrix} $$

Plugging in for the initial condition gives:

$$\vec z_{\lambda}(t) = \begin{bmatrix} 2 e^{-3t} \\ -5 e^{3t} \end{bmatrix} $$

}

\qitem Convert your solution back into the original coordinates to find $y_i(t)$.

\sol{
$$ \vec y(t) = V \vec z_{\lambda}(t) =  \begin{bmatrix} -1 & -1 \\ 4 & 1\end{bmatrix}\begin{bmatrix} 2 e^{-3t} \\ -5 e^{3t}  \end{bmatrix} = \begin{bmatrix} -2e^{-3t} + 5 e^{3t}\\ 8e^{-3t}-5e^{3t} \end{bmatrix}$$
}


\qitem We can solve this equation using a slightly shorter approach by observing that the solutions for $y_i(t)$ will all be of the form
$$y_i(t) = \sum_k K_{i,k} e^{\lambda_k t}$$

where $\lambda_k$ is an eigenvalue of our differential equation
relation matrix and the $K_{i,k}$ are constants derived from our
initial conditions and the coordinate changes involved. 

Since we have observed that the solutions will include
$e^{\lambda_i t}$ terms, once we have found the eigenvalues for our
differential equation matrix, we can guess the forms of the $y_i(t)$ as

 	$$\begin{bmatrix}y_1(t) \\ y_2(t)\end{bmatrix}
        = \begin{bmatrix}\alpha e^{\lambda_1t} + \beta e^{\lambda_2t}
          \\ \gamma e^{\lambda_1 t}  + \kappa e^{\lambda_2 t} \end{bmatrix}$$
where $\alpha, ~\beta, ~\gamma, ~\kappa$ are all constants.

\begin{enumerate}
	\qitem Take the derivative to write out 
	$$\begin{bmatrix}\frac{d}{dt}y_1(t) \\ \frac{d}{dt}y_2(t)\end{bmatrix}.$$ in matrix-vector form.
	
	\sol{
	 	$$\begin{bmatrix}y_1(t) \\ y_2(t)\end{bmatrix} = \begin{bmatrix}\alpha e^{-3t} + \beta e^{3t}  \\ \gamma e^{-3t}  + \kappa e^{3t} \end{bmatrix}$$
		$$\frac{d}{dt} \vec y(t) = \begin{bmatrix}-3\alpha e^{-3t} +3 \beta e^{3t}  \\ -3\gamma e^{-3t} + 3 \kappa e^{3t} \end{bmatrix}$$
		If we notice that the right-hand side can be written as linear combinations of $e^{-3t}$ and $e^{3t}$, we can write the previous equation as:
		$$ \frac{d}{dt} \vec y(t) = 
		\begin{bmatrix}
			-3 \alpha & 3 \beta  \\ -3 \gamma & 3 \kappa
		\end{bmatrix}
		\begin{bmatrix}
			e^{-3t} \\ e^{3t}
		\end{bmatrix}
		$$
	}
	
	\qitem Connect this differential equation to the matrix-vector equation you found in part (a).\\
	\sol{
		$$\frac{d}{dt} \vec y(t) = 
		\begin{bmatrix}
			5 & 2 \\ 
			-8 & -5
		\end{bmatrix}
		\vec{y}(t) = 
		\begin{bmatrix}
			-3 \alpha & 3 \beta  \\ -3 \gamma & 3 \kappa
		\end{bmatrix}
		\begin{bmatrix}
			e^{-3t} \\ e^{3t}
		\end{bmatrix} 
		$$
		Substituting
		$$ \begin{bmatrix}
			\alpha e^{-3t} + \beta e^{3t}  \\ \gamma e^{-3t}  + \kappa e^{3t} 
		\end{bmatrix} = 
		\begin{bmatrix}
			\alpha & \beta  \\ \gamma & \kappa
		\end{bmatrix}
		\begin{bmatrix}
			e^{-3t} \\ e^{3t}
		\end{bmatrix}
		$$
		for $\vec{y}(t)$, we get:
		$$
		\begin{bmatrix}
			5 & 2 \\ 
			-8 & -5
		\end{bmatrix} 
		\begin{bmatrix}
			\alpha & \beta  \\ \gamma & \kappa
		\end{bmatrix}
		\begin{bmatrix}
			e^{-3t} \\ e^{3t}
		\end{bmatrix} = 
		\begin{bmatrix}
			-3 \alpha & 3 \beta  \\ -3 \gamma & 3 \kappa
		\end{bmatrix}
		\begin{bmatrix}
			e^{-3t} \\ e^{3t}
		\end{bmatrix} 
		$$		
		Doing the matrix multiplication on the left-hand side of the equation, we get:
		$$
		\begin{bmatrix}
			5 \alpha + 2 \gamma & 5 \beta + 2 \kappa \\ 
			-8 \alpha - 5 \gamma & -8 \beta - 5 \kappa
		\end{bmatrix} 
		\begin{bmatrix}
			e^{-3t} \\ e^{3t}
		\end{bmatrix} = 
		\begin{bmatrix}
			-3 \alpha & 3 \beta  \\ -3 \gamma & 3 \kappa
		\end{bmatrix}
		\begin{bmatrix}
			e^{-3t} \\ e^{3t}
		\end{bmatrix} 
		$$		
	}
	
	\qitem Use what you found in the previous step to solve for $\gamma$ and $\kappa$ in terms of $\alpha$ and $\beta$, respectively. \\
	\sol {
		Equating terms in the matrices on the left- and right-hand sides of the equation, we get:
		$5 \alpha + 2 \gamma = -3 \alpha$, 
		$5 \beta + 2 \kappa = 3 \beta$,
		$-8 \alpha - 5 \gamma = -3 \gamma$, and 
		$-8 \beta - 5 \kappa = 3 \kappa$.
		
		From the first equation, we get $\gamma = -4 \alpha$. From the second, we get $\kappa = - \beta$. \\
		We could get the same result from using the third and fourth equations. 
	}
	
	\qitem Use initial conditions to finish solving for $\vec{y}(t)$. \\
	\sol {
		From the initial condition, we have:
		$$\vec{y}(0) = 
		\begin{bmatrix} 3 \\ 3 \end{bmatrix} = 
		\begin{bmatrix}
			\alpha & \beta  \\ \gamma & \kappa
		\end{bmatrix}
		\begin{bmatrix}
			e^{-3 \cdot 0} \\ e^{3 \cdot 0}
		\end{bmatrix} = 
		\begin{bmatrix}
			\alpha & \beta  \\ \-4 \alpha & - \beta
		\end{bmatrix}
		\begin{bmatrix}
			1 \\ 1
		\end{bmatrix}$$
		From this, we get the equations $\alpha + \beta = 3$ and $-4 \alpha - \beta = 3$.
		$$\alpha=-2$$
	 	$$\beta=5$$
		
		We can now plug in our 4 constants to find $\vec{y}(t)$:
		$$\begin{bmatrix}y_1(t) \\ y_2(t)\end{bmatrix} = \begin{bmatrix} -2e^{-3t} + 5e^{3t}  \\ 8e^{-3t}  -5  e^{3t} \end{bmatrix}$$
		Note that this is the same as the answer from part (e)
	}
\end{enumerate}

\end{enumerate}