% Author: Taejin Hwang
% Email: taejin@berkeley.edu

\qns{An Introduction to Systems}

In the next couple of worksheets, we will be examining systems. 
Many physical systems such as the motion of a car, can be modeled using a system.
Often times, when we are describing a system, we will have a \textbf{state variable $\vec{x},$}
that will often be a multivariable function. 
For a given system, we can often write a differential equation describing its change over time as
\begin{align}
\ddt{\vec{x}(t)}{t} = f(\vec{x}(t))
\end{align}

In this problem, we will examine a specific form of systems that can be put in \textbf{State-space representation.} \vspace{0.5 cm}
For linear systems (we will define what it means to be linear later) the state-space representation is:
\begin{align}
\ddt{\vec{x}(t)}{t} = A \vec{x}(t) + B \vec{u}(t)
\end{align}

Where $A$ is the $n \times n$ state matrix, $\vec{x}$ is a state vector in $\mathbb{R}^n$, $B$ is a $n \times d$ input matrix, and $\vec{u}$ is an input vector in $\mathbb{R}^d$. We will usually consider a $B$ as a vector in $\mathbb{R}^n$ and $u(t)$ will be a scalar input.

Given the following system:
\begin{align*}
    \ddt{}{t}x_{1}(t) &= 3 x_{1}(t) - 2 x_{2}(t) + 4 \\
    \ddt{}{t}x_{2}(t) &= - x_{1}(t) + 5 x_{2}(t) + 2
\end{align*}

With initial conditions $x_{1}(0) = 2, \ x_{2}(0) = 3,$

\begin{enumerate}
    \qitem What is an appropriate state vector for this system?

    \sol{
        We have to variables $x_1$ and $x_2$ therefore we define our state vector as:
        $$\vec{x} = \begin{bmatrix} x_{1} \\ x_{2} \end{bmatrix}$$
    }

    \qitem What is the initial condition of this system?

    \sol{
        We have the individual initial conditions for $x_1$ and $x_2$ but we must also a define an initial condition for our state vector.
        $$\vec{x}(0) = \begin{bmatrix} x_{1}(0) \\ x_{2}(0) \end{bmatrix} = \begin{bmatrix} 2 \\ 3 \end{bmatrix} $$
    }

    \qitem Write out the system of differential equations as in state-space form.
    
    \sol{
        $$\ddt{}{t}\vec{x}(t) = \begin{bmatrix} 3 & -2 \\ -1 & 5 \end{bmatrix} \begin{bmatrix} x_{1}(t) \\ x_{2}(t) \end{bmatrix} 
        = \begin{bmatrix} 3 & -2 \\ -1 & 5 \end{bmatrix} \vec{x}(t) + \begin{bmatrix} 4 \\ 2 \end{bmatrix}$$
    }
\end{enumerate}