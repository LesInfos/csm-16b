\qns{Transistor Power Consumption}

In this problem, we'll investigate the energy consumption of a NOR gate while switching from low to high.

First let's look at the energy dissipated when we charge up an RC circuit.

\begin{figure}[H]
	\begin{centering}
		\begin{circuitikz}
			\draw (0, 4)
			to[V =$V_s$] (0, 0);
			\draw (0, 4)
			to[switch,l^=\mbox{$t = 0$}](4,4)
			(4,4) to[R = $R$,v=$V_R(t)$,i>^=$i_R(t)$] (7,4)	
			to [short] (9,4)
			to[C = $C$, v=$V_C(t)$,i>^=$i_C(t)$] (9,0)
			to [short] (0,0);
		\end{circuitikz}
		\caption{\label{fig:circuit}RC Circuit with Voltage Source}
	\end{centering}
\end{figure}


Assume that at time $t = 0$, the capacitor is fully uncharged.

\begin{enumerate}

\qitem We are interested in the energy dissipated by the resistor, $\Delta E_R = \int_0^\infty P_R(t) dt$, where $P_R(t)$ is the power consumption of the resistor. 

Using what you know about RC circuits, what is $P_R(t)$? Recall $P = IV = I^2R = V^2 / R$.


\sol{
\begin{align*}
\intertext{Remember that the voltage of a charging RC circuit over time is:}
V_C(t) = V_s(1 - e^{\frac{-t}{RC}})
\intertext{Solving for $V_R$:}
V_R(t) = V_s - V_c(t) = V_se^{\frac{-t}{RC}}
\intertext{So $P_R(t) = V^2 / R = (V_s^2 / R)\cdot e^{\frac{-2t}{RC}}$}
\end{align*}
}


\qitem How much energy is consumed by the resistor as the capacitor charges?


\sol{
\[\Delta E_R = \int_0^\infty P_R(t) dt = \int_0^\infty (V_s^2 / R)\cdot e^{\frac{-2t}{RC}} dt\]
\[\Delta E_R = \frac{1}{2}CV_s^2 (e^{\frac{-2t}{RC}})\Big|_0^\infty = -\frac{1}{2}CV_s^2\]

Note that the negative value for $\Delta E_R$ means that energy is being consumed.
}

\qitem Given that the capacitor stores $\frac{1}{2}CV_s^2$ Joules of energy when charging, how much energy is supplied by the voltage source? 

Why do we not include the energy stored by the capacitor in the energy dissipated by the circuit?

\sol{
Add the energy dissipated by the resistor and energy stored by the capacitor: $CV_s^2$

Alternatively, you can calculate $\Delta E_{V_s}$ by integrating the power supplied by the source:

\begin{figure}[H]
	\begin{centering}
		\begin{circuitikz}
			\draw (0,4)
			to[I=$i_s(t)$, V =$V_s$] (0, 1);
            \draw (0,1)
      to (0,1) node[ground] {};
			\draw (0, 4)
			to (1,4) node[label={right:$\cdots$}] {};
		\end{circuitikz}
	\end{centering}
\end{figure}

\[\Delta E_{V_s} = \int_0^\infty V_s i_s(t)dt = V_s\int_0^\infty-i_c(t)dt\]
\[= \int_0^\infty \frac{dQ_c}{dt}dt = V_s(Q_c) \big|_0^\infty = CV_s^2\]

We don't count the energy stored by the capacitor when looking at energy consumed because the energy is being stored and can be released later, so it isn't energy lost.
}
\end{enumerate}

Now let's look at a NOR gate, with a capacitor between the output and ground.

\begin{figure}[H]
	\begin{centering}
        \begin{circuitikz}
            \draw (6,5) node[pmos, emptycircle](pm){} ;
            \draw (6,3) node[pmos, emptycircle](nm){} ;
            \draw (6,1) node[nmos]
            (n){} ;
            \draw (3.5,1) node[nmos]
            (nq){} ;
            \draw (pm.gate) to [short, -*](1,5) node[left]{$V_{in, 1}$};
            \draw [short, *-] (2,5) to (2,1);
            \draw [short] (2,1) to (nq.gate);
            \draw (nq.drain) to [short](3.5,2) to [short, -*](6,2) ;
            \draw (pm.source) to [short, -*] (6,6) node[above]{$V_{DD}$};
            \draw (pm.drain) to (nm.source);
            \draw (nm.gate) to [short, -*](4, 3) to [short](4,1)  to (n.gate);
            \draw [short](4,3) to [short, -*](1, 3) node[left]{$V_{in, 2}$};
            \draw (6,2) to [short, -*] (9,2) node[right]{$V_{o}$};
            \draw (8,2) to [C = $C_\ell$] (8,-1) to 
            (8,-1) node[ground]{};
            \draw (n.drain) to (nm.drain);
            \draw (n.source) to [short] (6, -1) to (3.5, -1) to (nq.source);
            \draw [short] (4.75, -1) to (4.75,-1) node[ground]{};
        \end{circuitikz}
        \caption{\label{fig:circuit}NOR Gate}
	\end{centering}
\end{figure}


Assume that the output of the NOR gate has been $0$ for a long time before $t=0$, where $V_{in, 1}$ and $V_{in, 2}$ both switch to $0$.

\begin{enumerate}[resume]
\qitem Using the RC model of transistors, redraw the above circuit at $t\geq 0$ as a simple RC circuit.

\textit{Remember that a MOSFET can be modeled as a voltage-controlled switch and resistor between the source and drain, and a capacitor between the gate and source.}

\sol{
    Both NMOS switches are open, and both PMOS switches are closed.

    \begin{figure}[H]
	\begin{centering}
        \begin{circuitikz}
            \draw [short](0,4) to [short, -*] (0,4.5) node[above]{$V_{DD}$};
            \draw (0,-2) to [C=$C_\ell$](0,0) to [R=$R$] (0,2) to[R = $R$] (0,4);
            \draw [short] (0,-2) to (0,-2) node[ground]{};
            \draw [short,-*] (0,0) to (1,0) node[right]{$V_o$};
        \end{circuitikz}
         \caption{\label{fig:circuit}Simple RC NOR Gate}
	\end{centering}
\end{figure}

    Note: the gate capacitors aren't connected to the rest of the circuit, so we can ignore them.
}

\qitem Using your result from part (b), how much energy is lost when the NOR gate switches on?

\sol{
    \begin{figure}[H]
	\begin{centering}
		\begin{circuitikz}
			\draw (0, 4)
			to[V =$V_{DD}$] (0, 0);
			\draw (0, 4)
			to (1,4)
			(1,4) to[R = $2R$] (4,4)	
			to [short] (5,4)
			to[C = $C_\ell$, v=$V_{C}(t)$] (5,0)
			to [short] (0,0);
            \draw (0,0)
      to (0,0) node[ground] {};
		\end{circuitikz}
		\caption{\label{fig:circuit}RC Circuit with Voltage Source}
	\end{centering}
\end{figure}


\[\Delta E_R = -\frac{1}{2}C_\ell V_{DD}^2\]
}

\qitem OPTIONAL: How much energy is lost when it switches from high to low? Where does this energy come from?

\sol{
$\frac{1}{2}C_\ell V_{DD}^2$, this energy was stored in the capacitor and released as it discharged.
}


\end{enumerate}
