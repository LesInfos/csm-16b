%Authors: Kevin Zhu and Naomi Sagan

\qns{Digital Logic and NAND gates} \\
NAND gates are universal, which means that every boolean operation can be reprsented with a combination of NAND gates. In this problem, we'll examione the transister configuration of a NAND gate and then create an OR gate our of NAND gates. 

\begin{enumerate}
    \item A digital logic circuit typically consists of a pull-down network of NMOS transistors between the output and ground, and a pull-up nework of PMOS transistors between $V_{DD}$ and the output. 
    \begin{enumerate}
        \qitem We want the pull-down network to connect the output to ground when A NAND B = 0 \\
        When would A NAND B output 0? Design a pull-down netowrk for the NAND gate. \\
        \sol {
        $$\text{A NAND B = 0 when A=1 and B = 1}$$
        \begin{figure}[H]
	\begin{centering}
        \begin{circuitikz}
            \draw (6,1) node[nmos]
            (nm){} ;
            \draw (6,3) node[nmos]
            (nm1){} ;
            \draw (nm.gate) to [short, -*](3,1) node[left]{$V_{A}$};
            \draw (nm1.gate) to [short, -*](3, 3) node[left]{$V_{B}$};
            \draw (nm1.drain) to [short, -*] (6,4) node[above]{$V_{o}$};
            \draw (nm1.source) to (nm.drain);

            \draw (nm.source) to (6,0) node[ground]{};
        \end{circuitikz}
        \caption{\label{fig:circuit}NAND pulldown Gate}
	\end{centering}
\end{figure}

        }
        \qitem We want the pull-up network to connect the output to $V_{DD}$ when A NAND B = 1.
        Design a pull-up network for the NAND gate. \\
        \sol{
        \begin{figure}[H]
	\begin{centering}
        \begin{circuitikz}
            \draw (6,5) node[pmos, emptycircle](pm){} ;
            \draw (3.5,5) node[pmos, emptycircle](pm1){} ;
            \draw (pm1.gate) to [short, -*](1,5) node[left]{$V_{A}$};
            \draw (pm.source) to [short, -*] (6,6) node[above]{$V_{DD}$};
            \draw (pm1.source) to [short, -*] (3.5,6) node[above]{$V_{DD}$};
            \draw (pm.drain) to (6,4);
            \draw [short] (4.5,3) to (4.5,5);
            \draw [short] (4.5,5) to (pm.gate);
            \draw [short](4.5,3) to (1, 3) node[left]{$V_{B}$}
            \draw (pm1.drain) to (3.5,4) to [short] (7,4) node[right]{$V_o$};
        \end{circuitikz}
        \caption{\label{fig:circuit}NAND pullup Gate}
	\end{centering}
\end{figure}
        }
        \qitem Now put the two networks together to create a NAND gate. \\
        \sol{\begin{figure}[H]
	\begin{centering}
        \begin{circuitikz}
            \draw (6,5) node[pmos, emptycircle](pm){} ;
            \draw (3.5,5) node[pmos, emptycircle](pm1){} ;
            \draw (6,1) node[nmos]
            (nm){} ;
            \draw (6,3) node[nmos]
            (nm1){} ;
            \draw (pm1.gate) to [short, -*](1,5) node[left]{$V_{A}$};
            \draw (nm1.gate) to [short, -*](1, 3) node[left]{$V_{B}$};
            \draw [short, *-] (2,5) to (2,1);
            \draw [short] (2,1) to (nm.gate);
            \draw (pm.source) to [short, -*] (6,6) node[above]{$V_{DD}$};
            \draw (pm1.source) to [short, -*] (3.5,6) node[above]{$V_{DD}$};
            \draw (pm.drain) to (nm1.drain);
            \draw (nm1.source) to (nm.drain);
            \draw [short, *-] (4.5,3) to (4.5,5);
            \draw [short] (4.5,5) to (pm.gate);
            \draw (pm1.drain) to (3.5,4) to [short] (6,4) node[right]{$V_o$};
            \draw (nm.source) to (6,0) node[ground]{};
        \end{circuitikz}
        \caption{\label{fig:circuit}NAND Gate}
	\end{centering}
\end{figure}
}
    \end{enumerate}
    Now that we have successfully created a NAND gate, let us create an OR gate from NAND gates. 
    \qitem First, let's create a intermediate tool to help us build our OR gate. How can we build an inverter using just NAND gates? \\
    \sol{
    $$\text{NOT (A) = A NAND A}$$
    \begin{figure}[H]
	\begin{centering}
        \begin{circuitikz}
            \draw (2,1) node[nand port] (nand) {};
            \draw (0,0.7) node[left]{$A$} to (nand.in 2);
            \draw (0,1.3) node[left]{$A$} to (nand.in 1);
            \draw (nand.out) to (3,1) node[right]{NOT A};
        \end{circuitikz}
        \caption{\label{fig:circuit}NAND Inverter}
	\end{centering}
\end{figure}

    When A is 0, A NAND A = 1. Likewise when A =1, A NAND A = 0.
    }
    \qitem Now, let's examine an OR gate. How can you make an OR gate using NOT and NAND gates?
    
    \emph{Hint: draw out a truth table if you get stuck.} \\
    \sol{$$\text{A OR B = (NOT A) NAND (NOT B)}$$
	\begin{figure}[H]
	\begin{centering}
        \begin{circuitikz}
            \draw (2,1) node[nand port] (nand) {};
            \draw (0,0.7) node[left]{$A$} to (nand.in 2);
            \draw (0,1.3) node[left]{$A$} to (nand.in 1);
            \draw (2,3) node[nand port] (nand1) {};
            \draw (0,2.7) node[left]{$B$} to (nand1.in 2);
            \draw (0,3.3) node[left]{$B$} to (nand1.in 1);
            \draw (4,2) node[nand port] (nand2) {};
            \draw (nand1.out) to (2.5, 2.3) to (nand2.in 1)
            \draw (nand.out) to (2.5,1.7) to (nand2.in 2);
            \draw (nand2.out) to (5,2) node[right]{A OR B};
        \end{circuitikz}
        \caption{\label{fig:circuit}Or gate}
	\end{centering}
\end{figure}
}
    \qitem Finally, replace the NOT gates with their NAND equivalents to make an OR gate solely out of NAND gates. \\
    \sol{$$\text{A or B = (NOT A) NAND (NOT B) = (A NAND A) NAND (B NAND B)}$$
    See the solution to part (c) for the logic gate diagram!
    }
    \qitem Challenge: Now try making an XOR gate using only NAND gates. \\
    \sol{You can rewrite A XOR B as (A OR B) AND (A NAND B). Additionally, we can rewrite an AND gate as A AND B = NOT(A NAND B). Thus, we can plug in our gates that we have already created to get:
    \begin{figure}[H]
	\begin{centering}
        \begin{circuitikz}
            \draw (2,1) node[nand port] (nand) {};
            \draw (0,0.7) node[left]{$A$} to (nand.in 2);
            \draw (0,1.3) node[left]{$A$} to (nand.in 1);
            \draw (2,3) node[nand port] (nand1) {};
            \draw (0,2.7) node[left]{$B$} to (nand1.in 2);
            \draw (0,3.3) node[left]{$B$} to (nand1.in 1);
            \draw (4,2) node[nand port] (nand2) {};
            \draw (nand1.out) to (2.5, 2.3) to (nand2.in 1)
            \draw (nand.out) to (2.5,1.7) to (nand2.in 2);
            \draw (2,5) node[nand port] (nand3) {};
            \draw (0,4.7) node[left]{$B$} to (nand3.in 2);
            \draw (0,5.3) node[left]{$A$} to (nand3.in 1);
            \draw (6.5,3.5) node[nand port] (nand4) l
            \draw (nand2.out) to (5,2) to (nand4.in 2);
            \draw (nand3.out) to (5,5) to (nand4.in 1);
            \draw (8.4,3.5) node[nand port] (nand5) {};
            \draw (nand4.out) to (7,3.2) to (nand5.in 2); 
            \draw (nand4.out) to (7,3.8) to (nand5.in 1);
            \draw (nand5.out) to (9,3.5) node[right]{A XOR B};
        \end{circuitikz}
        \caption{\label{fig:circuit}XOR gate}
	\end{centering}
\end{figure}
}
\end{enumerate}
