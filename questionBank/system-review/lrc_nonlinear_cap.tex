\qns{LRC Circuit With a Nonlinear Capacitor}

In this problem, we'll consider an LRC circuit with a nonlinear capacitor, which we have drawn as a capacitor with a box around it.
Because of this nonlinearity, we'll try to understand our system around an operating point through linearization.
We'll then use stability and controllability to see if we can stabilize the system using feedback.

\begin{center}
    \begin{circuitikz}[american, scale=0.8, transform shape]
        \draw (0, 0) node[left]{$u(t)$} to[L=$L$, *-, i^>=$x_2(t)$] (3, 0)
        (3, 0) to[R=$R$, -*] (6, 0)
        (6, 0) to[short, -o] (6.5, 0) node[right] {$x_1(t)$}
        (6, 0) to[C] (6, -2) node[ground] {};
        \node[draw,minimum width=1cm,minimum height=1.5cm,anchor=north west] at (5.5,-0.3){};
        \draw [line width=0.08cm] (5.5, -1.8) -- (6.5, -1.8);
    \end{circuitikz}
\end{center}

We know that the capacitor has the following voltage-current relationship:
\begin{align*}
    \frac{d}{dt} v(t) = \exp\{\alpha i(t) + \beta\} - 1,
\end{align*}
for some unknown $\alpha$ and $\beta$.

\begin{enumerate}
    \qitem Before doing any circuit analysis, we must find the values $\alpha$ and $\beta$ that characterize our capacitor.

    To do so, we attach the capacitor to a voltage source and a resistor. We then apply different voltage inputs and measure the initial $I_c$ and rate of change of $V_c$. \\

    From this process, we get the data points
    \begin{align*}
        (i_0,\, \frac{d}{dt}v_0),\, (i_1, \frac{d}{dt}v_1), \cdots, (i_n, \frac{d}{dt}v_n).
    \end{align*}
    \textbf{Given these data points, set up a least squares problem to find $\alpha$ and $\beta$.}

    \sol{
        In order to perform least squares, we need a linear equation in terms of $\alpha$ and $\beta$. So, we need to do some algebraic manipulation on the capacitor's current-voltage relation.

        We will add $1$ to both sides, and then take the natural log of both sides.
        $$\frac{d}{dt} v(t) + 1 = \exp\{\alpha i(t) + \beta\}$$
        \begin{equation}
            \alpha i(t) + \beta = \ln\bigg(\frac{d}{dt} v(t) + 1\bigg) \label{eq:nonlin}
        \end{equation}

        Now, we want to write out an equation in the form
        \begin{align*}
            \vec{b} = D\begin{bmatrix}
                \alpha \\ \beta
            \end{bmatrix}.
        \end{align*}

        Using \eqref{eq:nonlin}, we can write out the following matrix-vector equation:

        \begin{align*}
            \begin{bmatrix}
                \ln\bigg(\frac{d}{dt} v_0 + 1\bigg) \\
                \vdots \\
                \ln\bigg(\frac{d}{dt} v_n + 1\bigg)
            \end{bmatrix} =
            \begin{bmatrix}
                i_0 & 1 \\
                \vdots & \vdots \\
                i_n & 1
            \end{bmatrix} \begin{bmatrix}
                \alpha \\ \beta
            \end{bmatrix}.
        \end{align*}

        We can then use least squares to find our unknowns:
        \begin{align*}
            \begin{bmatrix} \alpha \\ \beta \end{bmatrix} =
            (D^T D)^{-1} D^T \vec{b}.
        \end{align*}
    }

    \qitem Assume we found $\alpha = 2$ and $\beta = 0$.

    \textbf{Write out the state-space differential equations for our circuit} in terms the state variables $x_1$ and $x_2$ and input $u(t)$ labeled on the circuit above.

    \meta {
    Why can't we put this in matrix vector form?
    }

    \sol {
        Recall that we defined $x_1(t)$ as the voltage across the capacitor, $x_2(t)$ as the current through the inductor, and $u(t)$ as the potential to the left of the inductor.

        We can use the current-voltage relation for the nonlinear capacitor to get our first state-space equation:
        \begin{align*}
            \frac{d}{dt} x_1(t) = \exp\{2x_2(t)\} - 1.
        \end{align*}

        To get the second equation, use the current-voltage relation of the inductor:
        \begin{align*}
            \frac{d}{dt} x_2(t) = \frac{V_L(t)}{L} = \frac{u(t) - Ry_2(t) - y_1(t)}{L} \\
            \frac{d}{dt} x_2(t) = \frac{1}{L} u(t) - \frac{1}{L} y_1(t)(t) - \frac{R}{L} y_2(t)
        \end{align*}
    }

    \qitem \textbf{Now, linearize the system around operating point ($x_1$,\, $x_2$,\, $u$) = ($0$, $0$, $0$).}

    Write the state-space equations in matrix-vector form.

    \textit{Bonus: Is this linearization a good representation of the circuit for all values of $x_1$, $x_2$, and $u$? Explain.}

    \meta{
    Ask students what it means to linearize. Which parts of the system are we linearizing? And why?
    }

    \sol {
        The second equation is already linear, so we only have to linearize the first one:
        \begin{align*}
            \frac{d}{dt} x_1(t) = f(x_2(t)) = e^{2x_2(t)} - 1.
        \end{align*}

        Our operating point is ($x_1$,\, $x_2$,\, $u$) = ($0$, $0$, $0$), so our linear approximation is
        \begin{align*}
            \frac{d}{dt} x_1(t) = f(x_2(t) = 0) + \frac{d}{d x_2} f(x_2(t))\bigg\rvert_{x_2(t) = 0} (x_2(t) - 0) \\
            \frac{d}{dt} x_1(t) = \big(e^{2 \cdot 0} - 1\big) + \big(2e^{2x_2(t)}\big)\bigg\rvert_{x_2(t) = 0} x_2(t) \\
            \frac{d}{dt} x_1(t) = 2x_2(t)
        \end{align*}

        Now, we can write our state-space equations in matrix-vector form as follows:
        \begin{align*}
            \frac{d}{dt} \begin{bmatrix}
                x_1(t) \\ x_2(t)
            \end{bmatrix} =
            \begin{bmatrix}
                0 & 2 \\
                -1/L & -R/L
            \end{bmatrix} \begin{bmatrix}
                x_1(t) \\ x_2(t)
            \end{bmatrix} +
            \begin{bmatrix}
                0 \\ 1/L
            \end{bmatrix} u(t).
        \end{align*}

        Note that this linearization assumes that our state is close to the operating point, i.e. $x_2(t)$ is always close to 0. If $x_2(t)$ is large, this linear approximation is no longer a good model for our system.
    }

    \qitem For the rest of the problem, assume that $R = 0$ and $L = 1/2$.

    \textbf{Find the eigenvalues of the system. Is it stable?}

    \meta{
    Ask students to predict what the eigenvalues will look like. We have no resistance after all.
    }

    \sol {
        $R = 0$ and $L = 1/2$, so our state-space model is now:
        \begin{align*}
            \frac{d}{dt} \begin{bmatrix}
                x_1(t) \\ x_2(t)
            \end{bmatrix} =
            \begin{bmatrix}
                0 & 2 \\
                -2 & 0
            \end{bmatrix} \begin{bmatrix}
                x_1(t) \\ x_2(t)
            \end{bmatrix} +
            \begin{bmatrix}
                0 \\ 2
            \end{bmatrix} u(t).
        \end{align*}

        To find the eigenvalues, set the determinant of $A - \lambda I$ equal to $0$:
        \begin{align*}
            \lambda^2 + 4 = 0 \\
            \lambda = \pm 2i
        \end{align*}

        This system is not stable (technically, marginally unstable) because the real component of the eigenvalues is nonnegative.

        Recall the LRC problem you saw in module 1: this is the undamped case, which means that the initial condition will only oscillate and never decay.
    }

    \qitem \textbf{Is the system controllable?}

    \sol {
        Yes, the system is controllable.

        You can calculate the controllability matrix as
        \begin{align*}
            \mathcal{C} = \bigg\{
                \begin{bmatrix} 0 \\ 2 \end{bmatrix},\,
                \begin{bmatrix}
                0 & 2 \\
                -2 & 0
            \end{bmatrix}
            \begin{bmatrix} 0 \\ 2 \end{bmatrix}
            \bigg\} = \bigg\{
                 \begin{bmatrix} 0 \\ 2 \end{bmatrix},\,
                  \begin{bmatrix} 4 \\ 0 \end{bmatrix}
            \bigg\}
        \end{align*}
        The controllability matrix spans $\mathbb{R}^2$, so the system is controllable.
    }

    \qitem We want to apply feedback control to stabilize the system. Say we apply the input
    \begin{align*}
        u(t) = \begin{bmatrix} f_1 & f_2\end{bmatrix} x(t).
    \end{align*}
\textbf{Find feedback coefficients $f_1$ and $f_2$ such that the resulting system is stable and has purely real and negative eigenvalues.}

\textit{Note: you have some flexibility in terms of what feedback coefficients you choose. If you're stuck, start out by writing the characteristic polynomial in terms of $f_1$ and $f_2$, and then try plugging in different values $f_1$ and $f_2$.}

\sol {
    After applying feedback control, our system will be:

    \begin{flalign*}
        \frac{d}{dt} \vec{x}(t) &=
            \begin{bmatrix}
                0 & 2 \\
                -2 & 0
            \end{bmatrix} \vec{x}(t) +
            \begin{bmatrix}
                0 \\ 2
            \end{bmatrix} \begin{bmatrix}
                f_1 & f_2
            \end{bmatrix} \vec{x}(t) \\
            &= \begin{bmatrix}
                0 & 2 \\
                2f_1 - 2 & 2f_2
            \end{bmatrix} \vec{x}(t).
    \end{flalign*}

    Following the tip in the note, we will write out the characteristic polynomial in terms of $f_1$ and $f_2$.

    \begin{align*}
        \text{det} (A - \lambda I) = 0 \\
        -\lambda(2f_2 - \lambda) - 4f_1 + 4 = 0 \\
        \lambda^2 - 2\lambda f_2 - 4f_1 + 4 = 0.
    \end{align*}

    In order for the system to be stable, the eigenvalues must have a negative real component.
    Using the quadratic formula, we know that for $a\lambda^2 + b\lambda + c = 0$,
    \begin{align*}
    \lambda = \frac{-b \pm \sqrt{b^{2} - 4ac}}{2a}
    \end{align*}

    In this case, $a = 1$,  $b = -2f_2$ , and $c = -4f_1 + 4$.
    To get only a negative real component, the following must hold:
    \begin{itemize}
    \item $b^{2}-4ac \geq 0$ in order to not have an imaginary component.
    \item $-b \pm \sqrt{b^{2} - 4ac} < 0$ in order for $\lambda$ to be negative.
    \item $b > 0$ or else the numerator will be positive by condition 1.
    \end{itemize}

    Substituing in values for our first condition, we have
    \begin{align*}
    b^2 \geq 4ac \implies 4f_{2}^{2} \geq -16f_1 + 16 \\
     f_{2}^{2} \geq -4f_{1} + 4 \\
    \end{align*}
    We can simplify our second condition by noticing that $\sqrt{b^{2} - 4ac}$ must be nonnegative.
    Thus, we just need to analyze $-b + \sqrt{b^{2} - 4ac} < 0$ since fulfilling this condition implies the entire second condition is fulfilled.
    Thus,
    \begin{align*}
    -b < -\sqrt{b^{2} - 4ac} \implies b > \sqrt{b^{2} - 4ac}\\
    b^{2} > b^{2} - 4ac \\
    4ac > 0 \\
    f_1 < 1
    \end{align*}
    Finally, we also, add the restriction that $f_2 < 0$ by condition 3.


}
\end{enumerate}
