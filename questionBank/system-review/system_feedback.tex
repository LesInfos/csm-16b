% Authors: Yi Zhao, Taejin Hwang
% Emails: taejin@berkeley.edu

\qns{System Feedback}

Consider the following continuous time system:
$$
\frac{d^2}{dt^2} x(t) = -x(t)
$$
We convert this system to state space form:

\begin{equation}
  \ddt{}{t} \vec{x}(t) = A \vec{x}(t) + B \vec{u}(t)
\end{equation}

We will pick our state variable $\vec{x} = \begin{bmatrix} x_1(t) \\ x_2(t) \end{bmatrix},$
with $x_1(t) = x(t)$ and $x_2(t) = \frac{d}{dt} x(t)$.

\begin{enumerate}

\qitem What are the values of the $A$ matrix?

\sol {
	$$
    A = \begin{bmatrix}
    0 & 1 \\
    -1 & 0
    \end{bmatrix}
    $$
}

% \qitem How about the matrix $B$ and the input vector $\vec{u}(t)?$

% \sol {
%   The current system can be modeled as:
%   \begin{equation}
%   \ddt{}{t} \vec{x}(t) = A \vec{x}(t)
%   \end{equation}
%   We do not have an input vector for the system, nor do we have a matrix $B.$ \vskip 1pt
%   We can treat them as zero, but the dimensions of $B$ and $\vec{u}(t)$ are currently undefined.
% }

\qitem Is this continuous time system stable? How would you describe its behavior?

\sol{
	The eigenvalues of the $A$ matrix will be $\pm j.$
  Since $\mathfrak{Re}(\lambda) = 0,$ and is not less than $0,$ the system is unstable.
  It can be described as an oscillatory system that never dies out.
}

\end{enumerate}

We want to change the behavior of the system using a feedback control model:
$$
\frac{d}{dt}
\begin{bmatrix}
x_1(t) \\
x_2(t)
\end{bmatrix}
=
A
\begin{bmatrix}
x_1(t) \\
x_2(t)
\end{bmatrix}
+
\begin{bmatrix}
0 \\
1
\end{bmatrix}
u(t)
$$
To do this, we set $u(t) = K\vec{x}(t)$, where $K = \begin{bmatrix} k_1 & k_2 \end{bmatrix}$. Note that $K$ is a $1 \times 2$ matrix.

\meta {
  In the generalized state-space model, $B$ is an $n \times p$ matrix and $\vec{u}$ will be a vector in $\mathbb{R}^{p}.$ Therefore, since $\vec{u} = K \vec{x},$ and $\vec{x}$ is in $\mathbb{R}^{n},$ our $K$ matrix in the general state-space model will be of size $p \times n.$
}

\begin{enumerate}[resume]
\qitem We will now try to use our knowledge of controllability to stabilize this system. Is this continuous time system controllable?

\meta {
  Make sure to emphasize that we can only pick every eigenvalue of the system if it is controllable.
  We can also make an unstable system stable, if we can control all of the unstable eigenvalues, but we would not be able to control \textbf{every} eigenvalue for the system.
}

\sol{
  We can compute our controllability matrix as
  $$\mathcal{C} = \begin{bmatrix} \vec{b} & A \vec{b} \end{bmatrix} =
  \begin{bmatrix}
  0 & 1 \\
  1 & 0
  \end{bmatrix}$$
  This controllability matrix is full-rank since its columns are linearly independent, so the system is controllable.
}

\item After plugging in our input, $u(t) = K\vec{x}(t),$ what will our new system be? How can you write out this system of differential equations in matrix/vector form?

\sol {
  We know that our $B$ matrix is: $\begin{bmatrix} 0 \\ 1 \end{bmatrix}$ and our input $u(t) = k_1 x_1(t) + k_2 x_2(t).$ \vskip 1pt
  If we plug this into our system, we will get:
  \begin{equation}
    \ddt{}{t} \vec{x}(t) =
    \begin{bmatrix}
    0 & 1 \\
    -1 & 0
    \end{bmatrix}
    \begin{bmatrix} x_1(t) \\ x_2(t) \end{bmatrix} +
    \begin{bmatrix}
    0 & 0 \\
    k_1 & k_2 \\
    \end{bmatrix}
    \begin{bmatrix} x_1(t) \\ x_2(t) \end{bmatrix}
  \end{equation}

  This means our new system will be:
  \begin{equation}
    \ddt{}{t} \vec{x}(t) =
    \begin{bmatrix}
    0 & 1 \\
    -1 + k_1 & k_2
    \end{bmatrix}
    \begin{bmatrix} x_1(t) \\ x_2(t) \end{bmatrix}
  \end{equation}
  We can alternatively represent this as a matrix/vector system of differential equations:
  \begin{equation}
    \ddt{}{t} \vec{x}(t) = (A + BK) \vec{x}(t)
  \end{equation}
}

\qitem We will define the new matrix derived in the previous part as $A_{cl}$ for the closed loop system.
What are the eigenvalues of this new matrix $A_{cl}?$

\sol {
  We first compute the characteristic polynomial of $A_{cl}.$
  $$\text{det}(A_{cl} - \lambda I) =
  \text{det}
  \bigg(
  \begin{bmatrix}
    -\lambda & 1 \\
    -1 + k_1 & k_2 - \lambda
  \end{bmatrix}
  \bigg) = \lambda(\lambda - k_2) - 1(-1 + k_1) = \lambda^{2} - k_2 \lambda + 1 - k_1$$
  The eigenvalues will be the roots of this characteristic polynomial:
  $$\lambda = \frac{k_2}{2} \pm \frac{1}{2} \sqrt{k_2^{2} - 4(1 - k_1)}$$
}

\qitem Suppose $k_1 = 1$ and $k_2 = 2,$ what will the eigenvalues of this system be?
How about when $k_1 = -1$ and $k_2 = -2$?

\sol{
	If $k_1 = 1$ and $k_2 = 2,$ then the eigenvalues will be:
  $$1 \pm \frac{1}{2} \sqrt{4 - 4(1 - 1)} = 0, \ 2$$
  Since both of the eigenvalues have $\mathfrak{Re}(\lambda) = 0, 2 >= 0,$ this system will be unstable.

  Now suppose $k_1 = -1$ and $k_2 = -2,$ then the eigenvalues will be:
  $$-1 \pm \frac{1}{2} \sqrt{4 - 4(1 + 1)} = -1 \pm j$$
  Since $\mathfrak{Re}(\lambda) = -1 < 0,$ for both eigenvalues, this system will be stable.
}

\qitem What values of $k_1$ and $k_2$ will remove the oscillatory behavior completely and still stabilize the system?

\meta{
Have students sketch out/visualize and understand what varying k do to the eigenvalues.
}

\sol{
	To remove oscillatory behavior, and stabilize the system, the square root term $\sqrt{k_2^{2} - 4(1 - k_1)}$ must be positive, and the real part of both eigenvalues must be less than 0.

	As example of this would be $k_1 = -1$ and $k_ 2 = -3.$
  This gives the system eigenvalues of $-1$ and $-2$ while removing oscillatory behavior and being stable.
}

\end{enumerate}
