\qns{Filter Design}
% \qcontributor{Maxwell Chen}

\begin{enumerate}
    \qitem Suppose you have been hired to design a biomedical sensor that can detect and output recordings of Alpha brainwaves in the frequency range 8Hz to 13Hz. Unfortunately, our sensor is faulty: it is also picking up Gamma brainwaves in the frequency range 32Hz to 100Hz, interfering with our ability to get clean recordings of alpha brainwaves. We want to create a new design for our sensor that can remove this interference, giving us a clearer signal.
    \begin{enumerate}
        \qitem What kind of filter could you use to remove this interference?
	\sol {
	We should use a low-pass filter. The interference is at a higher frequency than our desired signal, so we filter out the higher frequencies and keep the lower frequencies.
	}	
	\\

        \qitem Assume we only have access to resistors and capacitors. Sketch the corresponding filter circuit and write out its transfer function.
	\sol {
	$H(\omega) = \frac{\widetilde{Z_{out}}}{\widetilde{Z_{in}}} = \frac{\frac{1}{j\omega C}}{R + \frac{1}{j \omega C}} = \frac{1}{1 + j\omega RC} = \frac{1}{1 + j \frac{\omega}{\omega_p}}$.
	}	
	\\

        \qitem $\omega_p = \frac{1}{RC}$ is the pole frequency which determines the frequency at which our filter starts attenuating the signal. Should we maximize or minimize $\omega_p$ to remove as much of the interference as possible?
	\sol {
	We should minimize the pole frequency. The transfer function will start filtering frequencies sooner, meaning that we maximize the amount by which higher frequencies are attenuated.
	}	
	\\

        \qitem Say we can set $\omega_p$ to 10Hz, 20Hz, 32Hz, 100Hz, or 120Hz. Which is the best pole frequency, and why?
	\sol {
	20Hz. This is the lowest pole frequency we can choose; the Gamma brainwaves will be attenuated more if we choose 20Hz than 32Hz because the transfer function will start filtering at a lower frequency. 10Hz is too low and risks cutting off Alpha brainwaves.
	}	
	\\

        \qitem We only have a 33 $\Omega$ resistor in our workstation. What capacitor value should we use for our filter?
	\sol {
	If we want a pole frquency of 20Hz, then $\omega_p = \frac{1}{RC} = \frac{1}{33C} = 20Hz$. So, $C = \frac{1}{33*20Hz} = 1.51 mF$.
	}	
	\\

        \qitem Draw the Bode plot for the magnitude and phase response of this filter.
	\sol {
	}	
	\\

        \qitem \textbf{Optional.} Say that our filter is successful at removing the interference we initially detected. We now notice that there is another source of interference--Delta brainwaves--in the frequency range 0.5Hz to 4Hz. How could we modify our design to remove both sources of interference and get a clear recording of Alpha brainwaves?
	\sol {
	Replace our low-pass filter with a bandpass filter. This will attenuate both sources of noise and leave our frequency band for Alpha brainwaves intact for our sensor to record.
	}	
	\\

    \end{enumerate}
\end{enumerate}