% Author: Taejin Hwang
% Email: taejin@berkeley.edu

\qns{QR Decomposition} 

While we have this amazing speedup through Gram-Schmidt, it is important to realize that our current signals $\vec{q}_{i}$ were modified to be orthonormal, when the actual signals $\vec{s}_{i}$ were not necessarily orthogonal. Our current solution $\hat{x}$ is in $Q$ basis coordinates as opposed to standard basis coordinates. Therefore, to tackle this, we will use a concept called \textbf{QR Decomposition}. 

\begin{enumerate}

  \item The $Q_{i}$ matrix in the previous question was constructed by performing Gram-Schmidt on the columns of $S_{i}.$
  We also know that the columns of the $Q_{i}$ matrix form a basis for the $\text{Col}(S_{i}).$ How can you represent the columns of $S_{i}$ using the basis $\{\vec{q}_{1}, \cdots, \vec{q}_{i} \}?$

  \sol {
    Since the columns of $Q_{i}$ form a basis for the $\text{Col}(S_{i}),$ any vector in $\text{Col}(S_{i}),$ can be represented as a linear combination of $\{\vec{q}_{1}, \cdots, \vec{q}_{i}\}.$
    $$\vec{s}_{1} = \alpha_{1} \vec{q}_{1} + \dotsc + \alpha_{i} \vec{q}_{i}$$
    We can solve for these $\alpha$ coefficients by taking the inner product of both sides with $\vec{q}_{1}.$
    \begin{align*}
    \innp{\vec{q}_{1}}{\vec{s}_{1}} &= \innp{\vec{q}_{1}}{\alpha_{1} \vec{q}_{1} + \dotsc + \alpha_{i} \vec{q}_{i}} \\
    &= \alpha_{1} \innp{\vec{q}_{1}}{\vec{q}_{1}} + \dotsc + \alpha_{i} \innp{\vec{q}_{1}}{\vec{q}_{i}} \\
    &= \alpha_{1} \innp{\vec{q}_{1}}{\vec{q}_{1}} = \alpha_{1}
    \end{align*}
    We can do this similarly to solve for any $\vec{s}_{j}$ to say that
    $$\vec{s}_{j} = \innp{\vec{q}_{1}}{\vec{s}_{j}} \vec{q}_{1} + \dotsc + \innp{\vec{q}_{j}}{\vec{s}_{j}} \vec{q}_{j}$$
  }

  \item While we have derived an expression for the coefficients $\alpha_{j},$ we can realize that most of the coefficients will be zero due to the way $q_{j}$ was constructed from $S.$ Try writing out the vectors $\vec{q}_{1}, \vec{q}_{2}, \vec{q}_{3}$ using the Gram-Schmidt algorithm, to conclude which coefficients will be zero.

  \sol {
    Let's first look at the vector $\vec{q}_{1}.$ We notice that $\vec{s}_{1}$ only depends on $\vec{q}_{1}.$
    $$\vec{q}_{1} = \frac{\vec{s}_{1}}{\norm{\vec{s}_{1}}}$$
    Therefore, only $\alpha_{1}$ will be nonzero, and all of the other $\alpha_{j}$ will be zero, due to the orthogonality of $\vec{q}_{j}.$ \\
    Now let's look at the vector $\vec{q}_{2}.$
    $$\vec{q}_{2} = \vec{s}_{2} - \innp{\vec{s}_{2}}{\vec{q}_{1}} \vec{q}_{1}$$
    Solving for $\vec{s}_{2},$ we see that it is only in terms of $\vec{q}_{1}$ and $\vec{q}_{2}$
    $$\vec{s}_{2} = \innp{\vec{s}_{2}}{\vec{q}_{1}} \vec{q}_{1} + \vec{q}_{2}$$
    Again, we conclude that $\alpha_{1}$ and $\alpha_{2}$ will be nonzero, but all other $\alpha_{j}$ will be zero.
    Now we look at $\vec{q}_{3}.$ We can again solve for $\vec{s}_{3}$ and conclude that it is only in terms of $\vec{s}_{1}, \vec{s}_{2}, \vec{s}_{3}.$
    \begin{align*}
    \vec{q}_{3} &= \vec{s}_{3} - \innp{\vec{s}_{3}}{\vec{q}_{1}} \vec{q}_{1} - \innp{\vec{s}_{3}}{\vec{q}_{2}} \vec{q}_{2} \\
    \vec{s}_{3} &= \innp{\vec{s}_{3}}{\vec{q}_{1}} \vec{q}_{1} + \innp{\vec{s}_{3}}{\vec{q}_{2}} \vec{q}_{2} + \vec{q}_{3} 
    \end{align*}
    We can continue doing this, and will continue noticing that for $\vec{s}_{j},$ all of the $\alpha_{1}, \cdots, \alpha_{j}$ will be nonzero, and $\alpha_{j+1}, \cdots, \alpha_{i}$ will be zero.
  }

  \item Using the expression derived from the previous part, how can you represent the vector $\vec{s}_{j}$ as a matrix-vector equation $\vec{s}_{j} = Q \vec{r}_{j}$

  \sol {
    Rewriting out the expression for $\vec{s}_{j},$ we get:
    $$\vec{s}_{j} = \innp{\vec{q}_{1}}{\vec{s}_{j}} \vec{q}_{1} + \dotsc + \innp{\vec{q}_{j}}{\vec{s}_{j}} \vec{q}_{j}$$
    Remember that for any matrix-vector equation can be written as a linear combination of the columns and vice versa.
    $$\vec{s}_{j} = \begin{bmatrix}
    | & | & | \\
    \vec{q}_{1} & \cdots & \vec{q}_{i} \\
    | & | & |
    \end{bmatrix} \begin{bmatrix} \innp{\vec{q}_{1}}{\vec{q}_{1}} \\ \vdots \\ \innp{\vec{q}_{j}}{\vec{s}_{j}} \end{bmatrix}$$
    Therefore, $\vec{r}_{j}$ can be represented as the following:
    $$\vec{r}_{j} = \begin{bmatrix} \innp{\vec{q}_{1}}{\vec{q}_{1}} \\ \vdots \\ \innp{\vec{q}_{j}}{\vec{s}_{j}} \end{bmatrix}$$
    Using the fact that all of the $\alpha$ values past $j + 1$ are zero, we conclude by saying
    $$\vec{r}_{j} = \begin{bmatrix} \innp{\vec{q}_{1}}{\vec{s}_{j}} \\ \vdots \\ \innp{\vec{q}_{j}}{\vec{s}_{j}} \\ 0 \\ \vdots \\ 0 \end{bmatrix}$$
  }

  \item Let's assume that the matrix $Q_{j}$ was constructed to have orthonormal columns. Using the $\vec{r}_{j}$s derived from the previous part, how can you write out the matrix equation $S_{i} = Q_{i} R$ where $R$ is an upper triangular matrix.

  \sol {
    We can aggregate the vectors $\vec{s}_{j}$ into the matrix $S_{i}$ and since all of the $\vec{s}_{j}$ were written as a linear combination of vectors in the $Q_{i}$ matrix, the $R$ matrix can be constructed by aggregating the $\vec{r}_{j}$ vectors.
    $$S_{i} = \begin{bmatrix}
    | & | & | \\
    \vec{s}_{1} & \cdots & \vec{s}_{i} \\
    | & | & |
    \end{bmatrix} = \begin{bmatrix}
    | & | & | \\
    \vec{q}_{1} & \cdots & \vec{q}_{i} \\
    | & | & |
    \end{bmatrix} \cdot \begin{bmatrix}
    \innp{\vec{q}_{1}}{\vec{s}_{1}} & \cdots & \innp{\vec{q}_{1}}{\vec{s}_{i}} \\
    \vdots & \ddots & \vdots \\
    0 & 0 & \innp{\vec{q}_{i}}{\vec{s}_{i}}
    \end{bmatrix}$$
  }


  \item What are the reprsentations of $\hat{y}$ using Q-basis coordinates and standard basis coordinates?

  \sol {
    The estimate $\hat{y}$ using coordinates from the $Q$ basis can be represented as:
    $$ \hat{y} = Q_{i} [\hat{x}]_{Q} = \begin{bmatrix}
    | & | & | \\
    \vec{q}_{1} & \cdots & \vec{q}_{m} \\
    | & | & |
    \end{bmatrix} \begin{bmatrix} x_{q_{1}} \\ \vdots \\ x_{q_{m}} \end{bmatrix} $$
    In standard basis coordinates, it will be: 
    $$ \hat{y} = S \hat{x} = \begin{bmatrix}
    | & | & | \\
    \vec{s}_{1} & \cdots & \vec{s}_{m} \\
    | & | & |
    \end{bmatrix} \begin{bmatrix} x_{1} \\ \vdots \\ x_{m} \end{bmatrix} $$
    Equating the two, we see that:
    $$S \hat{x} = Q_{i} [\hat{x}]_{Q}$$
  }

  \item How can we use our relation $S_{i} = Q_{i} R$ to solve for $\hat{x}?$

  \sol {
    Substituting $S_{i} = Q_{i} R,$ we see that
    $$Q_{i} R \hat{x} = Q_{i} [\hat{x}]_{Q}$$
    Since $Q_{i}$ has orthonormal columns, we know that $Q_{i}^{T} Q_{i} = I.$ Therefore, if we left multiply by $Q_{i}^{T}$
    $$Q_{i}^{T} Q_{i} R \hat{x} = R \hat{x} = Q_{i}^{T} Q_{i} [\hat{x}]_{Q} = [\hat{x}]_{Q}$$
    Therefore, taking the $R^{-1}$ we see that
    $$\hat{x} = R^{-1} [\hat{x}]_{Q}$$
  }

  \item How can we use the fact that $R$ is upper triangular to solve for $\hat{x}$ efficiently?

  \sol {
    Since $R$ is upper triangular, it is already in reduced echlon form. That means we can back-substitute to solve for $\hat{x}$ which is much more efficient than taking the inverse of $R.$ If $R$ is an $n \times n$ matrix, taking the inverse will require $O(n^{3})$ operations while back-substitution will only require $O(n^{2})$ operations. \\
    You can read more about the comparison of matrix inversion vs back-substitution here: \\
    \url{https://en.wikipedia.org/wiki/Computational_complexity_of_mathematical_operations}
  }

\end{enumerate}
