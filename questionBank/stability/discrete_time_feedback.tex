\qns{Discrete Time Feedback}
\qcontributor{Elena Jia}
   
\begin{enumerate}

\qitem Consider the scalar system: $x(i+1) = 1.5 x(i) + u(i)$. Given the controller $u(i) = k x(i)$, for what value of $k$ can we have the system to behave like:$x(i+1) = \lambda x(i)$ where $\lambda = 0.7$?

\sol{
	To make the system's eigenvalue be 0.7, we can choose $k=-0.8$.
}


\qitem Given the system $\vec{x}(i+1)  = \left [ \begin{array}{cc} 3&0\\0&5 \end{array}\right] \vec{x}(i) + \left[\begin{array}{c}1\\0 \end{array}\right] u_1(i)+ \left[\begin{array}{c}0\\1 \end{array}\right] u_2(i)$.
Let $u_1(i) = k_1 x_1(i)$ and $u_2(i) = k_2 x_2(i)$. What value of $k_1$ and $k_2$ would make the system stable with eigenvalues $\lambda_1 = \lambda_2 = \frac{1}{2}$?

\sol{We can choose $k_1= -2.5$ and
$k_2= -4.5$.}










\qitem Given the matrix $\begin{bmatrix}
2  & 1 \\
-3 + 2k_1 & 4+2k_2
\end{bmatrix}$,
what should $k_1$ and $k_2$ be for the matrix to have eigenvalues $\lambda_1=\frac{1}{2}$ and $\lambda_2=\frac{-1}{3}$?

\sol{
Coefficient match to find $k_1 = \frac{-1}{4}$ and $k_2 = \frac{-35}{12}$
}






\qitem Given the matrix $\begin{bmatrix}
2+ k_1  & 7 +k_2 \\
3 & -1
\end{bmatrix}$,
what should $k_1$ and $k_2$ be for the matrix to have eigenvalues $\lambda_1=5$ and $\lambda_2=2$?

\sol{
Coefficient match to find $k_1 = 6$ and $k_2 = -13$
}







\qitem Given the system $\vec{x}(t+1)  = \left [ \begin{array}{cc} 2&-1\\1&2 \end{array}\right] \vec{x}(t) + \left[\begin{array}{c}0\\1 \end{array}\right] u(t)$.
Is the system stable?

\sol{
For a discrete system to be stable, $|\lambda_i| < 1$
Characteristic polynomial:
\begin{align*}
(2-\lambda)^2+1 &= 0 \\
\lambda^2-4\lambda+5 &= 0\\
\lambda_{1,2} = \frac{4\pm \sqrt{16-20}}{2} &= 2\pm j \\
|\lambda_1|=|\lambda_2| &= \sqrt{5} > 1
\intertext{The magnitude of the eigenvalues are greater than 1, so the system is unstable.
} 
\end{align*}
}



\qitem Given the feedback controller $u(t) = K\vec{x}(t)$ for the previous system, where 
$K = \begin{bmatrix}
k_1 & k_2
\end{bmatrix}$. What should $k_1$ and $k_2$ be for the system to reach $\vec{x}(t)=0$ from any states in 2 time steps?


\sol{
\begin{align*}
\intertext{The system can be written as:}
\vec{x}(t+1) &= (A+BK)\vec{x}(t)
\intertext{For this system to converge in 2 steps, we need:}
\lambda_1=\lambda_2&=0
\intertext{Where $\lambda_1$ and $\lambda_2$ are the eigenvalues of $(A+BK)$}
A+BK &= \begin{bmatrix}
2 & -1 \\
1+k_1 & 2+k_2
\end{bmatrix}
\intertext{Characteristic polynomial:}
(2-\lambda)(2+k_2-\lambda)+1+k_1 &= \lambda^2+\lambda(-4-k_2)+5+k_1+2k_2
\intertext{Coefficient match to:}
(\lambda+0)(\lambda+0) &= \lambda^2 \\
-4-k_2 &= 0 \Rightarrow k_2 = -4 \\
5+k_1+2k_2 = 5+k_1-8 &= 0 \Rightarrow k_1=3
\end{align*}
}


% \qitem Given the system $\frac{dx(t)}{dt}= 
% \begin{bmatrix}
% 3 & 1 \\
% 5 & -1
% \end{bmatrix} x(t) +
% \begin{bmatrix}
% 0 \\
% 3
% \end{bmatrix} u(t)$. Is this system stable?

% \sol{
% The system is stable if the real part of both eigenvalues are negative.
% \begin{align*}
% \text{det}(\lambda I-A) &= (\lambda-3)(\lambda+1)-5 = 0 \\
% \lambda^2-2\lambda-8 &= (\lambda-4)(\lambda+2) = 0 \\
% \lambda_1 &= 4 \\
% \lambda_2 &= -2
% \intertext{$\lambda_1 \geq 0$, so the system is unstable} 
% \end{align*}
% }



% \qitem Is the previous \system controllable? 

% \sol{
% \begin{align*}
% R_n &= \begin{bmatrix}
% B & AB
% \end{bmatrix} = \begin{bmatrix}
% 0 & 1 \\
% 1 & -1
% \end{bmatrix}
% \end{align*}
% $R_n$ is full rank, so the system is controllable.
% }



% \qitem Using state feedback, find $k_1$ and $k_2$ that makes the system stable with $\lambda_1=\lambda_2=-2$

% \sol{
% State feedback sets the input $u(t) = \begin{bmatrix}
% k_1 & k_2
% \end{bmatrix} x(t)$
% \begin{align*}
% \frac{dx}{dt}&=(A+BK)x(t) = \begin{bmatrix}
% 3 & 1 \\
% 5+3k_1 & -1+3k_2
% \end{bmatrix} \\
% \text{det}(\lambda I-(A+BK)) &= \lambda^2-(2+3k_2)\lambda-8+9k_2-3k_1 = 0
% \intertext{We need to coefficient match with:}
% (\lambda+2)(\lambda+2)&=\lambda^2+4\lambda+4 \\
% -2+3k_2 &= 4 \Rightarrow k_2=-2 \\
% -8+9k_2-3k_1 &= 4 \\
% -8-18-3k_1 &=4 \Rightarrow k_1 = -10
% \end{align*}
% }



\qitem Given the system  $\vec{x}(t+1) = \begin{bmatrix} 0 & 1 \\ 0 & 0 \end{bmatrix} \vec{x}(t) + \begin{bmatrix} 0 \\ 1 \end{bmatrix} u(t)$. Is the system controllable? What should $k_1$ and $k_2$ be to put the eigenvalues of the system at $\lambda = -1 \pm j$

\sol{
The controllability matrix
\[
\mathcal{C} = \begin{bmatrix} B & AB \end{bmatrix} = \begin{bmatrix} 0 & 1 \\ 1 & 0 \end{bmatrix}
\]
has full rank. So the system is controllable.


A state feedback controller has the form
\[
u(t) = K\vec{x}(t) = \begin{bmatrix} k_1 & k_2 \end{bmatrix} \begin{bmatrix} x_1(t) \\ x_2(t) \end{bmatrix}.
\]
With this control the closed loop system is
\[
\vec{x}(t+1) = (A + BK) \vec{x}(t) = \begin{bmatrix} 0 & 1 \\ k_1 & k_2 \end{bmatrix} \vec{x}(t).
\]
The characteristic equation of the closed loop system is
\[
0 = \det(A - \lambda I) = \det \begin{bmatrix} -\lambda & 1 \\ k_1 & k_2 - \lambda \end{bmatrix} = \lambda^2 - k_2 \lambda - k_1.
\]
The desired closed loop characteristic equation is
\[
0 = (\lambda - (-1 + j))(\lambda - (-1 - j)) = \lambda^2 + 2 \lambda + 2
\]
So we should choose $k_1 = k_2 = -2$.	

}



\qitem Given the system $ \vec{z}(t+1) = \begin{bmatrix}
0 & 1 & 0 \\
0 & 0 & 1 \\
0 & 5 & 11
\end{bmatrix}
\vec{z}(t) +
\begin{bmatrix}
0 \\ 0 \\ 1
\end{bmatrix} u(t)$. Design state feedback so that the system has eigenvalue $0,1/2,-1/2$.

\sol{
The closed loop system in $z$ coordinates is given by
\[
\widetilde{A} + \widetilde{B} \widetilde{K} = \begin{bmatrix} 0 & 1 & 0 \\ 0 & 0 & 1 \\ k_1 & k_2 +5 & k_3 +11 \end{bmatrix}
\]
which has characteristic polynomial $\lambda^3 + (-11 -k_3) \lambda^2 + (-5 -k_2) \lambda - k_1$. To place the eigenvalues at 0, 1/2, -1/2, the desired characteristic polynomial is $\lambda(\lambda - \frac{1}{2})(\lambda + \frac{1}{2}) = \lambda^3 - \frac{1}{4} \ \lambda$. So we should choose $k_1 = 0, k_2 = \frac{-19}{4}, k_3 = -11$.	

}





\end{enumerate}