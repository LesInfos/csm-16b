\qns{Continuous Time System}
\qcontributor{Yi Zhao}

\begin{enumerate}

\qitem Consider the following continuous time system:
$$
\frac{d^2}{dt^2} x(t) = -x(t)
$$
We convert this system to state space form with $x_1(t) = x(t)$ and $x_2(t) = \frac{d}{dt} x(t)$. Our representation becomes:
$$
\frac{d}{dt}
\begin{bmatrix}
x_1(t) \\
x_2(t)
\end{bmatrix}
=
A
\begin{bmatrix}
x_1(t) \\
x_2(t)
\end{bmatrix}
$$
\\ \\
What is the correct values in $A$?

\sol {
	$$
    A = \begin{bmatrix}
    0 & 1 \\
    -1 & 0
    \end{bmatrix}
    $$
}

\qitem
Which best describes the behavior of the above system?

\sol{
	
	Unstable since the eigenvalues are $i$ and $-i$. Noise can cause oscillations that never die out 
}


\qitem
We want to change the behavior of the system using a feedback control model:
$$
\frac{d}{dt}
\begin{bmatrix}
x_1(t) \\
x_2(t)
\end{bmatrix}
=
A
\begin{bmatrix}
x_1(t) \\
x_2(t)
\end{bmatrix}
+ 
\begin{bmatrix}
0 \\
1
\end{bmatrix}
u(t)
$$
We set $u(t) = K\vec{x}(t)$, where $K = [k_1~k_2]$. What values of $k_1$ and $k_2$ will keep the oscillatory behavior of the system but stabilize it? Does $k_1 = -1$ and $k_ 2 = -2$ work? What about $k_1 = 3$ and $k_ 2 = 2$?

\sol{
	
	$k_1 = -1$ and $k_ 2 = -2$ will keep the oscillatory behavior and make the system stable, while $k_1 = 3$ and $k_ 2 = 2$ wouldn't. We can see this by computing the eigenvalues of the system after adding control.
}



\qitem
What values of $k_1$ and $k_2$ will remove the oscillatory behavior completely and still stabilize the system?

\sol{
	
	As an example, using feedback values $k_1 = -1$ and $k_ 2 = -3$ gives system eigenvalues of  -1 and -2, thus the system is not oscilllatory since they are both real and remains stable because they are negative.
}




\end{enumerate}
