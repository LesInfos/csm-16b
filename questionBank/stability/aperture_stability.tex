\qns{Aperture Stability}
\qcontributor{Akash Velu}
\qcontributor{Shreyas Krishnaswamy}

As an intern at Aperture Laboratories, it is your job to make sure the robots being built are stable systems.
As a reminder, a system is stable if the following conditions are met:

\begin{itemize}

\item $|\lambda{}| < 1$, for all eigenvalues in matrix A, for discrete time systems of the form:\\
$\vec{x}(t+1) = A\vec{x}(t) + Bu(t) + \vec{w}(t)$

\item $re(\lambda{})< 0$, for all eigenvalues in matrix A, for continuous time systems of the form:\\
$\frac{d}{dt}\vec{x}(t) = A\vec{x}(t) + Bu(t) + \vec{w}(t)$

\end{itemize}

\begin{enumerate}

\qitem According to your boss, the first robot, GLaDOS, can be described with the following discrete time system:\\

\begin{equation*}
    \vec{x}(t+1) =
    \begin{bmatrix}
    \frac{3}{8} & \frac{1}{8}\\
    \frac{1}{8} & \frac{3}{8}\\
    \end{bmatrix}
    \vec{x}(t) +
    \begin{bmatrix}
    1\\
    0\\
    \end{bmatrix}
    u(t)
\end{equation*}

Is she stable?

\sol {

\begin{align*}
A\vec{v} &= \lambda\vec{v} \\
A\vec{v} - \lambda\vec{v} &= 0 \\
(A - \lambda{}I)\vec{v} &= 0 \\
det(A - \lambda{}I) &= 0
\end{align*}

\begin{equation*}
det(A - \lambda{}I) =
\begin{vmatrix}
\frac{3}{8} - \lambda{} & \frac{1}{8}\\
\frac{1}{8} & \frac{3}{8} - \lambda{}\\
\end{vmatrix} = 0
\end{equation*}

\begin{align*}
det(A - \lambda{}I) &= (\frac{3}{8} - \lambda{})(\frac{3}{8} - \lambda{}) - (\frac{1}{8})(\frac{1}{8})\\
&= \lambda{}^2 - \frac{3}{4}\lambda{} + \frac{1}{8}\\
&= (\lambda{} - \frac{1}{2})(\lambda{} - \frac{1}{8}) = 0\\
\lambda{} &= \{\frac{1}{2}, \frac{1}{8}\}
\end{align*}

Since both eigenvalues are lower than 1 in magnitude and the system is a discrete time system, GLaDOS is stable.

}

\qitem
Now your boss gives you data on another robot, Atlas.
Is he stable?
His movements can be described with the following continuous time system:\\

\begin{equation*}
    \frac{d}{dt}\vec{x}(t) =
    \begin{bmatrix}
    -2 & -1\\
    1 & -2\\
    \end{bmatrix}
    \vec{x}(t) +
    \begin{bmatrix}
    1\\
    1\\
    \end{bmatrix}
    u(t)
\end{equation*}

\sol{

\begin{align*}
A\vec{v} &= \lambda\vec{v} \\
A\vec{v} - \lambda\vec{v} &= 0 \\
(A - \lambda{}I)\vec{v} &= 0 \\
det(A - \lambda{}I) &= 0
\end{align*}

\begin{equation*}
det(A - \lambda{}I) =
\begin{vmatrix}
-2 - \lambda{} & -1\\
1 & -2 - \lambda{}\\
\end{vmatrix} = 0
\end{equation*}

\begin{align*}
det(A - \lambda{}I) &= (-2 - \lambda{})(-2 - \lambda{}) - (-1)(1)\\
&= \lambda{}^2 - 4\lambda{} + 5\\
&= (\lambda{} - (-2 + i))(\lambda{} - (-2 - i)) = 0\\
\lambda{} &= \{-2 + i, -2 - i\}
\end{align*}

Since the real parts of both eigenvalues are lower than 0 and the system is a continuous time system, Atlas is stable.

}

\qitem Lastly, your boss gives you data on the Wheatley robot. Is he stable?
His motion is described with the following discrete time system:\\

\begin{equation*}
    \vec{x}(t+1) =
    \begin{bmatrix}
    \frac{\sqrt{3}}{2} & -\frac{1}{2}\\
    \frac{1}{2} & \frac{\sqrt{3}}{2}\\
    \end{bmatrix}
    \vec{x}(t) +
    \begin{bmatrix}
    0\\
    0\\
    \end{bmatrix}
    u(t)
\end{equation*}

\sol {

Note that A is the rotation matrix.
Without even doing any calculations, we know Wheatley is stable since for any bounded input, he has a bounded output.
After all, he is only able to rotate!\\

Good job intern!
You've helped progress the field of science by confirming that Aperture Science never makes unstable machines.
Never.

}

\end{enumerate}
