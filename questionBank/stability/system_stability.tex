% Author: Taejin Hwang
% Email: taejin@berkeley.edu

\qns{System Stability}

In the previous question, we took a look at stability for systems modeled by a \textbf{scalar} differential equation. 
We will now extend this to systems modeled as a \textbf{matrix/vector} differential equation of the form:
\begin{equation}\label{eqn:std}
  \ddt{}{t} \vec{x}(t) = A \vec{x}(t) + B \vec{u}(t)
\end{equation}
To do this, we will use a technique called \textbf{Schur-Decomposition} which says that for any linear operator, we can pick a basis in which the linear operator has an upper triangular matrix representation:
\begin{equation}\label{eqn:schur}
  \ddt{}{t} \vec{z}(t) = R \vec{z}(t) + \widetilde{B} \vec{u}(t)
\end{equation}
Since $\widetilde{B}$ is a matrix of scalars, we can redefine $\vec{\widetilde{{u}}}(t)$ as a new input vector.
\begin{equation}\label{eqn:input}
  \ddt{}{t} \vec{z}(t) = R \vec{z}(t) + \vec{\widetilde{{u}}}(t)
\end{equation}
In this case, $R$ will be the upper triangular matrix of the form: $$R = \begin{bmatrix} 
      \lambda_{1} & r_{1,2} & \cdots & r_{1,n} \\ 
      0 & \lambda_{2} & \cdots & r_{2,n} \\ 
      \vdots & \ddots & \cdots & \vdots \\
      0 & \cdots & 0 & \lambda_{n}
      \end{bmatrix}$$

The entries above the diagonal in the matrix $R$ are all arbitrary constants. The basis which represents $A$ in this upper-triangular fashion is $V = \{ \vec{v}_{1}, \cdots, \vec{v}_{n} \}$, which can be picked to be \textit{orthonormal}. Therefore, we say that $A = VRV^{T}$ where $V$ is the change of coordinates matrix from $z$ coordinates to $x$ coordinates, or $\vec{x} = V \vec{z}$.

\begin{enumerate}
  \qitem \textbf{Show that if the system represented by equation \eqref{eqn:schur} is stable in $z$ coordinates, then it must also be stable in $x$ coordinates.}

  \ws{\vspace{150px}}
  \sol {
    If equation \eqref{eqn:schur} is stable in $z_i$ coordinates, then for any bounded input $\vec{\widetilde{{u}}}(t),$
    $\vec{z}(t)$ must be bounded as well.
    To show that this system is stable in $x_i$ coordinates, as well, we must show for any bounded input $\vec{u}(t),$ the output $\vec{x}$ must be bounded as well. \vskip 1pt
    We can take any bounded input $u(t)$ and convert it to $\vec{\widetilde{u}}(t)$ by using the change of coordinates matrix from $x_i$ to $z_i$ coordinates, $V^{-1}.$
    \begin{align*}
      \ddt{}{t} \vec{x}(t) &= A \vec{x}(t) + B \vec{u}(t) \\
      V \ddt{}{t} \vec{z}(t) &= A V\vec{z}(t) + B \vec{u}(t) \\
      \ddt{}{t} \vec{z}(t) &= V^{-1}AV \vec{z}(t) + V^{-1} B \vec{u}(t) \\
      \ddt{}{t} \vec{z}(t) &= V^{-1}AV \vec{z}(t) + \widetilde{B} \vec{u}(t) \\
      \ddt{}{t} \vec{z}(t) &= V^{-1}AV \vec{z}(t) + \vec{\widetilde{{u}}}(t)
    \end{align*}
    We know that the system represented by equation \eqref{eqn:schur} is stable in $z_i$ coordinates, so $\vec{z}(t)$ is bounded as well.
    We conclude by converting $\vec{z}(t)$ back into standard coordinates through the matrix $V:$
    \begin{equation}
      \vec{x}(t) = V \vec{z}(t)
    \end{equation}
    Since $\vec{z}(t)$ is bounded, $\vec{x}(t)$ must be bounded as well.
  }

  \qitem \textbf{Uncouple the system in equation \eqref{eqn:input} into first order scalar differential equations.}

  \ws{\vspace{200px}}
  \sol {
    Since the $R$ matrix is upper triangular, we can uncouple the system into $n$ first order scalar differential equations.
    The $r_{i,j}$ represents the entry in the $i^{th}$ row and $j^{th}$ column. 
    \begin{align*}
      \ddt{}{t} \vec{z}(t) = \ddt{}{t} \begin{bmatrix} z_{1}(t) \\ \vdots \\ z_{n}(t) \end{bmatrix} &= R \vec{z}(t) + \vec{\widetilde{{u}}}(t) = 
      \begin{bmatrix} 
      \lambda_{1} & r_{1,2} & \cdots & r_{1,n} \\ 
      0 & \lambda_{2} & \cdots & r_{2,n} \\ 
      \vdots & \ddots & \cdots & \vdots \\
      0 & \cdots & 0 & \lambda_{n}
      \end{bmatrix} 
      \begin{bmatrix} z_{1}(t) \\ \vdots \\ z_{n}(t) \end{bmatrix} + \begin{bmatrix} \widetilde{u}_{1}(t) \\ \vdots \\ \widetilde{u}_{n}(t) \end{bmatrix}
    \end{align*}
    Individually, the equations will be:
    \begin{align*}
    \ddt{}{t} z_{1}(t) &= \lambda_{1} z_{1}(t) + r_{1,2} z_{2}(t) + \cdots + r_{1,n} z_{n}(t) + \widetilde{u}_{1}(t) \\
    \ddt{}{t} z_{2}(t) &= \lambda_{2} z_{2}(t) + r_{2,3} z_{3}(t) + \cdots + r_{2,n} z_{n}(t) + \widetilde{u}_{2}(t)\\
    \vdots \ \  \ \ \ &= \ \ \ \ \ \vdots \\
    \ddt{}{t} z_{n}(t) &= \lambda_{n} z_{n}(t) + \widetilde{u}_{n}(t)
    \end{align*}
  }

  \qitem \textbf{Under what conditions will $\ddt{}{t} z_{n}(t) = \lambda_{n} z_{n}(t) + \widetilde{u}_{n}(t)$ be stable?}
  
  \textit{Hint: Recall our definition of BIBO stability, which says a system is stable $\iff x(t) \leq B < \infty$ for all choices of bounded inputs $(t)$. What conditions does this imply on the eigenvalues?}

  \ws{\vspace{150px}}
  \meta {
    This is an if and only if condition since we are discussing scalar BIBO stability.
  }

  \sol {
    This scalar differential equation will only be stable when $\mathfrak{Re}(\lambda_{n}) < 0$ by the criterion we defined for BIBO stability in the previous question.
  }

  \qitem \textbf{Using back-substitution, confirm that $\ddt{}{t} \vec{z}(t) = R \vec{z}(t) + \vec{\widetilde{u}}(t)$ will be stable when $\operatorname{Re}(\lambda_i) < 0$ for all $i \in {0, 1, \dots, n}$}.

  \meta {
    This however, is not a necessary condition, and will only be sufficient. 
    It is possible to have a system with eigenvalue $0$ and still have it be BIBO stable, but we will say that it is not system stable, or asymptotically stable.
    An example of this is when $A = \begin{bmatrix} 0 & 0 \\ 0 & -1 \end{bmatrix}, B = \begin{bmatrix} 0 \\ 1 \end{bmatrix}.$

    Therefore, we say the system is guaranteed to be BIBO stable when $\operatorname{Re}(\lambda_i) < 0.$ 
    When discussing system stability however, we must be able to control all of our eigenvalues in order to say that the eigenvalue test works for BIBO stability. In this case, the system is uncontrollable, and we cannot control the response with eigenvalue $0.$

    Note that this question has been worded very carefully to claim that a system will be BIBO stable when the eigenvalue check is satisfied. Also, in general, when a question asks, "Is the system stable?" it is usually referring to system stability through the eigenvalue test.
  }

  \sol {
    The differential equation for $z_{n-1}(t)$ will be:
    \begin{equation}
      \ddt{}{t} z_{n-1}(t) = \lambda_{n-1} z_{n-1}(t) + r_{n-1, n} \ z_{n}(t) + \widetilde{u}_{n-1}(t)
    \end{equation}
    We've shown that $z_{n}(t)$ is stable only when $\mathfrak{Re}(\lambda_{n}) < 0,$ so we will now only have to consider the stability of $z_{n-1}(t).$ \vskip 1pt
    We can treat $w_{n-1}(t) = r_{n-1, n} \ z_{n}(t) + \widetilde{u}_{n-1}(t)$ as an input to the differential equation for $z_{n-1},$ and again reapply the definition of stability to claim that $z_{n-1}$ is stable when $\mathfrak{Re}(\lambda_{n-1}) < 0.$ \vskip 1pt
    Continuing this back-substitution, we will eventually see that when $\mathfrak{Re}(\lambda_{i}) < 0$ for all $i,$ all of the $z_i$ will be stable.
  }

\end{enumerate}