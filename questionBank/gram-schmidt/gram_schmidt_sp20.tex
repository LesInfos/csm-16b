% Author: Maxwell Chen
\pgfplotsset{width=7cm,compat=1.16}
\usepgfplotslibrary{polar}

% % eg, ie, etc, wrt, viz newcommands
\newcommand{\eg}{{\it e.g.\@}} % \@ for normal spacing after. use: \eg\/ or \eg,
\newcommand{\ie}{{\it i.e.\@}} 
\newcommand{\wrt}{{\it wrt}}
\newcommand{\viz}{{\it viz.\@}}
\newcommand{\vs}{{\it vs.\@}}
\newcommand{\aka}{{\it aka}}
\newcommand{\apriori}{{\it a priori}}
\newcommand{\etc}{\textit{etc.\@}}    
\newcommand{\bank}{../../questionBank}                                            
% etc defined so that there is only a single period after it if it is followed 
% by a period
% see: http://en.wikibooks.org/wiki/TeX/def                                      
% see: http://www.tug.org/pipermail/texhax/2004-July/002449.html                 
% see: http://en.wikibooks.org/wiki/TeX/if                                       
%\ifx \etc \undefined                                                             
%  %\def\myarg{#1}                                                                
%  %\def\myfullstop{.}                                                            
%  %\def\etc{#1}{\textit{etc} arg is ``\myarg'' \ifx\myarg\myfullstop\then.\else.#1\fi\/}
%  \def\etc#1{\textit{etc}\if#1..\else.#1\fi}                                     
%\fi                                                                              
%                        

% this should be moved to a new newcommands file
\newcommand{\ignore}[1]{}


\qns{Gram-Schmidt Process}

\meta{
Ask students why gram-schmidt might be useful?
Examples include construction of SVD vectors, QR factorization (not taught), upper triangularization, etc.
}

A set of vectors is orthonormal \textbf{if and only if}:
\begin{itemize}
    \item Any pair of vectors are orthogonal (for any vectors $\vec{u}, \vec{v} \in S$ where $\vec{u} \neq \vec{v}$, the angle between them is 90$^\circ$).
    \item Each vector is of unit length (for vector $\vec{v} \in $ $S$, $\norm{\vec{v}} = 1$).
\end{itemize}

Gram-Schmidt orthonormalization is a process that allows us to take such a set $S$ of arbitrary, linearly-independent vectors, and create a new set $S'$ of vectors that span the exact same space, yet are now orthonormal.

\begin{enumerate}

\qitem For linearly independent vectors $\vec{v}_1, \vec{v}_2$, \textbf{justify that Span($\vec{v}_1, \vec{v}_2$) is the same as Span($\vec{v}_1, \vec{v}_2 - \alpha\vec{v}_1$).}

\ws{\vspace{75px}}
\sol {
The definition of span is all vectors that can be "reached" using a linear combination of a set of vectors. So,
$$span(\vec{v_1}, \vec{v_2}) = \vec{v}_{i}=\beta_{1} \vec{v}_{1}+\beta_{2} \vec{v}_{2} \text { for arbitrary } \vec{v}_{i} \text { and all real } \beta_{1}, \beta_{2}$$
Following a similar train of thought,
$$\begin{aligned} \operatorname{span}\left(\vec{v}_{1}, \vec{v}_{2}-\alpha \vec{v}_{1}\right) &=\beta_{3} \vec{v}_{1}+\beta_{4}\left(\vec{v}_{2}-\alpha \vec{v}_{1}\right) \\ &=\beta_{3} \vec{v}_{1}+\beta_{4} \vec{v}_{2}-\beta_{4} \alpha \vec{v}_{1} \\ &=\left(\beta_{3}-\beta_{4} \alpha\right) \vec{v}_{1}+\beta_{4} \vec{v}_{2} \end{aligned}$$
All $\alpha$ and $\beta$ are arbitrary constants that we can set. If we set $\beta_4 = \beta_2$ and $\beta_3 = \beta_1 + \alpha\beta_2$, then these spans are the same!
}

\qitem Let us iteratively generate the orthonormal set $S'$ from $S$. Start by arbitrarily picking a vector $\vec{v}_1$ from $S$: \textbf{how can we create a new vector $\vec{u}_1$ for $S'$ so that it spans the same space as the original vector, yet ensures that orthonormality is preserved for our new set?}

\ws{\vspace{75px}}

\sol {
 Since we are only taking into consideration a single vector, we do not need to worry about orthogonality. So,
$$\vec{u}_1 = \frac{\vec{v}_1}{\norm{\vec{v}_2}}$$
}

\qitem Now, consider the next vector $\vec{v}_2$ in $S$. \textbf{How can we convert $\vec{v}_2$ to some $\vec{u}_2$ that is orthonormal with respect to the $\vec{u}_1$ that we've already added to $S'$?}

\ws{\vspace{75px}}

\sol {
Now, we do need to consider orthogonality. In order for two vectors $\vec{u}, \vec{v}$ to be orthogonal, their dot product has to be 0. In other words, this means that there is nothing--no trace, no component--of $\vec{u} \text { in } \vec{v}$, and vice versa. We can do this by defining the following where:
$$\vec{p_2} = \vec{v_2} - (\vec{v_2}^T\vec{u_1})\vec{u_1}$$
By subtracting the projection of $\vec{v_2}$ onto $\vec{u_1}$, we are removing the $\vec{u_1}$ component of $\vec{v_2}$ from $\vec{v_2}$. So, $\vec{p_2}$ is now orthogonal to $\vec{u_1}$, and mutual orthogonality is preserved for all vectors in our new set $S'$. $\vec{p_2}$ also needs to be normalized, so:
$$\vec{w_2} = \frac{\vec{p_2}}{|\vec{p_2}|}$$
}

\qitem \textbf{Write out the process for creating an arbitrary vector $\vec{u}_i$ for our orthonormal set $S'$.}

\ws{\vspace{75px}}

\sol {
Define the following vector:
$$\vec{p_i} = \vec{v_i} - \sum_{j=1}^{i-1}(\vec{v_i}^T\vec{w_j})\vec{w_j}$$
By subtracting the component of every other vector already in $S'$, this ensures that our new vector $\vec{w_i}$ is orthogonal. Then, we normalize this:
 $$\vec{w_i} = \frac{\vec{p_i}}{|\vec{p_i}|}$$
Rinse and repeat for all vectors in $S$ to generate a new, orthonormal set $S'$ that we can use in place of $S$.
}

\qitem \textbf{OPTIONAL:}
Let $A = \begin{bmatrix}
-1 & -1 & 1 \\
1 & 3 & 3 \\
-1 & -1 & 5 \\
1 & 3 & 7 \end{bmatrix}$.

We will use Gram-Schmidt to find an orthonormal basis $U = \begin{bmatrix} \vec{u}_{1} & \vec{u}_{2} & \vec{u}_{3} \end{bmatrix}$ for the $\text{Col}(A).$

\begin{enumerate}
  \qitem \textbf{Find $\vec{u}_{1}$.}

  \meta {
    It really is up to you, how you teach Gram-Schmidt, but I prefer to make all of the vectors orthogonal first, and then normalize at the end, once I've found an orthogonal set of vectors $\{ \vec{q}_{1}, \cdots \vec{q}_{n} \}.$
  }

  \ws{\vspace{75px}}
  \sol {
    For the first vector, we do not need to worry about orthogonality. Therefore we just have to normalize.
    $$\vec{u}_{1} = \frac{\vec{a}_{1}}{\norm{\vec{a}_{1}}} = \frac{1}{\sqrt{4}} \begin{bmatrix} -1 \\ 1 \\ -1 \\ 1 \end{bmatrix} =
    \begin{bmatrix} -\frac{1}{2} \\ \frac{1}{2} \\ -\frac{1}{2} \\ \frac{1}{2} \end{bmatrix}$$
  }

  \qitem \textbf{Find $\vec{u}_{2}$.}

  \ws{\vspace{75px}}
  \sol {
    We first compute $\vec{q}_{2}$ as a vector orthogonal to $\vec{u}_{1}$ by subtracting the projection of $\vec{a}_{2}$ onto $\vec{u}_{1}$
    $$\vec{q}_{2} = \vec{a}_{2} - \innp{\vec{a}_{2}}{\vec{u}_{1}} \vec{u}_{1} = \vec{a}_{2} - \frac{1}{2} \innp{\vec{a}_{2}}{\vec{a}_{1}} \frac{1}{2} \vec{a}_{1}
    = \begin{bmatrix} -1 \\ 3 \\ -1 \\ 3 \end{bmatrix} -
    \frac{8}{4} \begin{bmatrix} -1 \\ 1 \\ -1 \\ 1 \end{bmatrix} = \begin{bmatrix} 1 \\ 1 \\ 1 \\ 1 \end{bmatrix}$$
    If we were to normalize this vector, we would get:
    $$\vec{u}_{2} = \frac{1}{\sqrt{4}} \begin{bmatrix} 1 \\ 1 \\ 1 \\ 1 \end{bmatrix} =
    \begin{bmatrix} \frac{1}{2} \\ \frac{1}{2} \\ \frac{1}{2} \\ \frac{1}{2} \end{bmatrix}$$
  }

  \qitem \textbf{Find $\vec{u}_{3}$.}

  \ws{\vspace{100px}}
  \sol {
    We again compute $\vec{q}_{3}$ as a vector orthogonal to rest by subtracting its projections onto $\vec{a}_{1}$ and $\vec{a}_{2}.$
    \begin{align*}
    \vec{q}_{3} &= \vec{a}_{3} - \innp{\vec{a}_{3}}{\vec{u}_{1}} \vec{u}_{1} - \innp{\vec{a}_{3}}{\vec{u}_{2}} \vec{u}_{2} =
    \vec{a}_{3} - \frac{1}{4} \innp{\vec{a}_{3}}{\vec{a}_{1}} \vec{a}_{1} - \frac{1}{4} \innp{\vec{a}_{3}}{\vec{q}_{2}} \vec{q}_{2} \\
    &= \begin{bmatrix} 1 \\ 3 \\ 5 \\ 7 \end{bmatrix} - \frac{4}{4} \begin{bmatrix} -1 \\ 1 \\ -1 \\ 1 \end{bmatrix} -
    \frac{16}{4} \begin{bmatrix} 1 \\ 1 \\ 1 \\ 1 \end{bmatrix} = \begin{bmatrix} 1 \\ 3 \\ 5 \\ 7 \end{bmatrix} - \begin{bmatrix} -1 \\ 1 \\ -1 \\ 1 \end{bmatrix} - \begin{bmatrix} 4 \\ 4 \\ 4 \\ 4 \end{bmatrix} = \begin{bmatrix} -2 \\ -2 \\ 2 \\ 2 \end{bmatrix}
    \end{align*}
    Normalizing $\vec{q}_{3},$ we get:
    $$\vec{u}_{3} = \frac{1}{\sqrt{16}} \begin{bmatrix} -2 \\ -2 \\ 2 \\ 2 \end{bmatrix} = \begin{bmatrix} -\frac{1}{2} \\ -\frac{1}{2} \\ \frac{1}{2} \\ \frac{1}{2} \end{bmatrix}$$
  }

\end{enumerate}

\end{enumerate}
