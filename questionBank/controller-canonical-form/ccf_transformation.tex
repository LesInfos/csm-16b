\qns{Controllable Canonical Form (20 pts)}
\qcontributor{Yen-Sheng Ho}

When we are trying to stabilize a robot, it is sometimes useful to put
the dynamics into a standard form that lets us more easily adjust its
behavior. 

Consider the linear continuous-time system below.
\begin{align*}
\frac{d}{dt}\vec{s}(t) 
= A_r\vec{s}(t) + \vec{b}_r {u}(t)
=\begin{bmatrix}
 2 & -1 \\
 1 & 0 
\end{bmatrix} 
\vec{s}(t) +
\begin{bmatrix}
1 \\ 0 
\end{bmatrix}
u(t)
\end{align*} 

\begin{enumerate}

\qitem (10 pts)
There exists a transformation $\vec{z}(t) = T\vec{s}(t)$ such that the
resulting system is in controllable canonical form: 
\begin{align*}
\frac{d}{dt}\vec{z}(t) 
= \begin{bmatrix}
 0 & 1 \\
 \alpha_0 & \alpha_1 \\
\end{bmatrix} 
\vec{z}(t) +
\begin{bmatrix}
0 \\ 1 
\end{bmatrix}
u(t)
\end{align*}

{\bf Find this transformation matrix $T$ and the resulting $\alpha_0$
  and $\alpha_1$ in controllable canonical form.}

{\em HINT:
The column vectors of $T$ are just the standard basis vectors
$\left\{
\begin{bmatrix}
1 \\ 0 
\end{bmatrix},
\begin{bmatrix}
0 \\ 1 
\end{bmatrix}
\right\}
$
arranged in some order.
}

\sol{
From the hint and we know $T$ cannot be the identity matrix, so
\begin{align*}
T = \begin{bmatrix}
0 & 1\\
1 & 0
\end{bmatrix}
\end{align*}

With $T$, we can derive the canonical form matrix $A_c$ by
\begin{align*}
A_c = TA_rT^{-1} = \begin{bmatrix}
0 & 1\\
1 & 0
\end{bmatrix}
\begin{bmatrix}
 2 & -1 \\
 1 & 0 
\end{bmatrix} 
\begin{bmatrix}
0 & 1\\
1 & 0
\end{bmatrix}
= 
\begin{bmatrix}
 0 & 1 \\
-1 & 2 
\end{bmatrix} 
\end{align*}

Thus, $\alpha_0 = -1$ and $\alpha_1 = 2$.
}
% \vspace{1.5in}
\startnewpage

\qitem (10 pts) 
For the system given by ($A_r$, $\vec{b}_r$, $\vec{s}$, $u$), {\bf use the
controllable canonical form from the
previous part} to obtain a state feedback law $u(t) = \vec{f}^T
\vec{z}(t) = \begin{bmatrix} 
f_0 & f_1
\end{bmatrix} \vec{z}(t)$ such that the resulting closed-loop system
has eigenvalues $\lambda = -1, -2$ and then use the transformation $T$
to get $u(t) = \vec{g}^T \vec{s}(t) =  \begin{bmatrix} 
g_0 & g_1
\end{bmatrix} \vec{s}(t)$ a control law in terms of the original state
variable $\vec{s}(t)$. 

{\bf What are the vectors $\vec{f}$ and $\vec{g}$?}

{\em You will get full credit if you correctly use the properties of
  controllable canonical form to do this, but you may check your
  answer by another method if you so desire.}

\sol{ 

First we find a feedback law for $\vec{f}$ and $\vec{z}$.
With the feedback, we have
\begin{align*}
\frac{d}{dt}\vec{z}(t) = \left( A_c + \begin{bmatrix}
0 & 0 \\
f_0 & f_1
\end{bmatrix} 
\right)
\vec{z}(t)
= \begin{bmatrix}
0 & 1 \\
-1 + f_0 & 2 + f_1
\end{bmatrix}
\vec{z}(t)
\end{align*}
The characteristic polynomial of the closed-loop system is given by
\begin{align*}
\lambda^2 - (2+f_1)\lambda -(-1+f_0)
\end{align*}
Our goal is to place eigenvalues at $-1$ and $-2$, which is characterized by the polynomial
\begin{align*}
(\lambda +1)(\lambda+2) = \lambda^2 + 3\lambda + 2
\end{align*}
By comparison of coefficients, we have
\begin{align*}
-2 - f_1 = 3 \\
1 - f_0 = 2
\end{align*}
Thus,
\begin{align*}
\vec{f} = \begin{bmatrix}
-1 \\ -5
\end{bmatrix}
\end{align*}

Then, using the property of the controllable canonical form, we have
\begin{align*}
\vec{g}^T = \vec{f}^T T =
\begin{bmatrix}
-1 & -5
\end{bmatrix}
\begin{bmatrix}
0 & 1\\
1 & 0
\end{bmatrix}=
\begin{bmatrix}
-5 & -1
\end{bmatrix}
\end{align*}
}

% \startnewpage

% \qitem (6 pts) Consider another system below:
% \begin{align*}
% \frac{d}{dt}\vec{x}(t) 
% = A \vec{x}(t) + \vec{b}u(t)
% =\begin{bmatrix}
%  1 & -2 \\
%  0 & 1  \\
% \end{bmatrix} 
% \vec{x}(t) +
% \begin{bmatrix}
% \frac{1}{2} \\ -\frac{1}{2} 
% \end{bmatrix}
% u(t)
% \end{align*} 

% Compute a state feedback law $u(t) = \vec{k}^T \vec{x}(t) = \begin{bmatrix}
% k_0 & k_1 
% \end{bmatrix} \vec{x}(t)$ such that the resulting closed-loop system has eigenvalues $\lambda = -1, -2$.
% {\bf What is the vector $\vec{k}$?}

% {\em HINT: the matrices $A$ and $A_r$ are related by
% \begin{align*}
% \underbrace{\begin{bmatrix}
%  2 & -1 \\
%  1 & 0  
% \end{bmatrix} }_{A_r}
% = \begin{bmatrix}
% 1 & -1 \\
% 1 &  1 
% \end{bmatrix}
% \underbrace{
% \begin{bmatrix}
%  1 & -2  \\
%  0 & 1 \\
% \end{bmatrix}
% }_{A}
% \begin{bmatrix}
% \frac{1}{2} & \frac{1}{2} \\
% -\frac{1}{2} &  \frac{1}{2}
% \end{bmatrix}
% \end{align*}
% }

% \sol{ FIX THIS
% \begin{align*}
% \vec{k}^T = \vec{f}^T T = \begin{bmatrix}
% -14 & 0 & -16 
% \end{bmatrix}
% \begin{bmatrix}
% 1 & -1 & 0\\
% 1 & 1 & 0\\
% 0 & 0 & 1\\
% \end{bmatrix}
% =\begin{bmatrix}
% -14 & 14 & -16
% \end{bmatrix}
% \end{align*}
% }

%\qitem Consider the specific system below.
%\begin{align*}
%\frac{d}{dt}\vec{x}(t) 
%= \begin{bmatrix}
% 7 & 8 & -14 \\
% 0 & 0 & 1 \\
% 1 & 0 & 0\\
%\end{bmatrix} 
%\vec{x}(t) +
%\begin{bmatrix}
%1 \\ 0 \\ 0 
%\end{bmatrix}
%u(t)
%\end{align*} 
%Given the fact that there exists a transformation $\vec{s}(t) = \begin{bmatrix}
%1 & 0 & 0\\
%0 & 0 & 1\\
%0 & 1 & 0\\
%\end{bmatrix}\vec{x}(t)$ such that 
%\begin{align*}
%\frac{d}{dt}\vec{s}(t) 
%= A_r\vec{s}(t) + \vec{b}_r {u}(t)
%\end{align*} 
%Compute a state feedback $u(t) = \vec{k}^T \vec{x}(t) = \begin{bmatrix}
%k_0 & k_1 & k_2
%\end{bmatrix} \vec{x}(t)$ such that the resulting closed-loop system has eigenvalues $\lambda = -1, -2, -4$.
%What is the vector $\vec{k}$?
%
%\sol{
%\begin{align*}
%\vec{k}^T = \vec{f}^T T = \begin{bmatrix}
%-14 & 0 & -16 
%\end{bmatrix}
%\begin{bmatrix}
%1 & 0 & 0\\
%0 & 0 & 1\\
%0 & 1 & 0\\
%\end{bmatrix}
%=\begin{bmatrix}
%-14 & -16 & 0
%\end{bmatrix}
%\end{align*}
%}


\end{enumerate}


% \qitem What is the characteristic polynomial, $\mbox{det}(\lambda I - A_r)$, of the matrix $A_r$?

% \sol{
% \begin{align*}
% \lambda^n - \sum_{i = 0}^{n-1} a_i \lambda^{n-1-i} 
% \end{align*}
% }

%\qitem 
%Is the system (characterized by $A_r$ and $\vec{b}_r$) controllable? (You do not need to justify your answer.)
%
%\sol{
%
%\begin{align*}
%C = \begin{bmatrix}
%1 & a_0 & a_0^2 + a_1\\
%0 & 1   & a_0 \\
%0 & 0   & 1 \\
%\end{bmatrix}
%\end{align*}
%}

