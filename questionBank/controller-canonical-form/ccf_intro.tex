\qns{Controllable Canonical Form}
\qcontributor{Yen-Sheng Ho}

Consider the linear continuous-time system below.
\begin{align*}
\frac{d}{dt}\vec{s}(t) 
= A_r\vec{s}(t) + \vec{b}_r {u}(t)
=\begin{bmatrix}
 7 & -14 & 8 \\
 1 & 0 & 0 \\
 0 & 1 & 0\\
\end{bmatrix} 
\vec{s}(t) +
\begin{bmatrix}
1 \\ 0 \\ 0 
\end{bmatrix}
u(t)
\end{align*} 

\begin{enumerate}

\qitem
There exists a transformation $\vec{z}(t) = T\vec{s}(t)$ such that the
resulting system is in controllable canonical form: 
\begin{align*}
\frac{d}{dt}\vec{z}(t) 
= \begin{bmatrix}
 0 & 1 & 0 \\
 0 & 0 & 1 \\
 \alpha_0 & \alpha_1 & \alpha_2 \\
\end{bmatrix} 
\vec{z}(t) +
\begin{bmatrix}
0 \\ 0 \\ 1 
\end{bmatrix}
u(t)
\end{align*}

{\bf Find this transformation matrix $T$.}

{\em HINT:
The column vectors of $T$ are just the standard basis vectors
$\left\{
\begin{bmatrix}
1 \\ 0 \\ 0
\end{bmatrix},
\begin{bmatrix}
0 \\ 1 \\ 0
\end{bmatrix},
\begin{bmatrix}
0 \\ 0 \\ 1
\end{bmatrix}
\right\}
$
arranged in some order.
}

\sol{
\begin{align*}
T = \begin{bmatrix}
0 & 0 & 1\\
0 & 1 & 0\\
1 & 0 & 0\\
\end{bmatrix}
\end{align*}
}

\vspace{1.5in}


\qitem 
For the system given by ($A_r$, $\vec{b}_r$, $\vec{s}$, $u$), compute
a state feedback law $u(t) = \vec{f}^T \vec{s}(t) = \begin{bmatrix}
f_0 & f_1 & f_2
\end{bmatrix} \vec{s}(t)$ such that the resulting closed-loop system has eigenvalues $\lambda = -1, -2, -4$.

{\bf What is the vector $\vec{f}$?}

\sol{
\begin{align*}
\vec{f} = \begin{bmatrix}
-14 \\ 0 \\ -16
\end{bmatrix}
\end{align*}
}

\startnewpage

\qitem Consider another system below:
\begin{align*}
\frac{d}{dt}\vec{x}(t) 
= A \vec{x}(t) + \vec{b}u(t)
=\begin{bmatrix}
 -3 & -11 & 4 \\
 4 & 10 & -4 \\
 1 & 1 & 0\\
\end{bmatrix} 
\vec{x}(t) +
\begin{bmatrix}
\frac{1}{2} \\ -\frac{1}{2} \\ 0 
\end{bmatrix}
u(t)
\end{align*} 

Compute a state feedback law $u(t) = \vec{k}^T \vec{x}(t) = \begin{bmatrix}
k_0 & k_1 & k_2
\end{bmatrix} \vec{x}(t)$ such that the resulting closed-loop system has eigenvalues $\lambda = -1, -2, -4$.
{\bf What is the vector $\vec{k}$?}

{\em HINT: the matrices $A$ and $A_r$ are related by
\begin{align*}
\underbrace{\begin{bmatrix}
 7 & -14 & 8 \\
 1 & 0 & 0 \\
 0 & 1 & 0\\
\end{bmatrix} }_{A_r}
= \begin{bmatrix}
1 & -1 & 0\\
1 &  1 & 0\\
0 &  0 & 1\\
\end{bmatrix}
\underbrace{
\begin{bmatrix}
 -3 & -11 & 4 \\
 4 & 10 & -4 \\
 1 & 1 & 0\\
\end{bmatrix}
}_{A}
\begin{bmatrix}
\frac{1}{2} & \frac{1}{2} & 0\\
-\frac{1}{2} &  \frac{1}{2} & 0\\
0 &  0 & 1\\
\end{bmatrix}
\end{align*}
}

\sol{
\begin{align*}
\vec{k}^T = \vec{f}^T T = \begin{bmatrix}
-14 & 0 & -16 
\end{bmatrix}
\begin{bmatrix}
1 & -1 & 0\\
1 & 1 & 0\\
0 & 0 & 1\\
\end{bmatrix}
=\begin{bmatrix}
-14 & 14 & -16
\end{bmatrix}
\end{align*}
}

%\qitem Consider the specific system below.
%\begin{align*}
%\frac{d}{dt}\vec{x}(t) 
%= \begin{bmatrix}
% 7 & 8 & -14 \\
% 0 & 0 & 1 \\
% 1 & 0 & 0\\
%\end{bmatrix} 
%\vec{x}(t) +
%\begin{bmatrix}
%1 \\ 0 \\ 0 
%\end{bmatrix}
%u(t)
%\end{align*} 
%Given the fact that there exists a transformation $\vec{s}(t) = \begin{bmatrix}
%1 & 0 & 0\\
%0 & 0 & 1\\
%0 & 1 & 0\\
%\end{bmatrix}\vec{x}(t)$ such that 
%\begin{align*}
%\frac{d}{dt}\vec{s}(t) 
%= A_r\vec{s}(t) + \vec{b}_r {u}(t)
%\end{align*} 
%Compute a state feedback $u(t) = \vec{k}^T \vec{x}(t) = \begin{bmatrix}
%k_0 & k_1 & k_2
%\end{bmatrix} \vec{x}(t)$ such that the resulting closed-loop system has eigenvalues $\lambda = -1, -2, -4$.
%What is the vector $\vec{k}$?
%
%\sol{
%\begin{align*}
%\vec{k}^T = \vec{f}^T T = \begin{bmatrix}
%-14 & 0 & -16 
%\end{bmatrix}
%\begin{bmatrix}
%1 & 0 & 0\\
%0 & 0 & 1\\
%0 & 1 & 0\\
%\end{bmatrix}
%=\begin{bmatrix}
%-14 & -16 & 0
%\end{bmatrix}
%\end{align*}
%}


\end{enumerate}


% \qitem What is the characteristic polynomial, $\mbox{det}(\lambda I - A_r)$, of the matrix $A_r$?

% \sol{
% \begin{align*}
% \lambda^n - \sum_{i = 0}^{n-1} a_i \lambda^{n-1-i} 
% \end{align*}
% }

%\qitem 
%Is the system (characterized by $A_r$ and $\vec{b}_r$) controllable? (You do not need to justify your answer.)
%
%\sol{
%
%\begin{align*}
%C = \begin{bmatrix}
%1 & a_0 & a_0^2 + a_1\\
%0 & 1   & a_0 \\
%0 & 0   & 1 \\
%\end{bmatrix}
%\end{align*}
%}

