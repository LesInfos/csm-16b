\qns{Eigenvalue Placement Through Controller Canonical Form}
% \qcontributor{Yen-Sheng Ho}

Consider a linear discrete time system below ($\vec{x} \in
\mathbb{R}^n$, $u \in \mathbb{R}$, and $\vec{b} \in \mathbb{R}^n$).
\begin{align*}
\vec{x}(t+1) = A \vec{x}(t) + \vec{b} u(t)
\end{align*}

If the system is \textit{controllable},
then there exists a transformation $\vec{z} = T\vec{x}$ (where $T$ is an invertible $n\times n$ matrix)
such that in the transformed coordinates, the system is in
\textit{controllable canonical form}, which is given by
\begin{align*}
\vec{z}(t+1) = \widetilde{A}\vec{z}(t) + \widetilde{\vec{b}} u(t) = \begin{bmatrix}
0 & 1 & 0 & \cdots & 0 \\
0 & 0 & 1 & \cdots & 0 \\
\vdots & \vdots & \vdots  & \ddots & \vdots \\
 0 & 0 & 0 & \cdots & 1 \\
 -a_0 & -a_{1} & -a_{2} & \ldots & -a_{n-1} \\
\end{bmatrix} 
\vec{z}(t) +
\begin{bmatrix}
0 \\ 0 \\ 0 \\ \vdots \\ 1
\end{bmatrix}
u(t)
\end{align*}

The characteristic polynomials of the matrices $A$ and $\widetilde{A}$ are the same and given by
\begin{align}\label{eq:poly}
\mbox{det}(\lambda I - A ) = \mbox{det}(\lambda I - \widetilde{A} )
= \lambda^n  + a_{n-1} \lambda^{n-1} + a_{n-2} \lambda^{n-2} + \ldots + a_0
\end{align}

\begin{enumerate}
\qitem Show that $\widetilde{A} = TAT^{-1}$ and $\widetilde{\vec{b}} = T\vec{b}$.

\sol{

Starting from the original system, we have 
\begin{align*}
\vec{x}(t+1) &= A \vec{x}(t) + \vec{b} u(t) \\
\Rightarrow T\vec{x}(t+1) &= TA \vec{x}(t) + T\vec{b} u(t) \\
\Rightarrow T\vec{x}(t+1) &= TA (T^{-1}T) \vec{x}(t) + T\vec{b} u(t) \\
\Rightarrow \vec{z}(t+1) &= TAT^{-1}\vec{z}(t) + T\vec{b} u(t) 
\end{align*}

Comparing to the canonical form, we have
\begin{align*}
\widetilde{A} = TAT^{-1} \mbox{, and }  \widetilde{\vec{b}} = T\vec{b}
\end{align*}

}
\qitem Show that $A$ and $\widetilde{A}$ have the same eigenvalues (\textit{Hint: let $\vec{v}$ be an eigenvector of $A$; use $T\vec{v}$ for $\widetilde{A}$})

\sol{

Let $\lambda$ and $\vec{v}$ be an eigenvalue and its corresponding eigenvector of $A$. That is, $A\vec{v} = \lambda \vec{v}$.
Next we consider $\widetilde{A}T\vec{v}$:
\begin{align*}
\widetilde{A}(T\vec{v}) &=  TAT^{-1} T\vec{v} \\
&= TA\vec{v}\\
&= T(\lambda\vec{v}) \\
&= \lambda(T\vec{v})
\end{align*}
We have shown that $\lambda$ is also an eigenvalue of $\widetilde{A}$ and its corresponding eigenvector is $T\vec{v}$.
}
\qitem Let the controllability matrices $C$ and $\widetilde{C}$ be $C = \begin{bmatrix}
\vec{b} & A\vec{b} & \cdots & A^{n-1}\vec{b}
\end{bmatrix}$
and 
$\widetilde{C} = \begin{bmatrix}
\widetilde{\vec{b}} & \widetilde{A}\widetilde{\vec{b}} & \cdots & \widetilde{A}^{n-1}\widetilde{\vec{b}}
\end{bmatrix}$, respectively.
Show that $\widetilde{C} = TC$, which is equivalent to $T = \widetilde{C}C^{-1}$.

\sol{
\begin{align*}
\widetilde{C} &= \begin{bmatrix}
\widetilde{\vec{b}} & \widetilde{A}\widetilde{\vec{b}} & \cdots & \widetilde{A}^{n-1}\widetilde{\vec{b}}
\end{bmatrix} \\
&= \begin{bmatrix}
T{\vec{b}} & (T{A}T^{-1})(T{\vec{b}}) & \cdots & (T{A}^{n-1}T^{-1})(T{\vec{b}})
\end{bmatrix} \\
&= \begin{bmatrix}
T{\vec{b}} & T{A}{\vec{b}} & \cdots & T{A}^{n-1}{\vec{b}}
\end{bmatrix} \\
&= TC
\end{align*}

}

\end{enumerate}
