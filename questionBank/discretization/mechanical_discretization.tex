\qns{Mechanical Discretization}

You are building your 16B lab car, and have found the following relation between input voltage and velocity:

\begin{align*}
    \frac{d}{dt} v(t) = 5u(t)
\end{align*}
Your MSP can only provide a piecewise constant input voltage with interval $T$. This means that $u(t)$ is constant during each interval $[nT,(n+1)T)$.

\begin{enumerate}
    \qitem You want to analyze your velocity in discrete time as follows:
    \begin{align*}
        v_d(t + 1) = \alpha v_d(t) + u_d(t)
    \end{align*}
    where t is an integer and $v_d(t) = v(tT)$. \\
    \textit{Note: the $t$ in $v_d(t)$ is the current timestep in discrete time (which is always an integer) and differs from the $t$ in $v(t)$, which is in continuous time.}
    \begin{enumerate}[label=(\roman*)]
        \item Find $v(t + T)$ in terms of $v(t)$ and $u(t)$. \\
        \textit{Hint: Integrate both sides and apply the Fundamental Theorem of Calculus.}

        \sol{
            \begin{align*}
                \frac{d}{dt} v(t) = 5u(t) \\
                \int _{v(t)}^{v(t + T)} \, dv = \int_t^{t + T} 5u(\tau) \, d\tau \\
                v(t + T) - v(t) =  5(T + t)u(t) - 5tu(t) \\
                v(t + T) = v(t) + 5Tu(t)
            \end{align*}

            \textit{Note: notice how the variable of integration is $\tau$ (not $t$) on the left and $v$ on the right. We move the $dt$ from the left hand side
            to the right hand side using the same differential equations trick from module 1, so we are left with different bounds on eitherside to integrate.
            Also, as stated in the problem statement, we treat $u(t)$ as a constant.}
        }

        \item Using your result from part (i), find $\alpha$ and $u_d(t)$ in the relation
        \begin{align*}
            v_d(t + 1) = \alpha v_d(t) + u_d(t)
        \end{align*}
        \sol{
            Let us say that $v_d(t)$ is $v(t)$, and $v_d(t + 1)$ is $v(t + T)$.
            \begin{align*}
                v_d(t + 1) = v_d(t) + 5Tu(t) \\
            \end{align*}
            So, $\alpha = 1$ and $u_d(t) = 5Tu(t)$
        }
    \end{enumerate}

    %\vspace{15cm}

    \qitem Now you want to examine your car's position in discrete time, knowing that
    \begin{align*}
        \frac{d}{dt} x(t) = v(t)
    \end{align*}

    \begin{enumerate}[label=(\roman*)]
        \item Integrate both sides to get $x(t + T)$ in terms of $x(t)$, $v(t)$, and $u(t)$. \\
        \textit{Hint: If we know $v(t)$, then we can write $v(\tau)$ as $v(t) + v'(t) \cdot (\tau - t)$.}

        \sol{
            \begin{align*}
                \frac{d}{dt} x(t) = v(t) \\
                \int _{x(t)}^{x(t + T)} \, dx = \int_t^{t + T} v(\tau) \, d\tau \\
            \end{align*}
            Using the hint, we have
            \begin{align*}
                x(t + T) = x(t) + \int_t^{t + T} (v(t) + 5u(t)(\tau - t)) \, d\tau \\
                x(t + T) = x(t) + Tv(t) + 5u(t) \frac{T^2}{2}
            \end{align*}
        }

        \item Discretize the system by finding $\alpha$, $\beta$, and $u_d(t)$ in the following relation
        \begin{align*}
            x_d(t + 1) = \alpha x_d(t) + \beta v_d(t) + u_d(t)
        \end{align*}
        \sol{
            Again, we set $x_d(t) = x(t)$, so $x_d(t + 1) = x(t + T)$ and $v_d(t) = v(t)$
            \begin{align*}
                x_d(t + 1) = x_d(t) + Tv(t) + 5u(t)\frac{T^2}{2} \\
            \end{align*}
            So, $\alpha = 1$, $\beta = T$, and $u_d(t) = 5u(t)\frac{T^2}{2}$
        }
    \end{enumerate}

\end{enumerate}
