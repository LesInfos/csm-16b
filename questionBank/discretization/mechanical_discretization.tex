\qns{Mechanical Discretization}

You are building your 16B lab car, and have found the following relation between input voltage and velocity:

\begin{align*}
    \frac{d}{dt} \vec{v}(t) = 5u(t)
\end{align*}
Your MSP can only provide a piecewise constant input voltage with interval $T$. This means that, for each interval $[nT, (n + 1)T)$, $u(t)$ is constant.

\begin{enumerate}
    \qitem You want to analyze your velocity in discrete time as follows:
    \begin{align*}
        v_d(t + 1) = \alpha v_d(t) + u_d(t)
    \end{align*}
    where t is an integer and $v_d(t) = v(tT)$. \\
    \textit{Note: the $t$ in $v_d(t)$ is the current timestep in discrete time (which is always an integer) and differs from the $t$ in $v(t)$, which is in continuous time.}
    \begin{enumerate}[label=(\roman*)]
        \item Find $v(t + T)$ in terms of $v(t)$ and $u(t)$. Say that the input is constant between times $t$ and $t + T$. \\
        \textit{Hint: Integrate both sides and apply the Fundamental Theorem of Calculus.}

        \item Using your result from part (i), find $\alpha$ and $u_d(t)$ in the relation
        \begin{align*}
            v_d(t + 1) = \alpha v_d(t) + u_d(t)
        \end{align*}
    \end{enumerate}

    \qitem Now you want to examine your car's position in discrete time, knowing that
    \begin{align*}
        \frac{d}{dt} x(t) = v(t)
    \end{align*}

    \begin{enumerate}[label=(\roman*)]
        \item Integrate both sides to get $x(t + T)$ in terms of $x(t)$, $v(t)$, and $u(t)$. Again, assume a constant input between times $t$ and $t + T$. \\
        \textit{Hint: If we know $v(t)$, then we can write $v(\tau)$ as $v(t) + v'(t) \cdot (\tau - t)$.}

        \item Discretize the system by finding $\alpha$, $\beta$, and $u_d(t)$ in the following relation
        \begin{align*}
            x_d(t + 1) = \alpha x_d(t) + \beta v_d(t) + u_d(t)
        \end{align*}
    \end{enumerate}

\end{enumerate}