\qns{Controllability Practice}

Consider the system 
\[\vec{x}(t + 1) = 
\begin{bmatrix}
    0 & 1 & -2 \\
    0 & 2 & 0 \\
    -1 & 1 & 0
\end{bmatrix} \vec{x}(t) + 
\begin{bmatrix}
    2 \\ 0 \\ 0
\end{bmatrix} u(t)\]

\begin{enumerate}
    \qitem \textbf{What is the controllability matrix, $\mathcal{C}$, for this system?}
    \ws{\vspace{100px}}
    \sol{
        For a 3-dimensional system,
        \[\mathcal{C} = 
        \begin{bmatrix}
            \vec{b} & A\vec{b} & A^2 \vec{b}
        \end{bmatrix}\]
        Plugging in this system's $A$ and $\vec{b}$, we have
        \[\mathcal{C} = 
        \begin{bmatrix}
            2 & 0 & 4 \\
            0 & 0 & 0 \\
            0 & -2 & 0
        \end{bmatrix}\]
        The column space of $\mathcal{C}$ is the span of its column vectors. 
        The third column is linearily dependent, so $\text{col}(\mathcal{C})$ is equivalent to the span of its first two columns.
        \[\text{col}(\mathcal{C}) = \text{span}\{ \begin{bmatrix} 1 \\ 0 \\ 0 \end{bmatrix}, \begin{bmatrix} 0 \\ 0 \\ 1 \end{bmatrix}\}\]
    }

    \qitem \textbf{What is the rank of the controllability matrix? Is the system controllable?}

    \ws{\vspace{75px}}

    \sol{
        The controllability matrix has rank 2, so this system is not controllable.
        Note that the initial state $\vec{x}(0)$ has no effect on whether a system is controllable.
    }

    \qitem \textbf{Starting at $\vec{x}(0) = \begin{bmatrix} 1 \\ 0 \\ 0 \end{bmatrix}$, what possible states can we reach after one timestep? Two timesteps? Three?}
    
    \ws{\vspace{150px}}

    \sol{
        For one timestep,
        \[\vec{x}(1) = 
        \begin{bmatrix}
            0 & 1 & -2 \\
            0 & 2 & 0 \\
            -1 & 1 & 0
        \end{bmatrix} 
        \begin{bmatrix} 
            1 \\ 0 \\ 0 
        \end{bmatrix}
        + \begin{bmatrix}
            2 \\ 0 \\ 0
        \end{bmatrix} u(0)\]

        Simplifying, we get
        \[\vec{x}(1) =
        \begin{bmatrix} 
            0 \\ 0 \\ -1 
        \end{bmatrix} + 
        \begin{bmatrix} 
            2u(0) \\ 0 \\ 0 
        \end{bmatrix} = 
        \begin{bmatrix} 
            2u(0) \\ 0 \\ -1 
        \end{bmatrix}\]

        Since $u(0)$ is an input we have control over, we can set it arbitrarily, and reach any state of the form 
        \[ \begin{bmatrix} c_1 \\ 0 \\ -1\end{bmatrix}. \]
        For two timesteps,
        \[\vec{x}(2) = 
        \begin{bmatrix}
            0 & 1 & -2 \\
            0 & 2 & 0 \\
            -1 & 1 & 0
        \end{bmatrix} 
        \begin{bmatrix} 
            2u(0) \\ 0 \\ -1
        \end{bmatrix}
        + \begin{bmatrix}
            2 \\ 0 \\ 0
        \end{bmatrix} u(1) = 
        \begin{bmatrix} 
            2 + 2u(1) \\ 0 \\ -2u(0)
        \end{bmatrix}\]
        Again, since we have control over $u(0)$ and $u(1)$, we can reach any state of the form $\begin{bmatrix} c_1 \\ 0 \\ c_2 \end{bmatrix}$. \\
        \newline
        For 3 timesteps,
        \[\vec{x}(3) = 
        \begin{bmatrix}
            0 & 1 & -2 \\
            0 & 2 & 0 \\
            -1 & 1 & 0
        \end{bmatrix} 
        \begin{bmatrix} 
            2 + 2u(1) \\ 0 \\ -2u(0)
        \end{bmatrix}
        + \begin{bmatrix}
            2 \\ 0 \\ 0
        \end{bmatrix} u(2) = 
        \begin{bmatrix} 
            4u(0) + 2u(2) \\ 0 \\ -2 - 2u(1)
        \end{bmatrix}\]
        As with 2 timesteps, we can reach any state of the form $\begin{bmatrix} c_1 \\ 0 \\ c_2 \end{bmatrix}$.
    }

    \qitem \textbf{What is the minimum number of timesteps it takes to reach $\begin{bmatrix} 1 \\ 0 \\ 2 \end{bmatrix}$? 
    What about $\begin{bmatrix} 1 \\ 1 \\ 1 \end{bmatrix}$?}

    \ws{\vspace{100px}}

    \sol{
        It takes two timesteps to reach $\begin{bmatrix} 1 \\ 0 \\ 2 \end{bmatrix}$. \\
        $\begin{bmatrix} 1 \\ 0 \\ 2 \end{bmatrix}$ is in the form $\begin{bmatrix} c_1 \\ 0 \\ c_2 \end{bmatrix}$, but not in the form $\begin{bmatrix} c_1 \\ 0 \\ -1 \end{bmatrix}$, so it can be reached in at least two timesteps. \\
        \newline
        It is impossible to reach $\begin{bmatrix} 1 \\ 1 \\ 1 \end{bmatrix}$ in any amount of timesteps.
        The system is not controllable, and we can only reach states of the form $\begin{bmatrix} c_1 \\ 0 \\ c_2 \end{bmatrix}$.
    }
\end{enumerate}

Now, consider the system, with $A$ modified slightly:
\[\vec{x}(t + 1) = 
\begin{bmatrix}
    0 & 1 & 0 \\
    0 & 2 & -2 \\
    -1 & 1 & 0
\end{bmatrix} \vec{x}(t) + 
\begin{bmatrix}
    2 \\ 0 \\ 0
\end{bmatrix} u(t)\]

\begin{enumerate}[resume]
  \qitem \textbf{What is the controllability matrix, $\mathcal{C}$, for this system? What is its column space?}
  \ws{\vspace{100px}}
  \sol {
    \[\mathcal{C} = 
    \begin{bmatrix}
        \vec{b} & A\vec{b} & A^2 \vec{b}
    \end{bmatrix} = 
    \begin{bmatrix}
        2 & 0 & 0 \\
        0 & 0 & 4 \\
        0 & -2 & 0
    \end{bmatrix}\]
    The column space of this matrix spans $\mathbb{R}^3$.
  }

  \qitem \textbf{What is the rank of the controllability matrix? Is the system controllable?}

  \ws{\vspace{50px}}
  \sol {
    The controllability matrix has rank 3, so the system is controllable.
  }

  \qitem \textbf{Starting at $\vec{x}(0) = \begin{bmatrix} 1 \\ 0 \\ 0 \end{bmatrix}$, what possible states can we reach after one timestep? Two timesteps? Three?}

  \ws{\vspace{100px}}
  \sol {
    For one timestep, 
    \[\vec{x}(1) = 
    \begin{bmatrix}
        0 & 1 & 0 \\
        0 & 2 & -2 \\
        -1 & 1 & 0
    \end{bmatrix} 
    \begin{bmatrix} 
        1 \\ 0 \\ 0 
    \end{bmatrix}
    + \begin{bmatrix}
        2 \\ 0 \\ 0
    \end{bmatrix} u(0) =
    \begin{bmatrix} 
        2u(0) \\ 0 \\ -1 
    \end{bmatrix}\]
    So, we can reach any state of the form $\begin{bmatrix} c_1 \\ 0 \\ -1 \end{bmatrix}$. \\
    \newline

    For two timesteps,
    \[\vec{x}(2) = 
    \begin{bmatrix}
        0 & 1 & 0 \\
        0 & 2 & -2 \\
        -1 & 1 & 0
    \end{bmatrix} 
    \begin{bmatrix} 
        2u(0) \\ 0 \\ -1 
    \end{bmatrix}
    + \begin{bmatrix}
        2 \\ 0 \\ 0
    \end{bmatrix} u(1) =
    \begin{bmatrix} 
        2u(1) \\ 2 \\ -2u(0) 
    \end{bmatrix}\]
    We can reach any state of the form $\begin{bmatrix} c_1 \\ 2 \\ c_2 \end{bmatrix}$. \\
    \newline

    For three timesteps,
    \[\vec{x}(3) = 
    \begin{bmatrix}
        0 & 1 & 0 \\
        0 & 2 & -2 \\
        -1 & 1 & 0
    \end{bmatrix} 
    \begin{bmatrix} 
        2u(1) \\ 2 \\ -2u(0) 
    \end{bmatrix}
    + \begin{bmatrix}
        2 \\ 0 \\ 0
    \end{bmatrix} u(2) =
    \begin{bmatrix} 
        2 + 2u(2) \\ 4 + 4u(0) \\ 2 - 2u(1) 
    \end{bmatrix}\] 
    We can reach any state in $\mathbb{R}^3$ in three timesteps.
  }

  \qitem \textbf{What is the minimum number of timesteps it takes to reach $\begin{bmatrix} 1 \\ 0 \\ 2 \end{bmatrix}$?} \\
    \textit{Hint: 
      Since the system is controllable, we can reach any state in three time steps, however, it may be possible to reach a state in fewer than three time steps. Look at your answer to the previous part, and check which possible states we can reach in one, two, and three time steps.
    }

  \ws{\vspace{50px}}
    
  \sol {
    We unfortunately cannot reach this vector in fewer timesteps because $\begin{bmatrix} 1 \\ 0 \\ 2 \end{bmatrix}$ is not in the form $\begin{bmatrix} c_1 \\ 0 \\ -1 \end{bmatrix}$ or $\begin{bmatrix} c_1 \\ 2 \\ c_2 \end{bmatrix}.$
  }

  \qitem \textbf{What is the minimum number of timesteps it takes to reach $\begin{bmatrix} 1 \\ 2 \\ 3 \end{bmatrix}$?}
  \ws{\vspace{50px}}
  \sol {
    We can reach this vector in two time steps since it is in the form $\begin{bmatrix} c_1 \\ 0 \\ -1 \end{bmatrix}$ or $\begin{bmatrix} c_1 \\ 2 \\ c_2 \end{bmatrix}.$
  }

\end{enumerate}