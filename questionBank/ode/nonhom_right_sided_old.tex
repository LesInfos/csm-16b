% Authors: Justin Yu, Taejin Hwang
% Email: justinvyu@berkeley.edu, taejin@berkeley.edu

\qns{Non-homogeneous Differential Equations}

\meta {
	Prereqs: Solving a first-order homogeneous differential equation. 
}

Ordinary differential equations play a key role in modeling physical and electrical systems.
It was shown in lecture, and in homework that for a first-order, constant coefficient ordinary differential equation,
\begin{equation} \label{eq:h}
\ddt{}{t} x(t) = \lambda x(t)
\end{equation}

With initial condition $x(0) = x_0,$ the particular solution to the equation, $x_p(t),$ was uniquely determined as:
\begin{equation} \label{eq:hs}
x_p(t) = x_0 e^{\lambda t}
\end{equation}

For the purposes of this question, we will look at the the case in which we have a constant on the right hand side.
This is referred to as the non-homogeneous differential equation.

Now consider the non-homogeneous differential equation of the form:
\begin{equation} \label{eq:nh}
    \ddt{}{t} x(t)  = \alpha x(t) + \beta, \ \forall t \geq 0
\end{equation}
With initial condition $x(0) = x_0.$ \\
To solve this, we will use a change of variables.
Let $\widetilde{x}(t) = x(t) + \frac{\beta}{\alpha}$.

\begin{enumerate}

\qitem Try writing the original equation as a differential equation in terms of $\widetilde{x}(t)$. \\
Don't forget about the inital condition!

\meta {
  Hint to students: Try calculating $\ddt{}{t}x(t)$ and see if you can substitute.
}

\sol{
  Since $\alpha \widetilde{x}(t) = \alpha x(t) + \beta,$ and $\ddt{}{t}x(t) = \ddt{}{t}\widetilde{x}(t).$ 
  Substituting these values, we see that $\ddt{}{t}\widetilde{x}(t) = \alpha \widetilde{x}(t)$
  The initial condition will be $\widetilde{x}(0) = x(0) + \beta / \alpha.$ 
}

\qitem Does this equation look familiar? How can you solve this equation?

\sol{
  This is the homogeneous equation \eqref{eq:h} in terms of $\widetilde{x}(t)!$ 
  Since this is a homogeneous differential equation, in terms of $\widetilde{x}(t),$ using the particular solution $x_p(t)$ provided above, our solution is $\widetilde{x}_{p}(t) = \widetilde{x}_{0} e^{\alpha t}.$
}


\qitem What is the final solution $x(t)$? Assume $x(0)$ is given.

\sol{
  Converting back to $x(t),$ our solution is $x(t) = \widetilde{x}_{0} e^{\alpha t} - \beta / \alpha.$
  Then substituting the initial condition in, our final solution is $x(t) = x(0) e^{\alpha t} - \frac{\beta}{\alpha} (1 - e^{\alpha t}). $
}

\end{enumerate}

To recap, given a first order differential equation $\ddt{}{t} x(t) = \alpha x(t) + \beta$, the solution will be determined as:
\begin{equation} \label{eq:nhs}
    x(t) = x(0) e^{\alpha t} - \frac{\beta}{\alpha} (1 - e^{\alpha t})
\end{equation}

\meta{
  You can explain that $x(0)e^{\alpha t}$ is the decaying exponential of the initial condition, while \\ $\frac{\beta}{\alpha} (1 - e^{\alpha t})$ represents the growth of the steady state response.
}

% Another form you might find useful is the steady state form:
% \begin{align}
%     x(t) = y(\infty) + (x(0) - y(\infty)) e^{-\alpha t}
% \end{align}

% \meta{
% The second equation can be broken down into $y(\infty) = \beta / \alpha$ which is the steady state, and

% $[x(0) - y(\infty)] e^{-\alpha t}$ which is the decaying exponential of the initial state, from the steady state response.
% }



