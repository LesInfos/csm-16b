% {\Large \textbf{Mechanical:}}
\qns{An Introduction to Solving Differential Equations}
\qcontributor{Justin Yu, Taejin Hwang}

In this question, we will examine the process behind solving a 
first order differential equation and provide some motiviation for each step.

Consider the following first order differential equation:
\begin{align}
a \cdot \ddt{y(t)}{t} + b \cdot y(t) = c
\end{align}

We can divide the equation by $a$ to make the coefficient of $\ddt{y(t)}{t},$ one.
\begin{align}
\ddt{y(t)}{t} + \alpha \cdot y(t) = \beta
\end{align}

Our goal is to find a function $y(t)$ such that our differential equation 
is true for all values of $t.$
To do this, we use a guess and check approach.

\meta{
This question is supposed to walk through the process of solving a first order 
differential equation. While students may be familiar with the process, this 
question was made to motivate each step of the solving process.
}

\begin{enumerate}

\qitem Can you think of a function where $\ddt{y(t)}{t} = y(t)$ for all $t$?

\sol{
It can be seen either through inspection or integration that $y(t) = e^t.$
}

\qitem Now, how can you modify the function above to solve 
$\ddt{y(t)}{t} + \alpha y(t) = 0$?
This equation is known as the homogenous equation.

\sol{
We can notice that if $y(t) = e^{rt},$ then $\ddt{y(t)}{t} = re^{rt}.$
Therefore if we subtract $\alpha y(t),$ we get $\ddt{y(t)}{t} = -\alpha y(t).$ 
It follows by picking $r = -\alpha$ that $y(t) = e^{-\alpha t}$ 
satisfies our differential equation.
}

\end{enumerate}

You might notice that the solution above is not unique.
This is the reason a differential equation will often come 
with an initial condition such as $y(0) = 2$.

\begin{enumerate}[resume]

\qitem Try using this initial condition to solve for a 
unique solution to the differential equation above.

\meta{
Students may try plugging in $t = 0$ and arrive at a contradiction that $1 = 2.$
You may want to emphasize that $y(t) = e^{-\alpha t}$ isn't unique since 
$y(t) = Ke^{-\alpha t}$ works for any nonzero choice of $K \in \mathbb{R}.$

}

\sol{
Notice that our solution isn't unique since $y(t) = Ke^{-\alpha t}$ 
satisfies our differential equation for any nonzero choice of $K.$
Plugging in $t = 0,$ we get $y(0) = K = 2,$ so our solution is
$y(t) = 2 e^{-\alpha t}.$
}

\qitem Now, let's try solving our original equation:
\begin{align}
    \ddt{y(t)}{t} + \alpha y(t) = \beta
\end{align}.
To do this, we will use a change of variables.
Let $\widetilde{y}(t) = y(t) - \frac{\beta}{\alpha}$.

\begin{enumerate}

\qitem Try writing the original equation as a differential 
equation in terms of $\widetilde{y}(t)$.

\sol{
Since $y(t) = \widetilde{y}(t) + \beta / \alpha, 
\ddt{y(t)}{t} = \ddt{\widetilde{y}(t)}{t}.$ 

Substituting these values, 
we see that $\ddt{\widetilde{y}(t)}{t} + \alpha \widetilde{y}(t) + \beta = \beta$
or $\ddt{\widetilde{y}(t)}{t} + \alpha \widetilde{y}(t) = 0.$
}

\qitem Does this equation look familiar? How can you solve this equation?

\sol{
This is the homogenous equation in terms of $\widetilde{y}(t)!$ \\
So by parts (b) and (c), our solution is $\widetilde{y}(t) = Ke^{-\alpha t}.$	
}


\qitem What is the final solution $y(t)$? Assume $y(0)$ is given.

\sol{
Converting back to $y(t),$ our solution is $y(t) = Ke^{-\alpha t} + \beta / \alpha.$
Plugging in $t = 0,$ we get $y(0) = K + \beta / \alpha$ or $K = y(0) - \beta / \alpha.$
Our final solution is $y(t) = y(0) e^{-\alpha t} + \beta / \alpha (1 - e^{-\alpha t}).$
}

\end{enumerate}

\end{enumerate}

To recap, given a first order differential equation $\ddt{y(t)}{t} + \alpha y(t) = \beta$, the solution is:
\begin{align}
    y(t) = y(0) e^{-\alpha t} + \frac{\beta}{\alpha} (1 - e^{-\alpha t})
\end{align}

\meta{
You can explain that $y(0)e^{-\alpha t}$ is the decaying exponential
of the initial condition, while \\ $\frac{\beta}{\alpha} (1 - e^{-\alpha t})$ represents
the growth of the steady state response.
}

Another form you might find useful is the steady state form:
\begin{align}
    y(t) = y(\infty) + (y(0) - y(\infty)) e^{-\alpha t}
\end{align}

\meta{
The second equation can be broken down into $y(\infty)$ which is the steady state
and $[y(0) - y(\infty)] e^{-\alpha t}$ which is the decaying exponential
of the initial state, from the steady state response.
}

