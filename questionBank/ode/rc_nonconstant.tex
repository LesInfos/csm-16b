\qns{RC Circuits with Non-Constant Inputs}
\qcontributor{Taejin Hwang}
\qcontributor{Shreyas Krishnaswamy}
\qcontributor{Naomi Sagan}
\qcontributor{Justin Yu}

In an earlier question we looked at a way to solve differential equations with a constant $\beta.$
\par

We will now generalize this to an input, $u(t),$ which is an function independent of $x(t)$.
It was shown in homework that if we had the differential equation
\begin{equation} \label{eqn:odeinput}
  \ddt{x(t)}{t} = \lambda x(t) + u(t)
\end{equation}

With an initial condition $x(0) = x_0,$ the solution was uniquely determined by:
\begin{equation} \label{eqn:odesol}
    x_{p}(t) = x_{0}e^{\lambda{}t} + \int_0^t \! u(\tau{})e^{\lambda{}(t - \tau{})} \, d\tau{}
\end{equation}

\begin{enumerate}

\qitem Show that $x_{p}$(t) does indeed satisfy the differential equation \eqref{eqn:odeinput}. \\
\textit{Hint: Is there anything you can pull out of the integral to make sure there are no terms with $t$ in the integrand?}

\meta{
	The second line of work uses the product rule of derivatives combined with the Second Fundamental Theorem of Calculus.
}

\sol{
To show that $x_{p}$(t) satisfies equation \eqref{eqn:odeinput}, we have to show that it satisfies the the general differential equation and the initial condition. \\
For the general differential equation, we will show that $\ddt{x_p{t}}{t} = \lambda x_p(t) + u(t)$.
$$\ddt{x_p{t}}{t} = \lambda x_0 e^{\lambda t} + \ddt{}{t}(e^{\lambda t} \int_0^t \! u(\tau)e^{-\lambda \tau} \, d\tau)$$
$$\ddt{x_p{t}}{t} = \lambda x_0 e^{\lambda t} + \lambda e^{\lambda t} \int_0^t \! u(\tau)e^{-\lambda \tau} \, d\tau +  e^{\lambda t}u(t)e^{-\lambda t} = \lambda x_p(t) + u(t)$$
We also have to show that $x_p(0) = x_0$:
$$x_p(0) = x_0e^{\lambda \cdot 0} + \int_0^0 \! u(\tau)e^{-\lambda \tau} \, d\tau = x_0 + 0 = x_0$$
}

\end{enumerate}

We will now take this insight and apply it to an RC circuit with non-constant input. Suppose $V_{c}(0) = 5V$.

\begin{figure}[H]
 \begin{figure}[H]
	\begin{centering}
		\begin{circuitikz}
			\draw (0, 2)
			to[V = $u(t)$] (0, 0)
			to node[ground]{} (0, 0);
			\draw (0, 2)
			to[R = $R$] (2.5, 2)
			to node[circ, label={$V_{C}(t)$}]{} (2.5, 2);
			\draw (2.5, 2)
			to[C = $C$] (2.5, 0)
			to node[ground]{} (2.5, 0);
		\end{circuitikz}
		\caption{\label{fig:circuit}RC Circuit with Voltage Source}
	\end{centering}
\end{figure}

\end{figure}
Let $u(t) = cos(\omega{}t)$, where $\omega{}$ denotes the frequency of the cosine wave.

\begin{enumerate}[resume]

\qitem Write out the KCL equations combined with Ohm's Law and the voltage-current relation of a capacitor.

\sol{
For this solution, we labeled the currents in the circuit as the following: \\
\begin{figure}[H]
 \begin{figure}[H]
	\begin{centering}
		\begin{circuitikz}
			\draw (0, 2)
			to[V = $u(t)$] (0, 0)
			to node[ground]{} (0, 0);
			\draw (2.5, 2)
			to node[circ, label={$V_{C}(t)$}]{} (2.5, 2)
			to[R = $R$, i^<=$i_R$] (0, 2);
			\draw (2.5, 2)
			to[C = $C$, i=$i_C$] (2.5, 0)
			to node[ground]{} (2.5, 0);
		\end{circuitikz}
		\caption{\label{fig:circuit}RC Circuit with Voltage Source}
	\end{centering}
\end{figure}

\end{figure}
Applying the KCL at the $V_C$ node, we get $I_R(t) = I_C(t)$
Substituting resistor and capacitor IV relationships, we have
  $$\frac{u(t) - V_C(t)}{R} = C \ddt{V_C(t)}{t}$$
}

\qitem Show that your equation above can be put into the form of equation \eqref{eqn:odeinput}.

\sol{
Rearranging terms in out KCL equation,
  $$\ddt{V_C(t)}{t} = - \frac{V_C(t)}{RC} + \frac{cos(\omega t)}{RC}$$
}

\qitem In this case, what are the values of $\lambda{}$, $x(t)$, $u(t),$ and $x_{0}$?

\meta{
	We wrote this question to make sure students are familiar with type-checking.	
}

\sol{
  $\lambda$ is $-\frac{1}{RC}, \, x(t)$ is $V_C(t)$, $u(t)$ is $\cos (\omega t)$ and our initial condition $x_{0} = V_C(0) = 5V$ 
}

\qitem Using this particular solution, write out an expression for $V_{C}(t)$.

\sol{
  Plugging our values of $x(t)$, $\lambda{}$, and $x_{0}$ into equation \eqref{eqn:odesol}, we get
    $$V_C(t) = 5e^{-\frac{t}{RC}} + \int_0^t \! \frac{cos(\omega \tau)}{RC} \text{exp}(-\frac{t - \tau}{RC}) \, d\tau$$
}

\qitem What is the closed form solution to the voltage across the capacitor, $V_{c}(t)$.\\
\textit{Hint:} $\int \! e^{at}cos(bt) \, dt = \frac{e^{at} (a \cdot cos(bt) + b \cdot sin(bt))}{a^2 + b^2} + C$

\meta{
	This is an extremely nasty integral. The purpose of the question obviously isn't to understand the nastiness of the integral.
	Most of this problem is done once they realize what the values of $a$ and $b$ are in the hint.
	The main purpose of this part is to show students how difficult transient analysis can get, and should foreshadow phasors.	
}

\sol{
  Rearranging terms in our expression for $V_{C}(t)$, we get 
  $$V_C(t) = 5e^{-\frac{t}{RC}} + \frac{e^{-\frac{t}{RC}}}{RC} \int_0^t \! cos(\omega \tau) e^{\frac{\tau}{RC}} \, d\tau$$
  Using the hint above to evaluate the integral (with $a = \frac{1}{RC}$ and $b = \omega$):
  $$V_C(t) = 5e^{-\frac{t}{RC}} + \frac{e^{-\frac{t}{RC}}}{RC} \cdot \frac{e^{\frac{t}{RC}} (\frac{1}{RC} \cdot cos(\omega t) + \omega \cdot sin(\omega t)) - \frac{e^{0}}{RC} \cos(\omega \cdot 0)}{\frac{1}{R^2C^2} + \omega^2}$$
  Simplifying slightly,
  $$V_C(t) = 5e^{-\frac{t}{RC}} + \frac{1}{(RC)^2} \frac{cos(\omega t) + \omega RC \cdot sin(\omega t) - e^{-\frac{t}{RC}}}{\frac{1}{(RC)^2} + \omega^2}$$
  So it follows that,
  $$V_C(t) = 5e^{-\frac{t}{RC}} + \frac{cos(\omega t) + \omega RC \cdot sin(\omega t) - e^{-\frac{t}{RC}}}{1 + (\omega RC)^2}$$
}

\end{enumerate}
