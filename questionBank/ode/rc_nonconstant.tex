\qns{RC Circuits with Non-Constant Inputs}
\qcontributor{Taejin Hwang}
\qcontributor{Shreyas Krishnaswamy}
\qcontributor{Naomi Sagan}
\qcontributor{Justin Yu}

We can recall that for a scalar first-order ODE,

\begin{equation}
    \frac{d}{dt} x(t) = \lambda{}x(t) \text{}
\end{equation}

with initial condition $x(0) = x_{0}$,
the solution is

\begin{equation*}
    x(t) = x_{0}e^{\lambda{}t}
\end{equation*}

Recall from homework that we introduced an input $u(t)$ to this differential equation, where $u(t)$ is an independent function of $x(t)$.

\begin{equation}
    x_{p}(t) = x_{0}e^{\lambda{}t} + \int_0^\tau{} \! u(\tau{})e^{\lambda{}(t - \tau{})} \, d\tau{}
\end{equation}

\begin{enumerate}

\qitem Show why $x_{0}$(t) does indeed satisfy the differential equation (2).\\

Now we will take this insight and apply it to an RC circuit with non-constant input. Suppose $V_{c}(0) = 5V$.\\

\begin{figure}[H]
 \centering
 \begin{figure}[H]
	\begin{centering}
		\begin{circuitikz}
			\draw (0, 2)
			to[V = $u(t)$] (0, 0)
			to node[ground]{} (0, 0);
			\draw (0, 2)
			to[R = $R$] (2.5, 2)
			to node[circ, label={$V_{C}(t)$}]{} (2.5, 2);
			\draw (2.5, 2)
			to[C = $C$] (2.5, 0)
			to node[ground]{} (2.5, 0);
		\end{circuitikz}
		\caption{\label{fig:circuit}RC Circuit with Voltage Source}
	\end{centering}
\end{figure}

 \caption{Circuit A}
\end{figure}

Let $u(t) = cos(\omega{}t)$, where $\omega{}$ denotes the "frequency" the cosine wave.

\sol{

}

\qitem Write out the KCL equations combined with Ohm's Law and the voltage-current relation of a capacitor.

\sol{

}

\qitem Show that your equation above can be put into the form of equation (2).

\sol{

}

\qitem In this case, what is the value of $x(t)$, $\lambda{}$, and $x_{0}$?

\sol{

}

\qitem Using this particular solution, write out an expression for $V_{c}(t)$.

\sol{

}

\qitem What is the closed form solution to the voltage across the capacitor, $V_{c}(t)$.\\

Hint: $\int \! e^{-t}cos(t) \, dt = \frac{1}{2}(e^{-t}sin(t) + e^{-t}cos(t)) + C$

\sol{

}

\end{enumerate}
