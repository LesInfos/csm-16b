% Authors: Justin Yu, Taejin Hwang
% Email: justinvyu@berkeley.edu, taejin@berkeley.edu

\qns{Non-homogeneous Differential Equations}

\meta {
	Prereqs: Solving a first-order homogeneous differential equation. 
}

Ordinary differential equations play a key role in modeling physical and electrical systems.
It was shown in lecture and in homework that for a first-order, constant coefficient ordinary differential equation,

\begin{equation} \label{eq:h}
\ddt{}{t} x(t) = \lambda x(t)
\end{equation}

with initial condition $x(0) = x_0,$ the particular solution to the equation, $x_p(t),$ was uniquely determined as:
\begin{equation} \label{eq:hs}
x_p(t) = x_0 e^{\lambda t}
\end{equation}

For the purposes of this question, we will look at the case in which we have a constant on the right hand side.
This is referred to as a \textbf{non-homogeneous differential equation}.

Now consider the following non-homogeneous differential equation with initial condition $x(0) = x_0$:
\begin{equation} \label{eq:nh}
    \ddt{}{t} x(t)  = \alpha x(t) + \beta, \ \forall t \geq 0
\end{equation}


To solve this, we will use a change of variables.
Let $\widetilde{x}(t) = x(t) + \frac{\beta}{\alpha}$.

\begin{enumerate}

\qitem \textbf{Try writing equation (\ref{eq:nh}) as a differential equation in terms of $\widetilde{x}(t)$.} \\
Don't forget about the initial condition!

\ws{
	\vspace{75px}
}

\meta {
  Hint to students: Try calculating $\ddt{}{t}x(t)$ and see if you can substitute.
}

\sol{
  Since $\alpha \widetilde{x}(t) = \alpha x(t) + \beta,$ and $\ddt{}{t}x(t) = \ddt{}{t}\widetilde{x}(t).$ 
  Substituting these values, we see that $\ddt{}{t}\widetilde{x}(t) = \alpha \widetilde{x}(t)$
  The initial condition will be $\widetilde{x}(0) = x(0) + \beta / \alpha.$ 
}

\qitem \textbf{Does this equation look familiar? How can you solve this equation?}

\ws{
	\vspace{75px}
}

\sol{
  This is the homogeneous equation \eqref{eq:h} in terms of $\widetilde{x}(t)!$ 
  Since this is a homogeneous differential equation, in terms of $\widetilde{x}(t),$ using the particular solution $x_p(t)$ provided above, our solution is $\widetilde{x}_{p}(t) = \widetilde{x}_{0} e^{\alpha t}.$
}

\qitem \textbf{What is the final solution $x(t)$?} Assume that $x(0) = x_0$ is a given constant.

\ws{
	\vspace{75px}
}

\sol{
  Converting back to $x(t),$ our solution is $x(t) = \widetilde{x}_{0} e^{\alpha t} - \beta / \alpha.$
  Then substituting the initial condition in, our final solution is $x(t) = x(0) e^{\alpha t} - \frac{\beta}{\alpha} (1 - e^{\alpha t}). $
}

\end{enumerate}

To recap, given a first order differential equation $\ddt{}{t} x(t) = \alpha x(t) + \beta$, the solution will be determined as:
\begin{equation} \label{eq:nhs}
    x(t) = x(0) e^{\alpha t} - \frac{\beta}{\alpha} (1 - e^{\alpha t})
\end{equation}

\meta{
  You can explain that $x(0)e^{\alpha t}$ is the decaying exponential of the initial condition, while \\ $\frac{\beta}{\alpha} (1 - e^{\alpha t})$ represents the growth of the steady state response.
}

% Another form you might find useful is the steady state form:
% \begin{align}
%     x(t) = y(\infty) + (x(0) - y(\infty)) e^{-\alpha t}
% \end{align}

% \meta{
% The second equation can be broken down into $y(\infty) = \beta / \alpha$ which is the steady state, and

% $[x(0) - y(\infty)] e^{-\alpha t}$ which is the decaying exponential of the initial state, from the steady state response.
% }

% \begin{enumerate}[resume]

% \qitem Show that $x_{p}$(t) does indeed satisfy the differential equation \eqref{eqn:odeinput}. \\
% \textit{Hint: Is there anything you can pull out of the integral to make sure there are no terms with $t$ in the integrand?}

% \meta{
%   The second line of work uses the product rule of derivatives combined with the Second Fundamental Theorem of Calculus.
% }

% \sol{
% To show that $x_{p}$(t) satisfies equation \eqref{eqn:odeinput}, we have to show that it satisfies the the general differential equation and the initial condition. \\
% For the general differential equation, we will show that $\ddt{x_p(t)}{t} = \lambda x_p(t) + u(t)$.
% $$\ddt{x_p(t)}{t} = \lambda x_0 e^{\lambda t} + \ddt{}{t}(e^{\lambda t} \int_0^t \! u(\tau)e^{-\lambda \tau} \, d\tau)$$
% $$\ddt{x_p(t)}{t} = \lambda x_0 e^{\lambda t} + \lambda e^{\lambda t} \int_0^t \! u(\tau)e^{-\lambda \tau} \, d\tau +  e^{\lambda t}u(t)e^{-\lambda t} = \lambda x_p(t) + u(t)$$
% We also have to show that $x_p(0) = x_0$:
% $$x_p(0) = x_0e^{\lambda \cdot 0} + \int_0^0 \! u(\tau)e^{-\lambda \tau} \, d\tau = x_0 + 0 = x_0$$
% }

% \end{enumerate}

We will now take this insight and apply it to the process of charging up a capacitor.
\newline

Consider the circuit below, assume that when $t\leq 0$, the capacitor has no charge stored $(V_{C}(t=0) = 0)$.
At $t=0$, the switch closes. Assume that $V_s=\SI{5}{\volt}$, $R=\SI{100}{\ohm}$, and $C=\SI{10}{\micro\farad}$.

\begin{figure}[H]
	\begin{centering}
		\begin{circuitikz}
			\draw (0, 4)
			to[V =$V_s$] (0, 0);
			\draw (0, 4)
			to[switch,l^=\mbox{$t = 0$}](4,4)
			(4,4) to[R = $R$,v=$V_R(t)$,i>^=$i_R(t)$] (7,4)	
			to [short] (9,4)
			to[C = $C$, v=$V_C(t)$,i>^=$i_C(t)$] (9,0)
			to [short] (0,0);
		\end{circuitikz}
		\caption{\label{fig:circuit}RC Circuit with Voltage Source}
	\end{centering}
\end{figure}

\meta{
	The rest of this problem is meant to be an introduction to transient analysis of circuits.
	Make sure students fully understand the concepts presented (how to get a differential equation from the circuit, how to solve the equation, and the physical intuition of what is happening in the circuit - charging up a capacitor) before moving on.
}
\begin{enumerate}[resume]

\qitem \textbf{Write out the KCL equations associated with the circuit when the switch is closed.}

\ws{\vspace{75px}}

\meta{
	To simplify your circuit analysis, mark the negative end of the voltage source as ground.
}

\sol{
\begin{align*}
\intertext{We denote the voltage to the left of resistor to be $V_s.$}
\intertext{By KCL, the current through the resistor } i_R = i_{\text{C}}.
\intertext{Ohm's law tells us that } i_R = \frac{V_s - V_{C}}{R}
\intertext{The voltage-current relationship of a capacitor tells us that }i_\text{C} = C\ddt{V_{C}}{t}
\intertext{Therefore, } \frac{V_s - V_{C}}{R} = C \ddt{V_{C}}{t}
\end{align*}
}

\qitem \textbf{Write out the differential equation for $V_{\text{C}}(t)$ after the switch closes.}

\ws{
\vspace{75px}
}

\meta{
	If students are confused about how to get started, point them to equation \eqref{eq:nh} in problem 1 as a template.
	Which KCL equation from part (a) of this problem resembles a first order differential equation (i.e. which one has a differential term)?
}

\sol{
\begin{align*}
\intertext{From the previous problem we know that when the switch is closed,}
\frac{V_{\text{s}}- V_{C}}{R} = C\frac{dV_{C}}{dt} \\
\intertext{Thus we obtain}
\ddt{V_{C}}{t} + \frac{V_{C}}{RC} = \frac{V_{\text{S}}}{RC}
\end{align*}
}

\sol{
No charge is on the capacitor before time $t=0$. Using $q=VC$, we know that $V_{C}=\SI{0}{\volt}$ before $t=0$.
At $t=0$, the switch closes. Since voltage across the capacitor cannot change instantaneously, $V_{C}(t=0)=0.$
As $t$ goes to infinity, the capacitor will become fully charged and the current goes to zero.
Therefore, the voltage of the capacitor equals the voltage source: $V_{C}(t \to \infty) = V_{\text{S}}.$
}

\qitem \textbf{Using the initial conditions found in the previous parts, find an expression for
$V_{\text{C}}(t)$ in terms of $V_{\text{s}}$, $R$, and $C$.}

\meta{
	You can point at equation \eqref{eq:nhs} and find the appropriate values of $\alpha$ and $\beta,$ 
	but let them know that they are still expected to know how to do the process below. 	
}


\sol{
	To find an expression for $V_{out}(t)$:
	\[V_c(t)=V_{out}(t)\]
	\[\frac{dV_{out}(t)}{dt}+\frac{1}{RC}V_{out}(t)=\frac{V_{S}}{RC}\]

	Using substitution of variables:
	\[x(t) = V_{out}(t)-V_{S}\]
	\[V_{out}(t)=x(t)+V_{S}\]
	\[\frac{dV_{out}(t)}{dt}=\frac{dx(t)}{dt}\]
	\[\frac{dx(t)}{dt}+\frac{1}{RC}x(t)=0\]
	\[x(t)=Ae^{-\frac{t}{RC}}\]

	Now substituting back to find $V_{out}$:
	\[V_{out}(t)=V_{S}+Ae^{-\frac{t}{RC}}\]

	Using our initial condition:
	\[V_{out}(0)=0=V_{S}+A\]
	\[A=-V_{S}\]
	\[V_{out}(t)=V_{S}\left(1-e^{-\frac{t}{RC}}\right)\]
	\[V_c(t) = V_S(1 - e^{-\frac{t}{\tau}}) \text{ where } \tau = RC \text{ ($\tau$ explained in next part)}\]
}

\end{enumerate}