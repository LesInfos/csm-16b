% Authors: Naomi Sagan, Justin Yu

\qns{RC Transistor Model}

Consider a single CMOS inverter with a voltage input, except there is some resistance between the voltage source and the gates of the transistors.

\vspace{-1em}

\begin{center}
    \begin{figure}[H]
	\begin{centering}
		\begin{circuitikz}[scale=0.8]
				\draw (-1,4) 
				to [V=$V_{in}$](-1, 2) to (-1,2) node[ground]{}; 
				\draw (4,5) node[pmos, emptycircle](pm){} ;
				\draw (4,3) node[nmos, ](nm){} ;
				\draw (-1,4) to [R = $R$] (2,4) to [short] (2,5)
				to (pm.gate);
				\draw (pm.source) to [short, -*] (4,6) node[above]{$V_{DD}$};;
				\draw (pm.drain) to (nm.drain);
				\draw (2,4) to [short] (2, 3) to (nm.gate);
				\draw (nm.source) to (4,2) node[ground]{};
				\draw (4,4) [short] to [short, -*] (5,4) node[right]{$V_{out}$};
		\end{circuitikz}
		\caption{\label{fig:circuit}CMOS Inverter with Resistance}
	\end{centering}
\end{figure}
\end{center}

\vspace{-4em}

\begin{enumerate}

\qitem First, let's introduce a way to model CMOS transistors.
A transistor can be modeled with a capacitor between the gate $G$ and source $S$, an internal resistance, and a voltage-controlled switch between drain $D$ and $S$.\\

\textbf{Using the NMOS and PMOS transistor models below, redraw the inverter as a simple RC circuit.} 

\textit{Note: the voltage we are examining for this problem is $V_{GS}$ of the NMOS transistor.}

\begin{figure}[H]
    \centering
    \begin{subfigure}{0.45\linewidth}
        \centering
        \begin{circuitikz}
            \draw (0,-2) node[ground,label=right:$S$] {}
            to[switch,l_= $V_{GS} \geq V_{tn}$]
            (0,0) to[R,-*,l=$R_{on}$,i<=$I_D$] ++(0,2)
            node[label=left:$D$] {};;
            \draw (0,-2) -- (-1,-2)
            to[C,-*,l=$C_G$] ++(0,2) node[label=left:$G$] {};;
            \draw (0,1.75) to [short,-*,l=$V_{out}$] ++ (1,0);;
        \end{circuitikz}
        \caption*{\label{fig:nmoscap} \small{NMOS Resistor-Switch-Capacitor Model}}
    \end{subfigure}
	\begin{subfigure}{0.45\linewidth}
        \centering
		\begin{circuitikz}
			\draw (0, -2) node[label=left:$D$] {}
			to[switch,l_= $V_{GS} \leq -|V_{tp}|$, *-] (0,0)
			to[R, -*,l_=$R_{on}$, i<=$I_D$] ++(0, 2)
			node[vcc,label=right:$S$] {};;
			\draw (0, 2) -- (-1, 2)
			to[C, -*, l_=$C_G$] ++(0, -2) node[label=left:$G$] {};;
			\draw (0, -1.75) to [short, -*, l=$V_{out}$] ++(1,0);;
		\end{circuitikz}
		\caption*{\label{fig:pmoscap} \small{PMOS Resistor-Switch-Capacitor Model}}
	\end{subfigure}
\end{figure}

\ws{\vspace{300px}}

\sol{
    We begin by replacing the transistor elements with the Resistor-Switch-Capacitor model:

    \begin{center}
        \begin{circuitikz}[scale=0.8]
                \draw (-2 ,4) 
                to [V=$V_{in}$](-2, 2) to (-2, 2) node[ground]{}; 

                \draw (-2,4) to [R = $R$, -*] (1, 4) node[right]{$G$}
                    to [short] (1, 8)
                    to [C = $C_{GS}$, v=$V_{GS, P}$] (4, 8)
                    to [short] (4, 9)
                    node[above, vdd]{$V_{DD}$};
                \draw (4, 8) to [R=$R_{on, p}$] (4, 6)
                    to [switch] (4, 4);

                \draw (1, 4) to [short] (1, 0)
                    to [C = $C_{GS}$, v=$V_{GS, N}(t)$] (4, 0);
                \draw (4, 0) to (4, -1) node[ground]{};
                \draw (4, 0) to [switch] (4, 2)
                    to [R=$R_{on, n}$] (4, 4);

                \draw (4,4) [short] to [short, -*] (5,4) node[right]{$V_{out}$};
        \end{circuitikz}
        % \caption{\label{fig:circuit}CMOS Inverter with Resistance}
    \end{center}

    Next, we want to condense it down to the following simple RC circuit. To do so, we want to remove all elements that \textbf{don't affect the voltage in question}, which is $V_{GS, N}$. \\

    In this case, the resistors $R_{on, n}$ and $R_{on, p}$ cannot affect the potential at $G$, $V_{DD}$, or ground, so we don't need to include them in our circuit. Making that simplification, we get:

    \begin{center}
        \begin{circuitikz}[scale=0.8]
                \draw (-2, 4) 
                to [V_=$V_{in}$](-2, 1) to (-2, 1) node[ground]{}; 

                \draw (-2,4) to [R = $R$, -*, i=$i_1$] (2, 4)
                    node[right]{$G$}
                    to [C = $C_{GS}$, i=$i_2$, v=$V_{GS, P}$] (2, 7)
                    node[above, vdd]{$V_{DD}$};

                \draw (2,4) to [C = $C_{GS}$, v=$V_{GS, N}(t)$, i=$i_3$] (2, 1)
                    node[ground]{};
        \end{circuitikz}
    \end{center}
}

\qitem
We want to analyze the behavior of the inverter by solving for the gate voltage $V_{GS, N}$ \textit{of the NMOS transistor} as a function of time. \textbf{Using KCL or KVL, write out a differential equation for $V_{GS, N}$.}

\ws{\vspace{125px}}

\sol {
    Using KCL at the $G$ node:

    \begin{align*}
        i_1 &= i_2 + i_3 \\
        \frac{V_{in} - V_{GS, N}(t)}{R} &= C_{GS} \frac{d}{dt} (V_{GS, N}(t) - V_{DD}) + C_{GS} \frac{d}{dt} V_{GS, N}(t) \\
        \frac{V_{in} - V_{GS, N}(t)}{R} &= 2C_{GS} \frac{d}{dt} V_{GS, N}(t) \\
        \frac{d}{dt} V_{GS, N}(t) &= -\frac{1}{2RC_{GS}} V_{GS, N}(t) + \frac{1}{2RC_{GS}} V_{in}
    \end{align*}
}

\qitem 
Assume that $V_{in}$ has been low ($V_{in} = 0$) for a long time and switches to high ($V_{in} = V_{DD}$) at time $t = 0$. \textbf{Solve for $V_{GS, N}(t)$ using the differential equation from the last part and the above initial condition.}

\ws{\vspace{100px}}

\sol{
    Since we know that $V_{in}$ has been low for a long time before $t = 0$, the inverter has stabilized to the top PMOS transistor having $V_{GS, P}(0) = V_{DD}$ and the bottom NMOS transistor having $V_{GS, N}(0) = 0$. Thus, we can plug in this initial condition when solving the differential equation. \\

	Using the substitution of variables $V_{GS, N}(t) = z(t) + V_{in}$ so that we get a homogeneous first-order ODE in terms of $z(t)$:
	\[\frac{dV_{GS, N}(t)}{dt}=\frac{d z(t) }{dt}\]
	\[\frac{dz(t)}{dt} = - \frac{1}{2RC_{GS}}z(t)\]
	\[z(t)=Ae^{-\frac{t}{2RC_{GS}}}\]

	Now substituting back to find $V_{out}$:
	\[V_{GS, N}(t)=Ae^{-\frac{t}{2RC_{GS}}} + V_{in}\]

	Using our initial condition:
	\[V_{GS, N}(0)= 0 = V_{in} + Ae^0 \]
	\[A=-V_{in}\]
	\[V_{GS, N}(t) = V_{in}\left(1-e^{-\frac{t}{2RC_{GS}}}\right) = V_{DD}\left(1-e^{-\frac{t}{2RC_{GS}}}\right) \]
}

\end{enumerate}