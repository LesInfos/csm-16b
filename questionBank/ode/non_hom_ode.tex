\qns{Non-homogeneous Differential Equations}
\qcontributor{Justin Yu, Taejin Hwang}

Ordinary differential equations play a key role into modeling physical and electrical systems.
It was shown in lecture, and in homework that for a first-order, constant coefficient ordinary differential equation,
\begin{align}
\ddt{y(t)}{t} + \lambda \cdot y(t) = 0
\end{align}

With initial condition $y(0) = y_0,$ the solution to the equation, $y_p(t),$ was uniquely determined as:
\begin{align}
y_p(t) = y_0 e^{-\lambda t}
\end{align}

For the purposes of this question, we will look at the the case in which the right hand side is no longer 0, and is a constant.
This is referred to as the non-homogeneous differential equation.

To recall, the homogeneous differntial equation is of the form:
\begin{align}
    \ddt{y(t)}{t} + \alpha y(t) = \beta
\end{align}
To solve this, we will use a change of variables.
Let $\widetilde{y}(t) = y(t) - \frac{\beta}{\alpha}$.

\begin{enumerate}

\qitem Try writing the original equation as a differential 
equation in terms of $\widetilde{y}(t)$.

\sol{
Since $y(t) = \widetilde{y}(t) + \beta / \alpha, \ddt{y(t)}{t} = \ddt{\widetilde{y}(t)}{t}.$ 

Substituting these values, we see that $\ddt{\widetilde{y}(t)}{t} + \alpha \widetilde{y}(t) + \beta = \beta$
or $\ddt{\widetilde{y}(t)}{t} + \alpha \widetilde{y}(t) = 0.$
}

\qitem Does this equation look familiar? How can you solve this equation?

\sol{
This is the homogeneous equation in terms of $\widetilde{y}(t)!$

Therefore using equation (2) provided above, our solution is $\widetilde{y}(t) = \widetilde{y}_{0} e^{-\alpha t}.$	
}


\qitem What is the final solution $y(t)$? Assume $y(0)$ is given.

\sol{
Converting back to $y(t),$ our solution is $y(t) = \widetilde{y}_{0} e^{-\alpha t} + \beta / \alpha.$

We can recall that $\widetilde{y}_{0} = \widetilde{y}(0) = y(0) - \beta / \alpha.$ 
Then substituting this in, our final solution is $y(t) = y(0) e^{-\alpha t} + \beta / \alpha (1 - e^{-\alpha t}).$
}

\end{enumerate}

To recap, given a first order differential equation $\ddt{y(t)}{t} + \alpha y(t) = \beta$, the solution will be determined as:
\begin{align}
    y(t) = y(0) e^{-\alpha t} + \frac{\beta}{\alpha} (1 - e^{-\alpha t})
\end{align}

\meta{
You can explain that $y(0)e^{-\alpha t}$ is the decaying exponential
of the initial condition, while \\ $\frac{\beta}{\alpha} (1 - e^{-\alpha t})$ represents
the growth of the steady state response.
}

% Another form you might find useful is the steady state form:
% \begin{align}
%     y(t) = y(\infty) + (y(0) - y(\infty)) e^{-\alpha t}
% \end{align}

% \meta{
% The second equation can be broken down into $y(\infty) = \beta / \alpha$ which is the steady state, and

% $[y(0) - y(\infty)] e^{-\alpha t}$ which is the decaying exponential of the initial state, from the steady state response.
% }



