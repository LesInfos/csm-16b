\qns{RC Circuit}
\qcontributor{Kyle Tanghe}
\qcontributor{Elena Jia}

Consider the circuit below, assume that when $t\leq0$, the capacitor has no charge stored $(V_{\text{c}}(t=0) = 0)$. At $t=0$, the switch closes. Assume that $V_s=\SI{5}{\volt}$, $R=\SI{100}{\ohm}$, and $C=\SI{10}{\micro\farad}$. 

\begin{figure}[H]
	\begin{centering}
		\begin{circuitikz}
			\draw (0, 4)
			to[V =$V_s$] (0, 0);
			\draw (0, 4)
			to[switch,l^=\mbox{$t = 0$}](4,4)
			(4,4) to[R = $R$,v=$V_R(t)$,i>^=$i_R(t)$] (7,4)	
			to [short] (9,4)
			to[C = $C$, v=$V_C(t)$,i>^=$i_C(t)$] (9,0)
			to [short] (0,0);
		\end{circuitikz}
		\caption{\label{fig:circuit}RC Circuit with Voltage Source}
	\end{centering}
\end{figure}


\begin{enumerate}

\qitem Write out the KCL equations associated with the circuit when the switch is open. Then write out the differential equations for the case when the switch is closed.

\meta{
	Testing meta here
}

\sol{
\begin{align*}
\intertext{When the swtich is open, we denote the voltage to the left of resistor to be $y$. We obtain the following differential equations. KCL at node between the resistor and capacitor gives}
\frac{y - V_{\text{C}}}{R} = C\frac{dV_{\text{C}}}{dt} \\
\intertext{Since the current going into the resistor and the voltage source is zero, KCL gives:}
\frac{y - V_{\text{C}}}{R} = 0 \\
\intertext{and}
i_{\text{Vs}} = 0\\
\intertext{When the switch is closed, the following still holds:}
\frac{y - V_{\text{C}}}{R} = C\frac{dV_{\text{C}}}{dt} \\
\intertext{Additionally, KCL at the node between the resistor and voltage sources gives:}
\frac{V_{\text{S}} - V_{\text{C}}}{R} +  i_{\text{Vs}} = 0
\end{align*}
}



\qitem Write out the differential equation for $V_{\text{c}}(t)$ after the switch closes.

\sol{
\begin{align*}
\intertext{From the previous problem we know that when the switch is closed,}
\frac{V_{\text{S}}- V_{\text{C}}}{R} = C\frac{dV_{\text{C}}}{dt} \\
\intertext{Thus we obtain}
C\frac{dV_{\text{C}}}{dt} + \frac{V_{\text{C}}}{R} - \frac{V_{\text{S}}}{R}=0
\end{align*}
}



\qitem What is the initial condition for $V_{\text{c}}(t)$ (i.e. $V_{\text{c}}(t=0)$) and what is $V_{\text{c}}(t \to \infty)$?

\sol{

\begin{align*}
\intertext{No charge is on the capacitor before time $t=0$. Using $q=VC$, we know that $V_{\text{c}}=\SI{0}{\volt}$ before $t=0$.} 
\intertext{At $t=0$, the switch closes. Since voltage across the capacitor cannot change instantaneously,}
V_{\text{c}}(t=0)&=0. \\
\intertext{As $t$ goes to infinity, the capacitor will become fully charged and the current goes to zero. Therefore, the voltage of the capacitor equals the voltage source:}
V_{\text{c}}(t \to \infty)&=V_{\text{S}}.
\end{align*}
}




\qitem Using the initial conditions found in the previous parts, find an expression for $V_{\text{c}}(t)$ in terms of $V_{\text{s}}$, $R$, and $C$.

\sol{
	\begin{align*}
		\intertext{The general solution to the equation}
		\frac{dy}{dt}&=\lambda y \\
		\intertext{is}
		y(t)&=Ke^{\lambda t}, \\
		\intertext{where $K$ is a constant and $\lambda$ is the eigenvalue of the equation. In our case, we know}
		C\frac{dV_{\text{C}}}{dt} + \frac{V_{\text{C}}}{R} - \frac{V_{\text{S}}}{R}=0,\\
		\intertext{which can be written as}
		\frac{dV_{\text{C}}}{dt} = - \frac{V_{\text{C}}}{RC} + \frac{V_{\text{S}}}{RC}.\\
		\intertext{The solution will be in the form}
		V_{\text{c}}(t)&=Ke^{-\frac{t}{RC}} + V_{\text{S}}. \\
		\intertext{To find $K$, we can plug in our initial condition at $t=0$:}
		V_{\text{c}}(t=0)&=K + V_{\text{S}} = 0.\\
		\intertext{So our overall equation ends up being}
		V_{\text{c}}(t)&=-V_{\text{S}} e^{-\frac{t}{RC}} + V_{\text{S}}. \\
		% % I_{\text{c}}(t=0)&=Ke^{-\frac{0}{RC}}=\frac{V_{\text{s}}}{R} \\
		% % \intertext{From this we can see}
		% % K&=\frac{V_{\text{s}}}{R} \\
		% % \intertext{So our overall equation ends up being}
		% % I_{\text{c}}(t)&=\frac{V_{\text{s}}}{R}e^{-\frac{t}{RC}} \\
		\intertext{We can also double check our answer by looking at the state for t $\to \infty$:}
		V_{\text{c}}(t \to \infty)&=-V_{\text{S}} e^{-\infty} + V_{\text{S}} =  V_{\text{S}},\\
		\intertext{which agrees with our answer from previous part.}
	\end{align*}
}

\qitem On what order of magnitude of time (nanoseconds, milliseconds, 10's of seconds, etc.) does this circuit settle ($V_{\text{c}}$ is $>95\%$ of its value as $t \to \infty$)?

\sol{

\begin{align*}
\intertext{The time constant $\tau$ of an RC circuit is just $\tau = RC$. For our circuit:}
\tau &= RC = \SI{100}{\ohm} \cdot \SI{10}{\micro\farad} = \SI{0.001}{\second} \\
\intertext{After 3 time constants, the voltage will be ~95\% of its steady state value}
3\tau &= \SI{0.003}{\second} \\
\intertext{The circuit will settle on the order of milliseconds.}
\end{align*}

}

\qitem Give 2 ways to reduce the settling time of the circuit if we are allowed to change one component in the circuit.

\sol{

To reduce settling time, reduce $\tau$. We can achieve this by

\begin{enumerate}
\item Lowering the value of $R$ or
\item Lowering the value of $C$.
\end{enumerate}

Notice how the value of $V_{\text{s}}$ does not change the settling time.

}


\end{enumerate}
