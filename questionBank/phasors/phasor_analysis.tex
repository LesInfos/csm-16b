\qns{Phasor analysis}
\qcontributor{Yen-Sheng Ho}

Any sinusoidal time-varying function $x(t)$,
representing a voltage or a current,
can be expressed in the form
\begin{align}
x(t) = \mathfrak{Re} [Xe^{j\omega t}],
\end{align} 
where $X$ is a time-independent function called the phasor counterpart of $x(t)$.
Thus, $x(t)$ is defined in the time domain,
while its counterpart $X$ is defined in the phasor domain.

The phasor analysis method consists of five steps.
Consider the RC circuit below.

	\begin{center}
		\begin{circuitikz}
			\draw (0,3)
			to[vsourcesin =$v_s$] (0,0)
			(0,3) -- (2,3)
			to[R = $R$] (3,3)
			to[short,i>= \mbox{$i(t)$}] (6,3)
			to[C = $C$, v=$v_C(t)$] (6,0)
			to[short] (0,0);
			
		\end{circuitikz}
	\end{center}

The voltage source is given by
\begin{align}
v_s = 12 \sin(\omega t - \frac{\pi}{4}),
\end{align}
with $\omega = 10^3$ rad/s, $R = \sqrt{3}$ $k\Omega$, and $C = 1$ $\mu F$.

Our goal is to obtain a solution for $i(t)$ with
the sinusoidal voltage source $v_s$.

\begin{enumerate}

\qitem \textbf{Step 1: Adopt cosine references}

All voltages and currents with known sinusoidal functions
should be expressed in the standard cosine format.
Convert $v_s$ into a cosine and write down its phasor representation $V_s$.

Note: To make things clear, the standard form of phasors in 16B will use
cosine references and include the $\frac{1}{2}$ scalar.

\sol{
\begin{align}
v_s(t) = 12 \cos (\omega t -\frac{\pi}{4} -\frac{\pi}{2}) = 12 \cos(\omega t - \frac{3\pi}{4})
\end{align}

The phasor is given by
\begin{align}
V_s = 6 e^{-j\frac{3\pi}{4}}
\end{align}
}

% \ans{
% \begin{align}
% v_s(t) = 12 \cos (\omega t -\frac{\pi}{4} -\frac{\pi}{2}) = 12 \cos(\omega t - \frac{3\pi}{4})
% \end{align}

% The phasor is given by
% \begin{align}
% V_s = 12 e^{-j\frac{3\pi}{4}}
% \end{align}
% }

\qitem \textbf{Step 2: Transform circuits to phasor domain}

The voltage source is represented by its phasor $V_s$.
The current $i(t)$ is related to its phasor counterpart $I$. 
% by
% \begin{align}
% i(t) = \mathfrak{Re}[I e^{j\omega t}].
% \end{align}
What are the phasor representations of $R$ and $C$?

\sol{
\begin{align}
Z_R &= R\\
Z_C &= \frac{1}{j\omega C}
\end{align}
}

% \ans{
% \begin{align}
% Z_R &= R\\
% Z_C &= \frac{1}{j\omega C}
% \end{align}
% }

\qitem \textbf{Step 3: Cast KCL and/or KVL equations in phasor domain}

Use Kirchhoff's laws to write down a loop equation that relates all phasors in Step 2. 

\sol{
\begin{align}
Z_R I + Z_CI &= V_s \\
(R + \frac{1}{j\omega C}) I &= 6 e^{-j\frac{3\pi}{4}}
\end{align}
}

% \ans{
% \begin{align}
% Z_R I + Z_CI &= V_s \\
% (R + \frac{1}{j\omega C}) I &= 12 e^{-j\frac{3\pi}{4}}
% \end{align}
% }


\qitem \textbf{Step 4: Solve for unknown variables}

Solve the equation you derived in Step 3 for $I$ and $V_C$.
What is the polar form of $I$ ($Ae^{i\theta}$, where $A$ is a positive real number) and $V_C$? 

\sol{
\begin{align}
I = \frac{6 e^{-j\frac{3\pi}{4}}}{R + \frac{1}{j\omega C}} = \frac{j6\omega C e^{{-j\frac{3\pi}{4}}}}{1+ j\omega RC}
\end{align}
\begin{align*}
V_C=IZ_C=\frac{j6\omega C e^{{-j\frac{3\pi}{4}}}}{1+ j\omega RC}*\frac{1}{j\omega C}= \frac{6e^{-j\frac{3\pi}{4}}}{1+j\omega RC}
\end{align*}

To derive the polar form,
\begin{align}
I = \frac{j6e^{-j\frac{3\pi}{4}}*10^{-3}}{1+j\sqrt{3}} = \frac{6e^{-j\frac{3\pi}{4}}e^{j\frac{\pi}{2}}*10^{-3}}{2e^{j\frac{\pi}{3}}}
= 3e^{-j\frac{7\pi}{12}} \mbox{ mA}.
\end{align}
\begin{align}
V=\frac{6e^{-j\frac{3\pi}{4}}}{1+j\omega RC} = \frac{6e^{-j\frac{3\pi}{4}}}{1+j\sqrt{3}}=\frac{6e^{-j\frac{3\pi}{4}}}{2e^{j\frac{\pi}{3}}}= 3e^{-j\frac{13\pi}{12}}  \mbox{ V}
\end{align}
}

% \ans{
% \begin{align}
% I = \frac{12 e^{-j\frac{3\pi}{4}}}{R + \frac{1}{j\omega C}} = \frac{j12\omega C e^{{-j\frac{3\pi}{4}}}}{1+ j\omega RC}
% \end{align}
% \begin{align*}
% V_C=IZ_C=\frac{j12\omega C e^{{-j\frac{3\pi}{4}}}}{1+ j\omega RC}*\frac{1}{j\omega C}= \frac{12e^{-j\frac{3\pi}{4}}}{1+j\omega RC}
% \end{align*}

% To derive the polar form,
% \begin{align}
% I = \frac{j12e^{-j\frac{3\pi}{4}}*10^{-3}}{1+j\sqrt{3}} = \frac{12e^{-j\frac{3\pi}{4}}e^{j\frac{\pi}{2}}*10^{-3}}{2e^{j\frac{\pi}{3}}}
% = 6e^{-j\frac{7\pi}{12}} \mbox{ mA}.
% \end{align}
% \begin{align}
% V=\frac{12e^{-j\frac{3\pi}{4}}}{1+j\omega RC} = \frac{12e^{-j\frac{3\pi}{4}}}{1+j\sqrt{3}}=\frac{12e^{-j\frac{3\pi}{4}}}{2e^{j\frac{\pi}{3}}}= 6e^{-j\frac{13\pi}{12}}  \mbox{ V}
% \end{align}
% }

\qitem \textbf{Step 5: Transform solutions back to time domain}

To return to time domain, we apply the fundamental relation between a sinusoidal function and its phasor counterpart.
What is $i(t)$ and $v_C(t)$? What is the phase difference between $i(t)$ and $v_C(t)$? 

\sol{
\begin{align}
i(t) = Ie^{j\omega t} + \overline{I}e^{-j\omega t} = 6 \cos (\omega t - \frac{7\pi}{12})  \mbox{ mA}
\end{align}
\begin{align}
v_C(t)= Ve^{j\omega t} + \overline{V}e^{-j\omega t}= 6 \cos(\omega t -\frac{13\pi}{12}) \mbox{ V}
\end{align}
The phase difference between the two, with respect to $i(t)$ is $-\frac{\pi}{2}$
}

% \ans{
% \begin{align}
% i(t) = \mathfrak{Re}[Ie^{j\omega t}] = \mathfrak{Re} [6e^{-j\frac{7\pi}{12}} e^{j\omega t}] = 6 \cos (\omega t - \frac{7\pi}{12})  \mbox{ mA}
% \end{align}
% \begin{align}
% v_C(t)=\mathfrak{Re}[V_Ce^{j\omega t}]=\mathfrak{Re}[6e^{-j\frac{13\pi}{12}}e^{j\omega t}]=6 \cos(\omega t -\frac{13\pi}{12}) \mbox{ V}
% \end{align}
% The phase difference between the two, with respect to $i(t)$ is $-\frac{\pi}{2}$
% }

\end{enumerate}

It's important to keep in mind that the phasor analysis above only applies to
sinusoidal inputs. Don't make the mistake of doing phasor analysis on DC inputs!