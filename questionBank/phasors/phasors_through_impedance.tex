% {\Large \textbf{Mechanical:}}

\qns{Phasors Through Impedance}

Analyzing circuits with only resistors is easy due to Ohm's Law: $V(t)=i(t) \cdot{} R$.
When adding capacitors or inductors, however, the voltage/current relationship becomes complicated due to derivatives resulting in differential equations.
In fact, for each capacitor/inductor added to a circuit, a higher order differential equation arises, which becomes difficult to solve.\\

But let's look at the $i-v$ relationship across a capacitor when $v(t) = V_{0} \cdot{} cos(\omega{} t+\phi{})$.
Taking a derivative and applying trig identities yields $i(t)=\omega{}C \cdot{} cos(\omega{} t + \phi{} + \frac{\pi{}}{2})$, which is still a cosine wave with frequency $\omega{}$!)\\

In this question, we look at an extension of Ohm's Law for capacitors and inductors through the relationship $\widetilde{V} = \widetilde{I} \cdot{} \widetilde{Z}$.
The quantity Z is defined as the impedance of a circuit component.

\begin{enumerate}

\qitem Let $V(t) = V_{0} \cdot cos(\omega{}t + \phi{})$.
Using Euler's Formula, write V(t) as a sum of two complex exponentials.

\sol{

}

\end{enumerate}

Using these complex exponentials, we define a phasor: $\widetilde{V} = V_{0}e^{j\phi{}}$ and its complex conjugate $\overline{\widetilde{V}} = \frac{V_{0}}{2}e^{-j\phi{}}$ and it follows that $V(t) = \widetilde{V}e^{j\omega{}t} + \overline{\widetilde{V}}e^{-j\omega{}t}$.
In the following parts, we define relationships between $\widetilde{V}$ and $\widetilde{I}$ where $i(t) = \widetilde{I}e^{j\omega{}t} + \overline{\widetilde{I}}e^{-j\omega{}t}$.

\begin{enumerate}[resume]

\qitem $V(t) = V_{0} \cdot{} cos(\omega{}t + \phi{})$, write an expression for $i(t)$ in the form $i(t) = A\widetilde{V}e^{j\omega{}t} + B\overline{\widetilde{V}}e^{j\omega{}t}$ for $A, B \in{} \Complexes{}$.\\ \\
Hint: Remember that $e^{j\frac{\pi{}}{2}}$.

\sol{

}

\qitem From the previous part, try writing the expression $\widetilde{V} = \widetilde{I} \cdot Z_{c}$, the quantity $Z_{c}$ is the impedance of a capacitor.

\sol{

}

\qitem Now let's look at an inductor with current $i(t) = I_{0} \cdot cos(\omega{}t + \phi{})$. Write an expression for $V(t)$, similar to the previous part but in the form $v(t) = A\widetilde{I}e^{j\omega{}t} + B\overline{\widetilde{I}}e^{j\omega{}t}$.

\sol{

}

\qitem What is the impedance of an inductor?

\sol{

}

\qitem (Optional) Show that the impedance of a resistor $Z_{R} = R$.

\sol{

}

\end{enumerate}
