\qns{Phasors Through Impedance}

\meta{Prereqs: Complex Numbers, Complex Exponentials, Euler's Formula}

Analyzing circuits with only resistors is easy due to Ohm's Law: $v(t)=i(t) \cdot R$.
When adding capacitors or inductors however, the voltage/current relationship becomes complicated due to derivatives resulting in differential equations.
In fact, for each capacitor/inductor added to a circuit, a higher order differential equation arises, which becomes difficult to solve.

However, let's look at the $i-v$ relationships across a capacitor when $v(t) = e^{st}$ for some scalar $s.$
\begin{equation}
i(t) = C\ddt{}{t} v(t) = sC e^{st}
\end{equation}
Notice that this is still the same exponential function multiplied by a scaling factor of $sC.$ \vskip 0.5pt
Now if we look at a sinusoidal voltage $v(t) = V_{0} \cos(\omega{} t+\phi{}),$ we can write this voltage as a sum of complex exponentials using Euler's formula.
\begin{equation}
v(t) = V_{0} \cos(\omega{} t+\phi{}) = \frac{V_{0}}{2} e^{j(\omega{} t + \phi{})} + \frac{V_{0}}{2} e^{-j(\omega{} t + \phi{})}
\end{equation}
We can similarly look at the current across a capacitor, use trigonometric identities and see that:
\begin{equation} 
i(t) = \omega C \cdot V_{0} \cos(\omega{} t + \phi{} + \frac{\pi}{2})
\end{equation}
Which again is a cosine wave with the same frequency $\omega.$ \vskip 0pt
Therefore, to relate the current and voltages, we will define a quantity called a phasor:
\begin{equation}
\widetilde{V} = \frac{1}{2} V_{0} e^{j \phi{}} \ \  \text{and} \ \ \widetilde{I} = \frac{1}{2} I_{0} e^{j \phi{}}
\end{equation}
We will use these phasors to create an extension of Ohm's Law for capacitors and inductors through the relationship
\begin{equation}
\widetilde{V} = \widetilde{I} \cdot Z
\end{equation}
The quantity $Z$ is defined as the \textbf{impedance} of a circuit component.

% \begin{enumerate}

% \qitem Let $v(t) = V_{0} \cos(\omega{}t + \phi{})$.
% Using Euler's Formula, write $v(t)$ as a sum of two complex exponentials.

% \sol{
% Euler's formula says that $e^{j \theta} = \cos(\theta) + j \sin(\theta).$ \vskip 1pt
% Since we want to write $\cos(\theta)$ as a sum of exponentials, we want to cancel the $j \sin(\theta)$ term. \vskip 1pt
% This can be done by using $e^{-j \theta} = \cos(-\theta) + j \sin(-\theta) = \cos(\theta) - j \sin(\theta).$ \vskip 1pt
% Adding the two equations, we see that $\cos(\theta) = \frac{1}{2} e^{j \theta} + \frac{1}{2} e^{-j \theta}.$ \vskip 1pt
% Using this information we see that, 
% $$v(t) = \frac{V_{0}}{2} e^{j(\omega{} t + \phi{})} + \frac{V_{0}}{2} e^{-j(\omega{} t + \phi{})}.$$
% }

% \end{enumerate}

% Using these complex exponentials, we define a phasor: $\frac{\widetilde{V}}{2} = V_{0}e^{j\phi{}}$ and its complex conjugate $\overline{\widetilde{V}} = \frac{V_{0}}{2}e^{-j\phi{}}$ and it follows that $v(t) = \widetilde{V}e^{j\omega{}t} + \overline{\widetilde{V}}e^{-j\omega{}t}$. \vskip 1pt
% In the following parts, we define relationships between $\widetilde{V}$ and $\widetilde{I}$ where $i(t) = \widetilde{I}e^{j\omega{}t} + \overline{\widetilde{I}}e^{-j\omega{}t}$.

\begin{enumerate}
\qitem Let $v(t) = V_{0} \cos(\omega{}t + \phi{})$ be the voltage across a capacitor. Derive an expression for $i(t)$ in the form $i(t) = A\widetilde{V}e^{j\omega{}t} + B\overline{\widetilde{V}}e^{-j\omega{}t}$ for $A, B \in{} \mathbb{C}$.\\ 
\textit{Hint: Remember that $e^{j\frac{\pi{}}{2}} = j$.}

\sol{
The capacitor voltage-current relation is: $i(t) = C\ddt{v(t)}{t} = \omega{}C \cdot V_0 \cos(\omega{}t + \phi + \frac{\pi{}}{2}).$ 
\begin{align*} \text{Then, } \, i(t) &= \omega{}C\big(\frac{V_{0}}{2} e^{j\phi}e^{j\omega{}t}e^{j\frac{\pi}{2}} + \frac{V_{0}}{2} e^{-j\phi}e^{-j\omega{}t}e^{-j\frac{\pi}{2}}\big)
= \omega{}C\big(\widetilde{V}e^{j\omega{}t}e^{j\frac{\pi}{2}} + \overline{\widetilde{V}}e^{-j\omega{}t}e^{-j\frac{\pi}{2}}\big) \\
&= \omega{}C\big(\widetilde{V}e^{j\omega{}t} \cdot{} j + \overline{\widetilde{V}}e^{-j\omega{}t} \cdot{} (-j) \big) = j\omega{}C \widetilde{V}e^{j\omega{}t} - j \omega{} C 
\overline{\widetilde{V}}e^{-j\omega{}t} \end{align*}
}
\qitem From the previous part, try writing the expression $\widetilde{V} = \widetilde{I} \cdot Z_{c}$. 
We refer to the quantity $Z_{c}$ as the impedance of a capacitor.

\sol{
Recalling that $i(t) = \widetilde{I}e^{j\omega{}t} + \overline{\widetilde{I}}e^{-j \omega{} t},$ and equating with the previous part, we see that 
$\widetilde{I} = j \omega{} C \cdot \widetilde{V}$ and $\overline{\widetilde{I}} = -j \omega{} C \cdot \overline{\widetilde{V}}.$ 
The reason we can equate the $e^{j \omega{} t}$ and $e^{-j \omega{} t}$ terms is because they are linearly independent functions.
It follows that $Z_c = \frac{\widetilde{V}}{\widetilde{I}} = \frac{1}{j \omega{} C}.$

}

\qitem Now let's look at an inductor with current $i(t) = I_{0} \cos(\omega{}t + \phi{})$.
Write an expression for $v(t)$, similar to the previous part but in the form $v(t) = A\widetilde{I}e^{j\omega{}t} + B\overline{\widetilde{I}}e^{-j\omega{}t},$ for $A, B \in \mathbb{C}.$ \\
\textit{Hint: The identity $cos(x + \frac{\pi}{2}) = -\sin(x)$ may be helpful.}

\sol{
The inductor voltage-current relation is: $v(t) = L\ddt{i(t)}{t} = - \omega{} L \cdot I_{0} \sin (\omega{} t + \phi{}).$ 
Applying the hint, we see that $v(t) = \omega{} L \cdot I_{0} \cos(\omega{} t + \phi{} + \frac{\pi}{2}),$ and reapplying the same steps from part (b), we see that
$v(t) = j \omega{} L \widetilde{I}e^{j\omega{}t} - j \omega{} L \overline{\widetilde{V}}e^{-j\omega{}t}$
}

\qitem What is the impedance of an inductor?

\sol{
Equating $v(t)$ with its phasor form, we see that $\widetilde{V} = \widetilde{I} \cdot j \omega{} L.$ Therefore, $Z_L = j \omega L.$
}

\qitem \textbf{(Optional):} Show that the impedance of a resistor $Z_{R} = R$.

\sol{
We can start with a sinusoid $v(t) = V_{0} \cos(\omega{}t + \phi),$ it follows from Ohm's Law that $i(t) = \frac{V_{0}}{R} \cos(\omega{}t + \phi{}) = \frac{1}{R} (\frac{V_{0}}{2} e^{j\phi}e^{j\omega{}t} + \frac{V_{0}}{2} e^{-j\phi}e^{-j\omega{}t}\big) = \frac{1}{R} \big(\widetilde{V}e^{j\omega{}t} + \overline{\widetilde{V}}e^{-j\omega{}t}\big) = 
\widetilde{I}e^{j\omega{}t} - \overline{\widetilde{I}}e^{-j\omega{}t}.$ Equating both sides, we see that $\widetilde{V} = \widetilde{I} \cdot R$ so $Z_{R} = R.$
}

\end{enumerate}
