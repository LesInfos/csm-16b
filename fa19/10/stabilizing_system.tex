\qns{Stabilizing a System}
% \qcontributor{John Maidens}

\meta {
  This time the system is no longer in Controllable Canonical Form, and we must now convert the system to a basis in which it is in CCF, and then convert back to use the $K$ values in the standard basis.
}

Consider the following linear discrete time system

\begin{align*}
\vec{x}(t+1) &= \begin{bmatrix}
-1 &  1 &  0 \\
 1 & -2 &  1 \\
 0 &  1 & -1
\end{bmatrix}
\vec{x}(t) +
\begin{bmatrix}
1 \\ 0 \\ 0
\end{bmatrix} u(t)
\end{align*}

\begin{enumerate}

\qitem Is this system controllable?

\sol{We calculate
$$C = [B,AB,A^{2}B] =
\begin{bmatrix}
1 & -1 & 2 \\
0 & 1 & -3 \\
0 & 0 & 1
\end{bmatrix}$$
Observe that $C$ matrix is full rank and hence our system is controllable.
}

\qitem Is the linear discrete time system stable?

\sol{We have to calculate the eigenvalues of matrix $A$. Thus,
\begin{align*}
0 = \det(A - \lambda I) = \lambda^3 + 4\lambda^2 + 3\lambda = \lambda(\lambda + 3)(\lambda + 1)
\end{align*}
Since the eigenvalue at -3 is outside the unit circle, this system is unstable.
}


\qitem Bring the system to the controllable canonical form

\begin{align*}
\vec{z}(t+1) &= \begin{bmatrix}
0 & 1 & 0 \\
0 & 0 & 1 \\
a_1 & a_2 & a_3
\end{bmatrix}
\vec{z}(t) +
\begin{bmatrix}
0 \\ 0 \\ 1
\end{bmatrix} u(t)
\end{align*}

using transformation $\vec{z}(t) = T\vec{x}(t)$

\meta {
  You should probably tell your students that the inverse of 
  $\begin{bmatrix} 1 & -1 & 2\\ 0 & 1 & -3 \\ 0 & 0 & 1 \end{bmatrix}$ is 
  $\begin{bmatrix} 1 & 1 & 1 \\ 0 & 1 & 3 \\ 0 & 0 & 1 \end{bmatrix}.$
}

\sol{
The characteristic polynomial for this system is $\lambda^3 + 4\lambda^2 + 3\lambda$. So putting this system in control canonical form gives
\begin{align*}
\vec{z}(t+1) &= \begin{bmatrix}
0 & 1 & 0 \\
0 & 0 & 1 \\
0 & -3 & -4
\end{bmatrix}
\vec{z}(t) +
\begin{bmatrix}
0 \\ 0 \\ 1
\end{bmatrix} u(t)
\end{align*}
To compute the transformation $T$, we need the matrices
\[
\begin{split}
C &= \begin{bmatrix} B & AB & \dots & A^{n-1}B \end{bmatrix} = \begin{bmatrix} 1 & -1 & 2\\ 0 & 1 & -3 \\ 0 & 0 & 1 \end{bmatrix}
 \\
\tilde{C} &= \begin{bmatrix}  \tilde B &  \tilde A \tilde B & \dots & \tilde A^{n-1} \tilde B \end{bmatrix}
= \begin{bmatrix} 0 & 0 & 1\\ 0 & 1 & -4 \\ 1 & -4 & 13 \end{bmatrix}
\end{split}
\]
Then $T$ is given by
\[
T = \tilde{C} C^{-1} =
\begin{bmatrix} 0 & 0 & 1\\ 0 & 1 & -4 \\ 1 & -4 & 13 \end{bmatrix}
\begin{bmatrix} 1 & 1 & 1 \\ 0 & 1 & 3 \\ 0 & 0 & 1 \end{bmatrix}
=
\begin{bmatrix} 0 & 0 & 1 \\ 0 & 1 & -1 \\ 1 & -3 & 2 \end{bmatrix}
\].
}

\qitem Using state feedback $u(t) =  \widetilde{K} \vec{z}(t) = \begin{bmatrix}
\widetilde{k}_1 & \widetilde{k}_2 & \widetilde{k}_3
\end{bmatrix} \vec{z}(t)$ place the eigenvalues at $0,1/2,-1/2$. 
What is the controller $K$ in standard basis coordinates?

\meta {
  Remember that $\widetilde{K}$ is the matrix in which we can pick our $k$ values easily but it is in the wrong basis, therefore we must convert it back into the standard basis if we want the feedback controller $K.$ 
  The relation $K = \widetilde{K} T$ can be derived by observing the following:
  \begin{align*}
  \vec{x}(t + 1) &= (A + BK) \vec{x}(t) \\
  T^{-1} \vec{z}(t + 1) &= (A + BK) T^{-1} \vec{x}(t) \\
  \vec{z}(t + 1) &= T(A + BK) T^{-1} \vec{x}(t) = TAT^{-1} + TBKT^{-1} \vec{x}(t)
  \end{align*}
  Since $\widetilde{A} = TAT^{-1}$ and $\widetilde{B} = TB, \widetilde{K}$ will be $KT^{-1}.$
}

\sol{
The closed loop system in $z$ coordinates is given by
\[
\widetilde{A} + \widetilde{B} \widetilde{K} = \begin{bmatrix} 0 & 1 & 0 \\ 0 & 0 & 1 \\ \widetilde{k}_1 & \widetilde{k}_2 -3 & \widetilde{k}_3 - 4 \end{bmatrix}
\]
which has characteristic polynomial $\lambda^3 + (4- \widetilde{k}_3) \lambda^2 + (3- \widetilde{k}_2) \lambda - \widetilde{k}_1$. To place the eigenvalues at 0, 1/2, -1/2, the desired characteristic polynomial is $\lambda(\lambda - 1/2)(\lambda + 1/2) = \lambda^3 - 1/4 \ \lambda$. So we should choose $\widetilde{k}_1 = 0, \widetilde{k}_2 = 13/4, \widetilde{k}_3 = 4$.
If we want the feedback controller in terms of $x(t)$, we write $u(t) = K \vec{x}(t)$ where
\[
K = \widetilde{K} T = \begin{bmatrix} 0 & 13/4 & 4 \end{bmatrix} \begin{bmatrix} 0 & 0 & 1 \\ 0 & 1 & -1 \\ 1 & -3 & 2 \end{bmatrix} =
\begin{bmatrix} 4 &  -35/4  &  19/4 \end{bmatrix}.
\]
}

\end{enumerate}
