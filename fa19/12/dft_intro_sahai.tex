\qns{Discrete Fourier Transform Introduction}

% Sketch (from content channel):
% Intro about sampling points from an AC signal and wanting to extract the original signal (frequencies, magnitudes, phase shifts).
% Define the DFT matrix and how it can be applied to signals (what you get out when you input a signal of a certain frequency).
% (a) Given N = 5, write out the kth DFT basis vector/ column of the DFT matrix.
% (b) Write out the DFT matrix for N = 5
% (c) Given a sinusoid (constant term + something with frequency (2 pi /5) * 2), write out your x vector
% (d) Apply the DFT to get your frequency domain (X) vector.
% (e) How do the elements of the X vector correspond to your original signal?
% (d) What is the highest frequency you can get out of the DFT for N = 5?

Let's say we have a real-world signal that is a function of time
$x(t)$. By recording its values at regular intervals, we can represent
it as a vector of discrete samples $\vec{x}$, of length $n$.   
\begin{align}
\vec{x}  = \begin{bmatrix} x[0] \\ x[1] \\ \vdots \\ x[n-1] \end{bmatrix}
\end{align}

In this section, we will explore the DFT and its interpretation through basis vectors.
The $k$th DFT basis vector will be viewed as
$\frac{\vec{r}^k}{\sqrt{N}}$ and represented as an $N$-length column by
evaluating the expression with $t$ being the $N$th roots of unity.
This alterative interpretation allows us to think about interpolation
more easily.

\meta {
  $\vec{r}$ is a vector containing the $N$ roots of unity. For example, if $N = 4,$ then $\vec{r} = 
  \begin{bmatrix}
  1 & e^{j \frac{\pi}{2}} & e^{j \frac{\pi}{2} \cdot 2} & e^{j \frac{\pi}{2} \cdot 3}
  \end{bmatrix}^{T}$
}

% that we want to transmit $\vec{s}$ % = \begin{bmatrix}s_1 & \cdots & s_n$.
We will let $\vec{X} = \begin{bmatrix} X[0] & \ldots & X[N-1] \end{bmatrix}^T$
as the signal $\vec{x}$ represented in the frequency domain, that is
\begin{align}
\vec{X} = U^{-1} \vec{x} = U^* \vec{x}
\end{align}
where $U$ is a matrix of the DFT basis vectors ($\omega =
e^{j\frac{2\pi}{n}}$).
\begin{align}
U =
\begin{bmatrix}
| & & | \\
\vec{u}_0 & \cdots & \vec{u}_{N-1} \\
| & & |
\end{bmatrix}
=
 \frac{1}{\sqrt{N}}
\begin{bmatrix}
1 & 1 & 1 & \cdots  & 1 \\
1 & \omega & \omega^2 & \cdots & \omega^{N-1} \\
1 & \omega^2 & \omega^4  & \cdots &  \omega^{2(N-1)}\\
\vdots & \vdots & \vdots & & \vdots \\
1 & \omega^{N-1} & \omega^{2(N-1)}  & \cdots  & \omega^{(N-1)(N-1)}
\end{bmatrix}
\end{align}

Alternatively, we have that $\vec{x} = U \vec{X}$ or more explicitly
\begin{align}
\vec{x} =  X[0] \vec{u}_0 + \cdots + X[N-1]\vec{u}_{N-1}
\end{align}

In other words, $\vec{x}$ is a linear combination of the complex exponentials $\vec{u}_i$ with coefficients $X[i]$.


Let $r_k = \omega^{k}$ be the $j$th $N$th-root-of-unity. The $N$ points
$r_{0}, r_{1}, \ldots, r_{N-1}$ are the $N$th-roots-of-unity.

Recall that $\omega_N$ is the basis $N^{th}$ root of unity, defined as $\omega_N = e^{j\frac{2\pi}{N}}$.

\vspace{0.5em}
When we apply the DFT to a vector of sample points, the elements of the resulting $\vec{X}$ correspond to the frequencies present in the signal. 
The first element $X[0]$ is $\sqrt{N}$ times whatever constant/DC term is present in the signal. 
For every wave of the form $A\cos(\frac{2\pi}{N}(kt) + \phi)$, $X[k]$ is $\frac{A \sqrt{N}}{2}e^{j\phi}$, and $X[N - k]$ is the complex conjugate of $X[k]$
or $\frac{A \sqrt{N}}{2}e^{-j\phi}$. 

\ws {
  \newpage
}

\begin{enumerate}

\qitem
Given $N = 4$, { \bf write out the $k^{th}$ DFT basis vector. }

{\em HINT:
The columns of the DFT matrix are its basis vectors.
}

\sol{
Following the hint, the $k^{th}$ DFT basis vector is the $k^{th}$ column of the DFT matrix. For $N = 4$ and an arbitrary $k$,
\begin{align*}
\vec{u_k} = \frac{1}{2}
\begin{bmatrix}
1 \\
e^{jk\frac{2\pi}{4}} \\
e^{2jk\frac{2\pi}{4}} \\
e^{3jk\frac{2\pi}{4}} \\
\end{bmatrix}
= \frac{1}{2}
\begin{bmatrix}
1 \\
e^{\frac{2{\pi}jk}{4}} \\
e^{\frac{4{\pi}jk}{4}} \\
e^{\frac{6{\pi}jk}{4}} \\
\end{bmatrix} 
\end{align*}
}

\qitem 
{ \bf Write out the DFT matrix for $N = 4$. }

{\em HINT:
Use the general formula from the previous part to fill in the full matrix.
}

\sol{
Using the formula above, we get:
\begin{align*}
U = \frac{1}{2}
\begin{bmatrix}
1 & 1 & 1 & 1 \\
1 & e^{j\frac{2\pi}{4}} & e^{2j\frac{2\pi}{4}} & e^{3j\frac{2\pi}{4}} \\
1 & e^{2j\frac{2\pi}{4}} & e^{4j\frac{2\pi}{4}} & e^{6j\frac{2\pi}{4}} \\
1 & e^{3j\frac{2\pi}{4}} & e^{6j\frac{2\pi}{4}} & e^{9j\frac{2\pi}{4}} \\
\end{bmatrix} = \frac{1}{2}
\begin{bmatrix}
  1 & 1 & 1 & 1 \\
  1 & j & -1 & 1 \\
  1 & -1 & 1 & -1 \\
  1 & -j & -1 & j
  \end{bmatrix}
\end{align*}
}

\qitem Given the time-domain signal $\vec{x} = \begin{bmatrix} 1 & 0 & 0 & 0 \end{bmatrix}^{T},$ what is the frequency representation of $\vec{x}?$

\sol {
  We can compute the frequency representation of $X$ by multiplying by the matrix $U^{-1}$ since $X = U^{-1} \vec{x}.$
  For $N = 4,$ the DFT matrix will be:
  \[ 
  U = \frac{1}{2} \begin{bmatrix}
  1 & 1 & 1 & 1 \\
  1 & j & -1 & -j \\
  1 & -1 & 1 & -1 \\
  1 & -j & -1 & j
  \end{bmatrix}
  \]
  Since $U$ is an orthonormal matrix, its inverse will be its conjugate transpose, so we see that 
  \[ 
  U^{-1} = U^{*} = \frac{1}{2} \begin{bmatrix}
  1 & 1 & 1 & 1 \\
  1 & -j & -1 & j \\
  1 & -1 & 1 & -1 \\
  1 & j & -1 & -j
  \end{bmatrix}
  \]
  Therefore we can perform the matrix multiplication and realize that 
  \[
  X = \frac{1}{2} \begin{bmatrix}
  1 & 1 & 1 & 1 \\
  1 & -j & -1 & j \\
  1 & -1 & 1 & -1 \\
  1 & j & -1 & -j
  \end{bmatrix} \begin{bmatrix} 1 \\ 0 \\ 0 \\ 0 \end{bmatrix} = 
  \begin{bmatrix} \frac{1}{2} \\ \frac{1}{2} \\ \frac{1}{2} \\ \frac{1}{2} \end{bmatrix}
  \]
}

% X_1 = [2 2 2 2]

\qitem Given the frequency-domain signal $\vec{X} = \begin{bmatrix} 1 & 0 & 0 & 0 \end{bmatrix}^{T},$ what is the time representation of $\vec{X}?$

\sol {
  This time we are given a frequency domain representation $\vec{X}$ and want to find the time domain representation $\vec{x}.$
  We know that $\vec{x} = U \vec{X},$ and can do the matrix multiplication to find $\vec{x},$ but we can also notice that our signal $\vec{X}$ represents coordinates of the vector $\vec{x}$ in the DFT-basis. 

  Therefore $\vec{x} = 1 \cdot \vec{u}_{0} + 0 \cdot \vec{u}_{1} + 0 \cdot \vec{u}_{2} + 0 \cdot \vec{u}_{3}$ and we see that $\vec{x} = \begin{bmatrix} \frac{1}{2} \\ \frac{1}{2} \\ \frac{1}{2} \\ \frac{1}{2} \end{bmatrix}.$
}

% x_2 = [1/2 1/2 1/2 1/2]

\qitem Given the time-domain signal $\vec{x} = \begin{bmatrix} 2 & 1 & 1 & 1 \end{bmatrix}^{T},$ what is the frequency representation of $\vec{x}?$ \textit{You may want to use the results from the two previous parts.}

\sol {
  We can get the frequency representation $\vec{X}$ by multiplying by $U^{-1} = U^{*}$ but we can also use the results from the two previous parts making the realization that
  \[ 
    \vec{x} = \begin{bmatrix} 1 \\ 0 \\ 0 \\ 0 \end{bmatrix} + 2 \begin{bmatrix} \frac{1}{2} \\ \frac{1}{2} \\ \frac{1}{2} \\ \frac{1}{2} \end{bmatrix}
  \]
  We can multiply by $U^{*}$ or use the fact that the DFT is linear to say that
  \[
    \vec{x} = U^{*} \begin{bmatrix} 1 \\ 0 \\ 0 \\ 0 \end{bmatrix} + 2 U^{*} \begin{bmatrix} \frac{1}{2} \\ \frac{1}{2} \\ \frac{1}{2} \\ \frac{1}{2} \end{bmatrix} 
  \]
  We already know the individual DFTs of both terms and can conclude by saying
  \[ 
    \vec{x} = \begin{bmatrix} 2 \\ 2 \\ 2 \\ 2 \end{bmatrix} + 2 \begin{bmatrix} 1 \\ 0 \\ 0 \\ 0 \end{bmatrix} = 
    \begin{bmatrix} 4 \\ 2 \\ 2 \\ 2 \end{bmatrix}
  \]
}
% x = x_1 + 2 * x_2

% (c) Given a sinusoid (constant term + something with frequency (2 pi /5) * 2), write out your x vector


\end{enumerate}


