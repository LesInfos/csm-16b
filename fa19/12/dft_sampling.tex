\qns{DFT Sampling}

To motivate the DFT, imagine a scenario where you would like to analyze an important signal
(for example, a radio transmission from someone sending a message to you).
First, you periodically sample the incoming signal, collecting a total of $N$
points. Then, you put all of these samples into a vector $\vec{x}$. As we will show in this
problem, applying the DFT to this vector $\vec{x}$ will allow you to extract important information such as
the different frequencies that are present in the signal. In other words, applying the DFT to $N$ samples
of a time-domain signal gives us its frequency-domain representation, showing us how much of each frequency is in our signal.


% Add visuals for next sem
\begin{enumerate}

  \qitem

Consider the following signal:

\begin{align*}
x(t) = \cos(\frac{4\pi}{5}t + \frac{\pi}{2})
\end{align*}

Let our sampling frequency be 1 Hz. Starting at t = 0, {\bf sample $x$ at $N = 5$ points to construct $\vec{x}$. } \\
Feel free to leave your answer in terms of the cosine function.

{\em HINT:
We can get samples of a time-domain signal by evaluating it at different points in time. \\
}

\sol{
    \begin{align*}
        \vec{x} = 
        \begin{bmatrix}
        f(0) \\ f(1) \\ f(2) \\ f(3) \\ f(4)
        \end{bmatrix} = 
        \begin{bmatrix}
        \cos(\frac{\pi}{2}) \\ 
        \cos(\frac{13\pi}{10}) \\
        \cos(\frac{21\pi}{10}) \\
        \cos(\frac{29\pi}{10}) \\
        \cos(\frac{37\pi}{10})
        \end{bmatrix}
    \end{align*}
}

\qitem
{\bf Apply the DFT to find $\vec{X}$}, your sampled signal in the frequency domain.

\sol{
    \begin{align*}
        \vec{X} =
        \begin{bmatrix}
        0 && 0 && \frac{\sqrt{5}}{2}e^{\frac{\pi}{2}j} && \frac{\sqrt{5}}{2}e^{-\frac{\pi}{2}j} && 0
        \end{bmatrix}^T
    \end{align*}
}

\qitem
How do the elements of $\vec{X}$ correspond to your original signal?

\sol{
    The element at index $2$ is $N$ times the phasor representation of our signal with frequency $\frac{4\pi}{5}$, 
    and the element at index $5 - 2 = 3$ is $N$ times the complex conjugate of that phasor representation.
}

\qitem
Consider the following signals: 
\begin{align*}
x_{1}(t) = 3\cos(\frac{2\pi}{5}t)
\end{align*}
\begin{align*}
x_{2}(t) = 3\cos(\frac{8\pi}{5}t)
\end{align*}

What would you expect the frequency representation of each of these signals to be? 
Then, try taking $5$ samples of both signals, and applying the DFT to the sampled signals. \\
If it is not what you expect, think about why. What could you do to fix this?

\meta{
    The motivation behind this part is to get students started thinking about aliasing. 
    They should notice that they get the exact same $\vec{x}$ for the two signals, even though the signals have different frequencies.
    Try to guide them towards the realization that we need to sample faster to pick up higher freuencies with the DFT.
}

\sol{
    For both signals, 
    \begin{align*}
        \vec{x} = 
        \begin{bmatrix}
        \cos(\frac{\pi}{2}) \\ 
        \cos(\frac{9\pi}{10}) \\
        \cos(\frac{13\pi}{10}) \\
        \cos(\frac{17\pi}{10}) \\
        \cos(\frac{21\pi}{10})
        \end{bmatrix}
    \end{align*}
    \begin{align*}
        \vec{X} =
        \begin{bmatrix}
        0 && \frac{3 \sqrt{5}}{2} && 0 && 0 && \frac{3 \sqrt{5}}{2}
        \end{bmatrix}^T
    \end{align*}

    We get the same exact frequency domain representation for two different signals because we end up sampling the same points for the two signals. \\
    In order to fix this, we need to sample faster and sample more points.
}

\end{enumerate}