% Author: Taejin Hwang
% Email: taejin@berkeley.edu

\qns{Transient analysis}

We have taken a look at how to take a \textbf{system} of differential equations, convert it the following \textbf{vector} differential equation.

\begin{equation}
\ddt{}{t} \vec{x}(t) = A \vec{x}(t) + \vec{b}(t)
\end{equation}

With initial condition $\vec{x}(0) = \begin{bmatrix} x_1(0) \\ x_2(0) \end{bmatrix}$

We made the observation that we could convert our differential equation from standard coordinates into coordinates using a basis entirely made up of eigenvectors, in which the linear operator of differentiation became a diagonal matrix, $D,$ where the diagonal entires are the eigenvalues of $A.$

\begin{equation}
\ddt{}{t} \vec{z}(t) = D \vec{z}(t) + \widetilde{\vec{b}}(t)
\end{equation}

With initial condition $\vec{z}(0) = \begin{bmatrix} z_1(0) \\ z_2(0) \end{bmatrix}$

In this problem, we will analyze the different types of transient responses that a system can have, based on the eigenvalues of the differentiation operator.

Consider the following RLC Circuit with the capacitor uncharged for $t < 0.$

\qns{RLC circuit \#1}
\qcontributor{Varun Mishra}

In this question, we will take a look at an electrical systems described by second order differential equations and analyze it using the phasor domain. Consider the circuit below where $R=3 \text{k}\Omega$, $L=1\mbox{mH}$, $C=100$nF, and $V_{\text{s}}=5\cos(1000t+\frac{\pi}{4})$:

\begin{center}
		\begin{circuitikz}[scale=0.8]
			\draw (0,4) 
			to [sV, l= $V_s$] (0,0)
			(0,4)
			to [R = $R$,v=$V_R$] (4,4)
			to [L = $L$,v=$V_L$,i=$i(t)$] (8,4)
			to [short] (10,4)
			to [C = $C$,v=$V_{\text{out}}$] (10,0)	
			to [short] (0,0);
		\end{circuitikz}
	\end{center}
\begin{enumerate}
	
    \qitem What are the impedances of the resistor, inductor and capacitor, $Z_R$, $Z_L$, and $Z_C$?
    
\sol{
\begin{align*}
\intertext{The impedance of a resistor is the same as its resistance}
Z_R &= 3000\Omega 
\intertext{We can find the frequency of the circuit by looking at $V_{\text{s}}$. The form of a cosine function is $A\cos(\omega t+\phi )$ where $A$ is amplitude, $\omega$ is frequency, and $\phi$ is phase. In this case, the frequency is $1000 \frac{\text{rad}}{\text{s}}$}
Z_L &= j\omega L = j1000*10^{-3}= j1 \Omega \\
Z_C &= \frac{1}{j\omega C} = \frac{1}{j1000*10^{-7}} =-j10^4 \Omega
\end{align*}

}  
\qitem Solve for $\widetilde{V}_{\text{out}}$ in phasor form.

\sol{
\begin{align*}
\intertext{converting $V_{\text{s}}$ into phasor form, we have}
\widetilde{V}_{\text{s}} &= \frac{1}{2} |A|e^{j\phi} = \frac{5}{2}e^{j\frac{\pi}{4}} 
\intertext{The circuit given is a voltage divider. Since impedances act like resistors, we can use the same equation as the resistive voltage divider.}
\widetilde{V}_{\text{out}} = \widetilde{V}_{\text{s}}\frac{Z_C}{Z_R+Z_L+Z_C}&= \widetilde{V}_{\text{s}}*\big|\frac{Z_C}{Z_R+Z_L+Z_C}\big|e^{j*\angle\big(\frac{Z_C}{Z_R+Z_L+Z_C}\big)} \tag{1} \\ 
\intertext{We can solve for the magnitude and angle of the divider using}
\big|\frac{Z_C}{Z_R+Z_L+Z_C}\big|&=\frac{|Z_C|}{|Z_R+Z_L+Z_C|}=\frac{10^4}{\sqrt{3000^2+(1-10^4)^2}} = 0.958 \\
\angle\big(\frac{Z_C}{Z_R+Z_L+Z_C}\big)&= \angle(Z_C)-\angle(Z_R+Z_L+Z_C)=\text{atan2}\Big(\frac{\mathfrak{Im}(Z_C)}{\mathfrak{Re}(Z_C)}\Big)-\text{atan2}\Big(\frac{\mathfrak{Im}(Z_R+Z_L+Z_C)}{\mathfrak{Re}(Z_R+Z_L+Z_C)}\Big) \\
\angle\big(\frac{Z_C}{Z_R+Z_L+Z_C}\big)&=-0.2915 \text{ rad}
\intertext{Plugging back into (1)}
\widetilde{V}_{\text{out}}&=\frac{5}{2}e^{j\frac{\pi}{4}}*0.958e^{-j0.2915}= 2.395 e^{j0.494}
\end{align*}
}
    
   
 \qitem Solve for the transfer function $H(\omega)=\frac{\widetilde{V}_{\text{out}}}{\widetilde{V}_{\text{s}}}$
 
  Leave your answer in terms of $R$, $L$, $C$, and $\omega$.
 
\sol{
\begin{align*}
\intertext{Looking back at equation (1)}
\widetilde{V}_{\text{out}} &= \widetilde{V}_{\text{s}}\frac{Z_C}{Z_R+Z_L+Z_C}
\intertext{Rearranging we get}
H(\omega)&=\frac{\widetilde{V}_{\text{out}}}{\widetilde{V}_{\text{s}}}= \frac{Z_C}{Z_R+Z_L+Z_C}=\frac{\frac{1}{j\omega C}}{R+j\omega L+\frac{1}{j\omega C}} \\
H(\omega) &= \frac{1}{LC(j\omega)^2+jRC\omega+1}
\end{align*}
}
  
\qitem Solve for the current $i(t)$

    \sol{
    \begin{align*}
    \widetilde{i}&=\frac{\widetilde{V}_{\text{s}}}{Z_R+Z_L+Z_C} = \frac{|\widetilde{V}_{\text{s}}|}{|Z_R+Z_L+Z_C|}e^{j\big(\angle \widetilde{V}_{\text{s}}- \angle (Z_R+Z_L+Z_C)\big)} = 2.395*10^{-4}e^{j2.0647}
    \intertext{Going back to the time domain:}
    i(t)&=4.790*10^{-4}\cos(1000t+2.0647) \text{ A}
    \end{align*}
    }

\qitem What is $V_{\text{out}}$ in the time domain?

\sol{
\begin{align*}
% V_{\text{out}}(t)&= \mathfrak{Re}(\widetilde{V}_{\text{out}}e^{j\omega t})
% \intertext{Using Euler's formula, we can say}
% \widetilde{V}_{\text{out}}e^{j\omega t} &= |\widetilde{V}_{\text{out}}|\big(\cos(\omega t+\angle \widetilde{V}_{\text{out}})+j\sin(\omega t+\angle \widetilde{V}_{\text{out}})\big) 
% \intertext{Taking the real part, we get}
% \mathfrak{Re}(\widetilde{V}_{\text{out}}e^{j\omega t}) &= |\widetilde{V}_{\text{out}}|\cos(\omega t+\angle \widetilde{V}_{\text{out}}) \\
V_{\text{out}}(t)&= 4.79\cos(1000t+0.494) \text{ V}
\end{align*}
}

\end{enumerate}



\qitem \textbf{Step 1: Define state variables}

\meta{
  The same process can be done by defining $x_{1}(t) = V_{out}(t)$ and $x_{2}(t) = \ddt{V_{out}(t)}{t}.$ Make sure students are familiar with how to handle both cases. We choose $i(t)$ here since $i(t) = C \ddt{V_{out}(t)}{t}.$
}

When setting up a system of differential equations, it is important to always start by definining a state variable to create a vector differential equation. For the sake of this question, we will define our state variable as:

\begin{equation}
\vec{x}(t) =  \begin{bmatrix} x_{1}(t) \\ x_{2}(t) \end{bmatrix} = \begin{bmatrix} V_\text{c}(t) \\ i(t) \end{bmatrix}
\end{equation}

% \ans{
% \begin{align}
% v_s(t) = 12 \cos (\omega t -\frac{\pi}{4} -\frac{\pi}{2}) = 12 \cos(\omega t - \frac{3\pi}{4})
% \end{align}

% The phasor is given by
% \begin{align}
% V_s = 12 e^{-j\frac{3\pi}{4}}
% \end{align}
% }

\qitem \textbf{Step 2: Write out a system of differential equations}

The next step is to set up a system of differential equations using the individual state variables that you defined in the previous part.

How can you write out a system of differential equations in terms of your state variables $x_{1}(t) = V_\text{c}(t)?$ and $x_{2}(t) = i(t)$

\sol {
	The voltage-current relationship of a capacitor tells us that
	$$i(t) = C \ddt{}{t} V_\text{c}(t)$$
	This can equivalently be written as:
	$$\ddt{}{t} x_{1}(t) = \frac{1}{C} x_{2}(t)$$
	The voltage-current relationship of an inductor tells us that
	$$V_{L}(t) = L \ddt{}{t} i(t)$$
	So we see that
	$$\ddt{}{t} i(t) = \frac{1}{L} V_{L}(t)$$
	However, $V_{L}(t)$ is not a state variable, meaning we must find a way to express $V_{L}(t)$ in terms of our state variables. Therefore by using KVL:
	$$V_{s} = V_{R}(t) + V_{\text{c}}(t) + V_{L}(t)$$
	Which implies that
	$$V_{L}(t) = V_{s} - V_{R}(t) - V_{\text{c}}(t) = V_{s} - i(t) \cdot R - V_{\text{c}}(t)$$
	Therefore:
	$$\ddt{}{t} i(t) = \frac{1}{L} \big( V_{s} - i(t) \cdot R - V_{\text{c}}(t) \big)$$
	As state variables:
	$$\ddt{}{t} x_{2}(t) = -\frac{1}{L} x_{1}(t) - \frac{R}{L} x_{2}(t) + \frac{1}{L} V_{s}$$

  The initial conditions must be:
  \begin{gather*}
  x_{1}(0) = 0 \\
  x_{2}(0) = 0
  \end{gather*}
  Since the capacitor starts of uncharged, and the current across the capacitor cannot change instantaneously.
}


% The voltage source is represented by its phasor $V_s$.
% The current $i(t)$ is related to its phasor counterpart $I$. 
% % by
% % \begin{align}
% % i(t) = \mathfrak{Re}[I e^{j\omega t}].
% % \end{align}
% What are the phasor representations of $R$ and $C$?

% \sol{
% \begin{align}
% Z_R &= R\\
% Z_C &= \frac{1}{j\omega C}
% \end{align}
% }

% % \ans{
% % \begin{align}
% % Z_R &= R\\
% % Z_C &= \frac{1}{j\omega C}
% % \end{align}
% % }

\qitem \textbf{Step 3: Turn this system of differential equations into a vector differential equation}

Given the system of differential equations you wrote in the previous part, the next step is to create a vector differential equation of the form:

\begin{equation}
\ddt{}{t} \vec{x}(t) = A \vec{x}(t) + \vec{b}
\end{equation}

\sol {
	Taking the derivative of $\vec{x}(t)$ we get:
	$$\ddt{}{t} \vec{x}(t) = \begin{bmatrix} \ddt{}{t} x_{1}(t) \\ \ddt{}{t} x_{2}(t) \end{bmatrix} = 
	\begin{bmatrix} \frac{1}{C} x_{2}(t) \\ \frac{1}{L} x_{1}(t) - \frac{R}{L} x_{2}(t) + \frac{V_s}{L} \end{bmatrix}$$
	We can write this in matrix-vector form:
	$$\ddt{}{t} \vec{x}(t) = \begin{bmatrix} 0 & \frac{1}{C} \\ -\frac{1}{L} & - \frac{R}{L} \end{bmatrix} 
	\begin{bmatrix} x_{1}(t) \\  x_{2}(t) \end{bmatrix} + \begin{bmatrix} 0 \\ \frac{V_s}{L} \end{bmatrix} $$
	So we can see that:
	\begin{gather*}
	A = \begin{bmatrix} 0 & \frac{1}{C} \\ -\frac{1}{L} & - \frac{R}{L} \end{bmatrix} 
	\text{ and  }
	\vec{b} = \begin{bmatrix} 0 \\ \frac{V_s}{L} \end{bmatrix}
	\end{gather*}
  With initial condition $\vec{x}(0) = \begin{bmatrix} 0 \\ 0 \end{bmatrix}$
}

\qitem \textbf{Step 4: Find the coordinate transformation to the eigenbasis}

The next step is to find a change of variables to the a basis made up of eigenvectors.
Remember that this is equivalent to finding the eigenvectors and eigenvalues of the matrix $A.$

\sol{
	To find the eigenvalues of $A,$ we try to solve for $\lambda$ such that $\text{det} (A - \lambda I) = 0.$
	$$\text{det} (A - \lambda I) = \begin{vmatrix} - \lambda & \frac{1}{C} \\ -\frac{1}{L} & - \frac{R}{L} - \lambda \end{vmatrix} 
	= (- \lambda) (-\frac{R}{L} - \lambda) + \frac{1}{LC} = \lambda^{2} + \frac{R}{L} \lambda + \frac{1}{LC} $$
	We can solve for $\lambda$ by using the quadratic formula:
	$$\lambda = -\frac{R}{2L} \pm \sqrt{\big(\frac{R}{L} \big)^2 - \frac{4}{LC}}$$
	To find the eigenvectors of $A,$ we look at the null-space of $A - \lambda I.$
	$$A - \lambda_{1} I = \begin{bmatrix} - \lambda_{1} & \frac{1}{C} \\ -\frac{1}{L} & - \frac{R}{L} - \lambda_{1} \end{bmatrix}$$
	A basis for the null-space of $A - \lambda_{1} I$ is: 
	$$\vec{v}_{1} = \begin{bmatrix} 1 \\ \lambda_1 C \end{bmatrix}$$
	A basis for the null-space of $A - \lambda_{2} I$ can be similarly calculated as:
	$$\vec{v}_{2} = \begin{bmatrix} 1 \\ \lambda_2 C \end{bmatrix}$$
}

% \ans{
% \begin{align}
% I = \frac{12 e^{-j\frac{3\pi}{4}}}{R + \frac{1}{j\omega C}} = \frac{j12\omega C e^{{-j\frac{3\pi}{4}}}}{1+ j\omega RC}
% \end{align}
% \begin{align*}
% V_C=IZ_C=\frac{j12\omega C e^{{-j\frac{3\pi}{4}}}}{1+ j\omega RC}*\frac{1}{j\omega C}= \frac{12e^{-j\frac{3\pi}{4}}}{1+j\omega RC}
% \end{align*}

% To derive the polar form,
% \begin{align}
% I = \frac{j12e^{-j\frac{3\pi}{4}}*10^{-3}}{1+j\sqrt{3}} = \frac{12e^{-j\frac{3\pi}{4}}e^{j\frac{\pi}{2}}*10^{-3}}{2e^{j\frac{\pi}{3}}}
% = 6e^{-j\frac{7\pi}{12}} \mbox{ mA}.
% \end{align}
% \begin{align}
% V=\frac{12e^{-j\frac{3\pi}{4}}}{1+j\omega RC} = \frac{12e^{-j\frac{3\pi}{4}}}{1+j\sqrt{3}}=\frac{12e^{-j\frac{3\pi}{4}}}{2e^{j\frac{\pi}{3}}}= 6e^{-j\frac{13\pi}{12}}  \mbox{ V}
% \end{align}
% }

\qitem \textbf{Step 5: Convert the system from standard to eigenbasis coordinates}

Since we found our eigenbasis basis $S = \{ \vec{v_1}, \vec{v_2} \},$ we will define the change of coordinates matrix from S-coordinates to standard basis coordinates as:

$$\vec{z} = \begin{bmatrix} z_1 \\ z_2 \end{bmatrix} {  ,  } \ \ \vec{x} = z_1 \vec{v_1} + z_2 \vec{v_2} \text{  and  } V = \begin{bmatrix} 1 & 1 \\ \lambda_1 C & \lambda_2 C \end{bmatrix} \text {  s.t.  } \vec{x} = V \vec{z} $$ 

How can we write out the differential equation using eigenbasis coordinates, or in terms of $\vec{z}(t)?$

\begin{equation}
	\ddt{}{t} \vec{z}(t) = D \vec{z}(t) + \widetilde{\vec{b}}
\end{equation}

\sol {
  We've defined the change of coordinates from eigenbasis to standard coordinates as $\vec{x} = V \vec{z}.$
  The vector differential equation that we currently have is:

  \begin{equation}
  	\ddt{}{t} \vec{x}(t) = A \vec{x}(t) + \vec{b}(t)
  \end{equation}

  We can substitute $\vec{x} = V \vec{z}$ to get:

  \begin{equation}
  	\ddt{}{t} V \vec{z}(t) = A V \vec{z}(t) + \vec{b}(t)
  \end{equation}

  Since we have a derivative of a linear combination, this is the same as a linear combination of the derivatives:

  \begin{equation}
  	V \ddt{}{t} \vec{z}(t) = A V \vec{z}(t) + \vec{b}(t)
  \end{equation}

  Then after applying $V^{-1}$ to both sides, we get:

  \begin{equation}
  	\ddt{}{t} \vec{z}(t) = V^{-1} A V \vec{z}(t) + V^{-1} \vec{b}(t)
  \end{equation}

  Therefore, we see that $D = V^{-1} A V$ and $\widetilde{\vec{b}} = V^{-1} \vec{b}.$

  We know the matrix $V^{-1} A V$ is diagonal with the eigenvalues of $A$ and we can compute $V^{-1}$
  $$V^{-1} = \frac{1}{\lambda_{2} C - \lambda_{1} C} \begin{bmatrix} \lambda_{2} C & -1 \\ -\lambda_{1} C & 1 \end{bmatrix}$$

  Therefore we can calculate $\widetilde{\vec{b}}$ as:
  $$V^{-1} \vec{b}(t) = \frac{1}{\lambda_{2} C - \lambda_{1} C} \begin{bmatrix} \lambda_{2} C & -1 \\ -\lambda_{1} C & 1 \end{bmatrix} \begin{bmatrix} 0 \\ \frac{V_s}{L} \end{bmatrix} = \frac{1}{\lambda_{2} C - \lambda_{1} C} \begin{bmatrix} -\frac{V_s}{L} \\ \frac{V_s}{L} \end{bmatrix} = \begin{bmatrix} -\frac{V_s}{LC(\lambda_{2} - \lambda_{1})} \\ \frac{V_s}{LC(\lambda_{2} - \lambda_{1})} \end{bmatrix}  $$

  Lastly, the initial condition will be:
  $$\vec{z}(0) = V^{-1} \vec{x}(0) = \begin{bmatrix} 0 \\ 0 \end{bmatrix}$$
}

\qitem \textbf{Step 6: Solve the differential equation in eigenbasis coordinates}

We can unroll our system of differential equations into a system of first-order differential equations:
  \begin{gather*}
    \ddt{}{t} z_{1}(t) = \lambda_1 z_{1}(t) - \frac{V_s}{LC(\lambda_{2} - \lambda_{1})} \\
    \ddt{}{t} z_{2}(t) = \lambda_2 z_{2}(t) + \frac{V_s}{LC(\lambda_{2} - \lambda_{1})}
  \end{gather*}

How can we solve these differential equations?

\sol{
  They are first-order differential equations with a constant input!
  We know the solutions will be:
  \begin{gather*}
    z_{1}(t) = -\frac{V_{s}}{LC(\lambda_{2} - \lambda_{1})\lambda_1} \big(e^{\lambda_{1}t} - 1 \big) \\
    z_{2}(t) = \frac{V_{s}}{LC(\lambda_{2} - \lambda_{1})\lambda_2} \big(e^{\lambda_{2}t} - 1 \big)
  \end{gather*}

  Therefore the solution in vector form will be:
  \begin{equation}
  \vec{z}(t) = \begin{bmatrix} z_{1}(t) \\ z_{2}(t) \end{bmatrix} =
  \begin{bmatrix} -\frac{V_s(e^{\lambda_{1} t} - 1)}{LC(\lambda_{2} - \lambda_{1})\lambda_{1}} \\ 
  \frac{V_s(e^{\lambda_{2} t} - 1)}{LC(\lambda_{2} - \lambda_{1})\lambda_{2}} \end{bmatrix}
  \end{equation}
}

\qitem \textbf{Step 7: Convert the solutions back into standard basis coordinates}

We have solved our differential equation in S-basis or $z_i$ coordinates.
However, our original differential equation was in terms of standard basis, or $x_i$ coordinates.
Convert the solutions that are currently in $z_i$ coordinates to a solution in $x_i$ coordinates.

\meta {
  Warning: The algebra here is very nasty. You may want to ask students to do it on their own time and give them the solution.
}

\sol {
  The relationship between $\vec{x}$ and $\vec{z}$ is $\vec{x} = V \vec{z}.$
  Therefore we see that the solution to our differential equation in $x_i$ coordinates is:
  $$\vec{x}(t) = \begin{bmatrix} 1 & 1 \\ \lambda_1 C & \lambda_2 C \end{bmatrix} \vec{z}(t) = 
  \begin{bmatrix} -\frac{V_s(e^{\lambda_{1} t} - 1)}{LC(\lambda_{2} - \lambda_{1})\lambda_{1}} + 
  \frac{V_s(e^{\lambda_{2} t} - 1)}{LC(\lambda_{2} - \lambda_{1})\lambda_{2}} \\ 
  -\frac{V_s(e^{\lambda_{1} t} - 1)}{L(\lambda_{2} - \lambda_{1})} + 
  \frac{V_s(e^{\lambda_{2} t} - 1)}{L(\lambda_{2} - \lambda_{1})} \end{bmatrix}   
  = \begin{bmatrix} -\frac{\lambda_{2} V_s (e^{\lambda_{1} t} - 1)}{LC(\lambda_{2} - \lambda_{1})\lambda_{1}\lambda_{2}} + 
  \frac{\lambda_{1} V_s (e^{\lambda_{2} t} - 1)}{LC(\lambda_{2} - \lambda_{1})\lambda_{1} \lambda_{2}} \\ 
  \frac{V_s(e^{\lambda_{2} t} - e^{\lambda_{1} t} + 1 - 1)}{L(\lambda_{2} - \lambda_{1})} \end{bmatrix} 
  $$

\meta {
  You can explain this by either looking at the product of the roots or by Vieta's Formulas.
  If students don't know what Vieta's Formulas are, then try to convince them that $\lambda_1 \lambda_2 = \frac{1}{LC}$ in some other way.
}

  A crucial realization that we can make is that $\lambda_1 \lambda_2 = \frac{1}{LC}$ therefore:
  $$\vec{x}(t) = \begin{bmatrix} \frac{-\lambda_{2} V_s (e^{\lambda_{1} t} - 1) + \lambda_{1} V_s (e^{\lambda_{2} t} - 1)}{\lambda_{2} - \lambda_{1}} \\ 
  \frac{V_s e^{\lambda_{2} t}}{L(\lambda_{2} - \lambda_{1})} - \frac{V_s e^{\lambda_{1} t}}{L(\lambda_{2} - \lambda_{1})} \end{bmatrix}
  = \begin{bmatrix} \frac{-\lambda_{2} V_s e^{\lambda_{1} t}}{\lambda_{2} - \lambda_{1}} + \frac{\lambda_{1} V_s e^{\lambda_{2} t}}{\lambda_{2} - \lambda_{1}} + \frac{\lambda_{2} V_s}{\lambda_{2} - \lambda_{1}} - \frac{\lambda_{1} V_s}{\lambda_{2} - \lambda_{1}} \\
  \frac{V_s e^{\lambda_{2} t}}{L(\lambda_{2} - \lambda_{1})} - \frac{V_s e^{\lambda_{1} t}}{L(\lambda_{2} - \lambda_{1})} \end{bmatrix}
  = \begin{bmatrix} \frac{-\lambda_{2} V_s e^{\lambda_{1} t}}{\lambda_{2} - \lambda_{1}} + \frac{\lambda_{1} V_s e^{\lambda_{2} t}}{\lambda_{2} - \lambda_{1}} + V_s \\
  \frac{V_s e^{\lambda_{2} t}}{L(\lambda_{2} - \lambda_{1})} - \frac{V_s e^{\lambda_{1} t}}{L(\lambda_{2} - \lambda_{1})} \end{bmatrix}
  $$ 

  Since we were initially solving for $x_{1}(t) = V_{\text{c}}(t) = V_{out}(t),$ our final result is:
  $$V_{out}(t) = \frac{-\lambda_{2} V_s e^{\lambda_{1} t}}{\lambda_{2} - \lambda_{1}} + \frac{\lambda_{1} V_s e^{\lambda_{2} t}}{\lambda_{2} - \lambda_{1}} + V_s$$
}

\qitem \textbf{Step 8: Analyze the eigenvalues}

The eigenvalues in step 4 were calculated using the quadratic formula: 
$$\lambda = -\frac{R}{2L} \pm \sqrt{\big(\frac{R}{L} \big)^2 - \frac{4}{LC}}$$
When are the eigenvalues of $A$ real, and when are they imaginary?

\sol {
  $\lambda$ will be real.
  When the value of $(\frac{R}{L})^2 - \frac{4}{LC}$ is negative, $\lambda$ will be real. 
  \begin{itemize}
  \item When the value of $(\frac{R}{L})^2 - \frac{4}{LC}$ is greater than or equal to zero, $\lambda$ must be real. 
  This occurs when $R > 2 \sqrt{\frac{L}{C}}.$
  \item When the value of $(\frac{R}{L})^2 - \frac{4}{LC}$ is negative, $\lambda$ must be complex. 
  This occurs when $R < 2 \sqrt{\frac{L}{C}}.$
  \item If the square root term is equal to 0, then there will be one distinct real eigenvalue. 
  We will discuss this case in more depth as a special part.
  \end{itemize}
}

\qitem \textbf{Step 9: The Real and Complex Responses}

The eigenvalues of the matrix $A$ determine the transient response of the RLC circuit. 
Consider both cases for which the eigenvalues are real, and for when they are complex.

\sol {
  If the eigenvalues are real, note that they will both be negative. 
  You can show this rigorously by observing that $-\frac{R}{2L}$ is negative, and
  $$\frac{1}{2} \sqrt{\big(\frac{R}{L}\big)^2 - \frac{4}{LC}} = \sqrt{\big(\frac{R}{2L}\big)^2 - \frac{1}{LC}} < \frac{R}{2L}$$
  Since $L$ and $C$ must be positive values. \\

  This means that the response will be a sum of decaying exponentials. \\

  However, when the eigenvalues are complex, notice that the real part of the exponential will still be negative.
  We can use the fact that for a complex number $z = a + bj,$
  \begin{equation}
    e^{z} = e^{a + bj} = e^{a} e^{bj} = e^{a} (\cos (b) + j \sin(b))
  \end{equation}

  

  Luckily, the two complex eigenvalues are complex conjugates of each other, so the final answer will be a linear combination of a decaying exponential multiplied with a cosine. The decaying exponential creates an envelope, and the cosine wave will bring out oscillations.
}

\qitem \textbf{Step 10: The critically damped response}

\meta {
  If students ask why we pick $\vec{v_2} = \begin{bmatrix} 0 \\ 1 \end{bmatrix}$ explain to them why this vector must be linearly independent to $\vec{v_1} = \begin{bmatrix} 1 \\ \lambda \end{bmatrix}.$ 
  Beyond that, $\vec{v_2}$ is considered a "generalized" eigenvector that makes the linear operator have upper triangular "Jordan" form, in this new basis.
  However, this is way out of scope of 16B, and it is enough to understand why $\vec{v_1}$ and $\vec{v_2}$ are linearly independent.
}

An issue arises when we have one distinct eigenvalue. This means it is no longer possible to convert our system into a basis made up of eigenvectors, since the eigenspace of $A$ is one dimensional. This can be seen by observing that the columns of $V$ will be linearly dependent. 

Therefore, we will convert our system to a basis where the linear operator will have an upper triangular matrix representation. To do this, we pick $\vec{v_2} = \begin{bmatrix} 0 \\ 1 \end{bmatrix}$ as our second basis vector. Therefore, 
the change of coordinates matrix will be:

$$ V = \begin{bmatrix} 1 & 0 \\ \lambda & 1 \end{bmatrix} $$

What is the solution $\vec{x}(t)$ in this case?

\sol {
  We start by computing $D = V^{-1} A V$
  \begin{align*} 
  D &= \begin{bmatrix} 1 & 0 \\ \lambda C & 1 \end{bmatrix}^{-1}
  \begin{bmatrix} 0 & \frac{1}{C} \\ -\frac{1}{L} & - \frac{R}{L} \end{bmatrix}
  \begin{bmatrix} 1 & 0 \\ \lambda C & 1 \end{bmatrix}
  = \begin{bmatrix} 1 & 0 \\ -\lambda C & 1 \end{bmatrix}
  \begin{bmatrix} 0 & \frac{1}{C} \\ -\frac{1}{L} & - \frac{R}{L} \end{bmatrix}
  \begin{bmatrix} 1 & 0 \\ \lambda C & 1 \end{bmatrix} \\
  &= \begin{bmatrix} 0 & \frac{1}{C} \\ -\frac{1}{L} & -\lambda - \frac{R}{L} \end{bmatrix}
  \begin{bmatrix} 1 & 0 \\ \lambda C & 1 \end{bmatrix}
  = \begin{bmatrix} \lambda & \frac{1}{C} \\ -\frac{1}{L} -\lambda^{2} C - \frac{RC}{L} & - \lambda - \frac{R}{L} \end{bmatrix}
  \end{align*}
  We make the observation that:
  $$-\frac{1}{L} -\lambda^{2} C - \frac{RC}{L} = -C(\lambda^{2} + \frac{R}{L} \lambda + \frac{1}{LC}) = 0$$
  In addition,
  $$-\lambda - \frac{R}{L} = \frac{R}{2L} - \frac{R}{L} = -\frac{R}{2L} = \lambda$$
  Therefore,
  $$D = \begin{bmatrix} \lambda & \frac{1}{C} \\ 0 & \lambda \end{bmatrix}$$
  We then calculate $\widetilde{\vec{b}}:$
  $$\widetilde{\vec{b}} = V^{-1} \vec{b} = \begin{bmatrix} 1 & 0 \\ -\lambda C & 1 \end{bmatrix} \begin{bmatrix} 0 \\ \frac{V_s}{L} \end{bmatrix} = \begin{bmatrix} 0 \\ \frac{V_s}{L} \end{bmatrix}$$
  Our initial condition will remain as:
  $$\vec{z}(0) = \begin{bmatrix} 0 \\ 0 \end{bmatrix}$$
  Therefore we can unroll our system of differential equations as:
  \begin{gather*}
  \ddt{}{t} z_{1}(t) = \lambda z_{1}(t) + \frac{1}{C} z_{2}(t) \\
  \ddt{}{t} z_{2}(t) = \lambda z_{2}(t) + \frac{V_s}{L} \\
  \end{gather*}
  With initial conditions:
  $$z_{1}(0) = 0, \ \  z_{2}(0) = 0$$
  We can now solve $z_{2}(t)$ since it is a first-order differential equation with a constant input. 
  We solve $z_{2}(t)$ first in a manner similar to back-substitution since $z_{1}(t)$ is currently in terms of both $z_{1}(t)$ and $z_{2}(t).$
  $$z_{2}(t) = \frac{V_s}{\lambda L} (e^{\lambda t} - 1)$$
  Now we plug our solution for $z_{2}(t)$ and try solving for $z_{1}(t).$ 
  Notice that our differential equation for $z_{1}(t)$ is now a first-order differential equation with non-constant input.
  $$\ddt{}{t} z_{1}(t) = \lambda z_{1}(t) + \frac{V_s}{\lambda LC} (e^{\lambda t} - 1)$$
  We can note that $u(t) = \frac{V_s}{\lambda LC} (e^{\lambda t} - 1)$ and solve accordingly.
  \begin{align*}
  z_{1}(t) &= \int\limits_{0}^t \frac{V_s}{\lambda LC} (e^{\lambda \tau} - 1) e^{\lambda(t - \tau)} d\tau = 
  e^{\lambda t} \frac{V_s}{\lambda LC} \int\limits_{0}^t (e^{\lambda \tau} - 1) e^{- \lambda \tau} d\tau 
  = e^{\lambda t} \frac{V_s}{\lambda LC} \big(\int\limits_{0}^t d\tau + \int\limits_{0}^t -e^{-\lambda \tau} d\tau \big) \\
  &= e^{\lambda t} \frac{V_s}{\lambda LC} (t + \frac{1}{\lambda}(e^{-\lambda t} - 1)) = \frac{V_s}{\lambda LC} t e^{\lambda t} - \frac{V_s}{\lambda^2 LC} e^{\lambda t} + \frac{V_s}{\lambda^2 LC}
  \end{align*}
  We make the observation that $LC = \frac{1}{\lambda^2}$ so our solution simplifies to:
  $$z_{2}(t) = \lambda V_s t e^{\lambda t} - V_s e^{\lambda t} + V_s$$

  Now that we have our solutions, we convert back into $x_i$ coordinates.
  $$\vec{x} = V \vec{z} = \begin{bmatrix} 1 & 0 \\ \lambda C & 1 \end{bmatrix} 
  \begin{bmatrix} \lambda V_s t e^{\lambda t} - V_s e^{\lambda t} + V_s \\ \frac{V_s}{\lambda L} (e^{\lambda t} - 1) \end{bmatrix}
  = \begin{bmatrix} \lambda V_s t e^{\lambda t} - V_s e^{\lambda t} + V_s \\ \lambda^2 V_s C t e^{\lambda t} - \lambda V_s C e^{\lambda t} + \lambda V_s C + \frac{V_s}{\lambda L} (e^{\lambda t} - 1) \end{bmatrix}
  $$
  Simplifying yields:
  $$\vec{x} = \begin{bmatrix} \lambda V_s t e^{\lambda t} - V_s e^{\lambda t} + V_s \\ \lambda^2 V_s C t e^{\lambda t} \end{bmatrix}$$
  Our final solution as before will be $x_{1}(t) = V_{\text{c}}(t) = V_{out}(t)$
  $$V_{out}(t) = \lambda V_s t e^{\lambda t} - V_s e^{\lambda t} + V_s$$
  The graph of this function looks like a decaying exponential with rate of decay faster than $\lambda.$
}







% \sol{
% \begin{align}
% i(t) = Ie^{j\omega t} + \overline{I}e^{-j\omega t} = 6 \cos (\omega t - \frac{7\pi}{12})  \mbox{ mA}
% \end{align}
% \begin{align}
% v_C(t)= Ve^{j\omega t} + \overline{V}e^{-j\omega t}= 6 \cos(\omega t -\frac{13\pi}{12}) \mbox{ V}
% \end{align}
% The phase difference between the two, with respect to $i(t)$ is $-\frac{\pi}{2}$
% }

% \ans{
% \begin{align}
% i(t) = \mathfrak{Re}[Ie^{j\omega t}] = \mathfrak{Re} [6e^{-j\frac{7\pi}{12}} e^{j\omega t}] = 6 \cos (\omega t - \frac{7\pi}{12})  \mbox{ mA}
% \end{align}
% \begin{align}
% v_C(t)=\mathfrak{Re}[V_Ce^{j\omega t}]=\mathfrak{Re}[6e^{-j\frac{13\pi}{12}}e^{j\omega t}]=6 \cos(\omega t -\frac{13\pi}{12}) \mbox{ V}
% \end{align}
% The phase difference between the two, with respect to $i(t)$ is $-\frac{\pi}{2}$
% }

\end{enumerate}